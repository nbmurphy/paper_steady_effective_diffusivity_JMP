\documentclass[english,12pt,jmp,graphicx]{revtex4-1}
%\documentclass[jmp,graphicx]{revtex4-1}
%\documentclass[aip,reprint]{revtex4-1}
\usepackage{microtype}

%\usepackage{mymacros}
%\usepackage{tgtermes}
\usepackage{amsfonts}
\usepackage{amsmath}
%\usepackage{amssymb}
\usepackage{graphicx}
\usepackage{mathrsfs}
%\usepackage{latexsym}





\newcommand{\Pen}{\mbox{\textit{Pe}}}  % Peclet number
\newcommand{\vecE}{\boldsymbol{E}}
\newcommand{\vecJ}{\boldsymbol{J}}
\newcommand{\vecw}{\boldsymbol{w}}
\newcommand{\vecu}{\boldsymbol{u}}
\newcommand{\vecx}{\boldsymbol{x}}
\newcommand{\vecv}{\boldsymbol{v}}
\newcommand{\vecxi}{\boldsymbol{\xi}}
\newcommand{\veczeta}{\boldsymbol{\zeta}}
\newcommand{\vecz}{\boldsymbol{z}}
%\newcommand{\vecg}{\boldsymbol{g}}
\newcommand{\vecg}{\mathfrak{g}}
\newcommand{\vece}{\boldsymbol{e}}
\newcommand{\thmref}[1]{Theorem~\ref{#1}}
\newcommand{\corref}[1]{Corollary~\ref{#1}}
\newcommand{\lemref}[1]{Lemma~\ref{#1}}
\newcommand{\secref}[1]{Section~\ref{#1}}
\newcommand{\appref}[1]{Appendix~\ref{#1}}
\newcommand{\figref}[1]{Fig.~\ref{#1}}
\newcommand{\numsecref}[1]{\ref{#1}}
\newcommand{\numfigref}[1]{\ref{#1}}
\newcommand{\bnabla}{\mbox{\boldmath${\nabla}$}}
\newcommand{\bGamma}{\mbox{\boldmath${\Gamma}$}}
\newcommand{\bcdot}{\mbox{\boldmath${\cdot}$}}
\newcommand{\btimes}{\mbox{\boldmath${\times}$}}
\newcommand{\Vc}{\mathcal{V}}
\newcommand{\Sg}{\mathfrak{S}}
\newcommand{\Ag}{\mathfrak{A}}
\newcommand{\Dg}{\mathfrak{D}}
\newcommand{\Hs}{\mathscr{H}}
\newcommand{\Hc}{\mathcal{H}}
\renewcommand{\d}{\mathrm{d}}
\newtheorem{theorem}{Theorem}
\newtheorem{corollary}[theorem]{Corollary}
\newtheorem{lemma}[theorem]{Lemma}
\newcommand{\Real}{\mbox{Re}\,} 
\newcommand{\Imag}{\mbox{Im}\,}
\newcommand{\bcolon}{\boldsymbol{:}}
\newcommand{\vecchi}{\boldsymbol{\chi}}
\newcommand{\vecr}{\boldsymbol{r}}
\newcommand{\hate}{\hat{\mathbf{e}}}
\newcommand{\Dm}{\mathsf{D}}
\newcommand{\Hm}{\mathsf{H}}
\newcommand{\Hmc}{\mathsf{H}}
\newcommand{\Um}{\mathsf{U}}
\newcommand{\Vm}{\mathsf{V}}
\newcommand{\Mm}{\mathsf{M}}
\newcommand{\Am}{\mathsf{A}}
\newcommand{\Bm}{\mathsf{B}}
\newcommand{\Cm}{\mathsf{C}}
\newcommand{\Zm}{\mathsf{Z}}
\newcommand{\Qm}{\mathsf{Q}}
\newcommand{\Wm}{\mathsf{W}}
\newcommand{\Pm}{\mathsf{P}}
\newcommand{\Gm}{\mathsf{G}}
\newcommand{\Rm}{\mathsf{R}}
\newcommand{\Ib}{\mathsf{I}}
\newcommand{\Om}{\mathsf{O}}
\newcommand{\Ec}{\mathcal{E}}
\newcommand{\Mb}{\mathbf{M}}
\newcommand{\Ab}{\mathbf{A}}
\newcommand{\Qb}{\mathbf{Q}}
\newcommand{\Qc}{\mathcal{Q}}




\draft % marks overfull lines with a black rule on the right

\begin{document}

\title{Spectral Analysis and Computation for Homogenization \\
  of Advection Diffusion Processes in Steady Flows} 
% \title{Spectral Analysis and Computation for Homogenization \\
%   of Advective Diffusion in Steady Flows}  

% \author{N.~B. Murphy, %\footnotemark[1]\ \footnotemark[2]  
% \and E.~Cherkaev, %\footnotemark[1]
% \and J.~Zhu, %\footnotemark[1]
% \and J.~Xin, %\footnotemark[2]
% \and K.~M.~Golden, %\footnotemark[1]
% }
% \author{N.~B.~Murphy   \footnotemark[1]\footnotemark[2],
%         E.~Cherkaev    \footnotemark[2],
%         J.~Zhu         \footnotemark[2],
%         J.~Xin         \footnotemark[1], and
%         K.~M.~Golden   \footnotemark[2]
%        }

\author{N.~B. Murphy}
\altaffiliation{University of Utah, Department of Mathematics, 155 S 1400
  E, RM 233, Salt Lake City, UT 84112-0090, USA}
\affiliation{University of California at Irvine, Department of
  Mathematics, 340 Rowland Hall, Irvine, CA 92697-3875, USA}

\author{E.~Cherkaev}
\affiliation{University of Utah, Department of Mathematics, 155 S 1400
  E, RM 233, Salt Lake City, UT 84112-0090, USA}

\author{J.~Zhu}
\affiliation{University of Utah, Department of Mathematics, 155 S 1400
  E, RM 233, Salt Lake City, UT 84112-0090, USA}

\author{J.~Xin}
\affiliation{University of California at Irvine, Department of
  Mathematics, 340 Rowland Hall, Irvine, CA 92697-3875, USA}

\author{K.~M.~Golden}
\affiliation{University of Utah, Department of Mathematics, 155 S 1400
  E, RM 233, Salt Lake City, UT 84112-0090, USA}

\date{\today}





% %  \renewcommand{\thefootnote}{\fnsymbol{footnote}}

%  \footnotetext[1]
%  {University of Utah, Department of Mathematics, 155 South 1400 East
%    Room 233, \\Salt Lake City, UT 84112-0090, USA
% % }
% % % %, Funding Information
% % % } 
%  \footnotetext[2]
%  {University of California at Irvine, Department of Mathematics, 340
%    Rowland Hall,  \\ Irvine, CA 92697-3875, USA
% % }  
%   % % , Funding Information
% % }

%\renewcommand{\thefootnote}{\arabic{footnote}}



\begin{abstract}
%
Advective diffusion plays a key role in the transport of salt, heat,
buoys and markers in geophysical flows, in the dispersion of
pollutants and trace gases in the atmosphere, and even in the dynamics
of sea ice floes influenced by winds and ocean currents. 
The long time, large scale behavior of such
systems is equivalent to an enhanced diffusion process with an
effective diffusivity matrix $\Dg^*$. Three decades ago a Stieltjes
integral representation for the homogenized matrix, involving a
spectral measure of a self adjoint operator, was developed. 
However, analytical calculations of $\Dg^*$ have been obtained for
only a few simple flows, and numerical computations of the effective
behavior based on this
spectral representation have apparently not
been attempted. We overcome these limitations by providing a
mathematical foundation for the
%rigorous
computation of Stieltjes integral
representations of $\Dg^*$. We explore two different approaches and
for each approach we derive new Stieltjes integral representations and
rigorous bounds for the symmetric and antisymmetric parts of $\Dg^*$,
involving the molecular diffusivity and a spectral measure $\mu$ of a
self-adjoint operator that depends on the characteristics of a
randomly perturbed periodic flow.  In
discrete formulations of each approach, we express $\mu$ explicitly in
terms of standard (or generalized) eigenvalues and eigenvectors of
Hermitian matrices. We develop and implement an efficient numerical
algorithm that combines beneficial numerical attributes of each approach.
%significantly reducing the algorithmic complexity in computations of
%spectral statistics.
We use this method to compute the effective behavior for model flows
and relate spectral characteristics to flow geometry and transport
properties.     
%
\end{abstract}



%\begin{center}
%\begin{keywords}%[class=AMS]
%\kwd[Primary ]{
%AMS subject classifications:  47B15, 65C60, 35C15, 76B99, 76M22,
%76M50, 76F25, 76R99 
%}
%\kwd{}
%\kwd[; secondary ]{}
%\end{keywords}
%\end{center}  


%\begin{center}
%\begin{keywords}
%  advective diffusion,  homogenization, effective diffusivity,
%  spectral measure, integral formula, generalized eigenvalue
%  computation
% \kwd{advective diffusion}
% \kwd{homogenization}
% \kwd{effective diffusivity}
% \kwd{spectral measure}
% \kwd{integral formula}
% \kwd{Fourier method}
% \kwd{generalized eigenvalue computation}
% \kwd{residual diffusion}
% %\kwd{}
%\end{keywords}
%\end{center}


\maketitle %\maketitle must follow title, authors, abstract and \pacs

%\tableofcontents
%\addtocontents{toc}{~\hfill\textbf{Page}\par}


\section{Introduction}\label{sec:Introduction}
%
The advective enhancement of diffusive transport of passive scalars by 
complex fluid flows plays a key role in many important processes in the global 
climate system~\cite{Washington:1986:9780935702521} and Earth's
ecosystems~\cite{Lorenzo:2013:26:4}. Advection by geophysical
fluids intensifies the dispersion and large scale transport of
heat~\cite{Moffatt:RPP:621},
pollutants~\cite{Samson:1988:88009978},
and nutrients~\cite{Lorenzo:2013:26:4,Hofmann:ANS:2004:265075} diffusing
in their environment. Advective processes also enhance the large scale
transport of plankton~\cite{Hofmann:ANS:2004:265075}, which is an
important component of the food web that sustains life in the polar
oceans.  The transport of sea ice floes in the polar oceans 
is diffusive in nature over short time scales as floes bump 
against each other, yet can be influenced and enhanced by eddie fluxes,
storms and prevailing winds in the atmosphere as well as ocean 
currents~\cite{Watanabe:2009JPO4010,Rampal:2008:JGRC10653,%
Rampal:2009:JGRC11312,Lukovich:2011:JGRC12087,Washington:1986:9780935702521}. 
Thermal transport \textit{through} sea ice, coupling the temperature field
in the upper ocean to the lower atmosphere, can be enhanced 
due to the presence of convective fluid flow in the porous brine microstructure
of sea ice \cite{Lytle:JGR-8853,trodahl2001thermal,Golden:GRL:L16501,Kraitzman_MMS_2019}.

%is driven by a seasonally and regionally changing balance in oceanic and
%atmospheric forces~\cite{Lukovich:2011:JGRC12087,Washington:1986:9780935702521}. 
%These forces
%are vorticity dominated and
%can enhance the transport of sea ice floes by eddie fluxes in the atmosphere and ocean currents
%~\cite{Watanabe:2009JPO4010,Rampal:2008:JGRC10653,Rampal:2009:JGRC11312,Lukovich:2011:JGRC12087}. 
%as well as intense jets through straits and
%canyons~\cite{Watanabe:2009JPO4010}.    



% Sea ice is a porous composite of ice, air, and brine which couples
% the atmosphere to the polar
% oceans~\cite{Washington:1986:9780935702521}. Thermal transport 
% through sea ice can be enhanced by a brine velocity
% field~\cite{Lytle:JGR-8853,Trodahl:2001:TCS}, which itself depends on
% the brine microstructure and its
% connectivity~\cite{Golden:S-2238,Golden:GRL:L16501}. Fluid flow
% through sea ice brine microstructure mediates a broad range of 
% biogeochemical processes including nutrient replenishment for algal
% and bacterial communities living in the ice~\cite{Golden:NAMS:2009},
% which are at the base of the food web in the polar oceans, feeding
% various planktonic organisms. Fluid flow through sea ice also mediates
% brine drainage, snow-ice formation (where sea water percolates upward,
% floods the surface, and subsequently freezes), and the evolution of
% melt ponds on the surface of Arctic sea
% ice~\cite{Hohenegger:TC-2012,Golden:NAMS:2009}. Melt ponds determine
% sea ice albedo --- a key parameter in climate
% modeling~\cite{Washington:1986:9780935702521}. Despite its importance, this parameter remains
% a major source of uncertainty in climate models.
% In fact, the lack of inclusion of melt ponds in previous generations
% of climate models is believed to partially account for 
% the inadequacy of these models to predict the dramatic rate
% of melting of the summer Arctic sea ice pack. The results here help
% to advance our understanding of the effective properties of the
% ice pack, and provide a rigorous path toward incorporating sea ice
% processes into climate models.
 
  


% Complex interactions between ocean waves, tidal currents, and
% wind drift, for example, give rise to complex
% %random
% behavior of the flow
% fields~\cite{Young:JPO:1982:515,Kullenberg:1972:TUS1529,Bowden:JFM:1965}.
It was discovered in the early 1900s~\cite{Taylor:1915:523} 
that complex fluid flows transport passive scalars in much the same
way as molecular diffusion. The mathematical description of
this phenomenon~\cite{Taylor:PRSL:196} demonstrated that the long
time, large scale behavior of a diffusing particle or tracer being
advected by an \emph{incompressible} fluid velocity field is equivalent to an
enhanced diffusion process with an effective diffusivity matrix
$\Dg^*$. Describing the enhancement of the effective transport
properties by
%randomly fluctuating
fluid advection is a challenging
problem with theoretical and practical importance in many fields of
science and engineering, ranging from turbulent
combustion~\cite{Aslanyan:BF00790149,Bilger:05:10.1016,Tabaczynski:1990:243,%
Williams:1985:TC:9781611971064,Peters:2000:TC:9780521660822,Xin:2009:Fronts:9780387876832}
to mass, heat, and salt transport in geophysical flows~\cite{Moffatt:RPP:621}. 


A broad range of mathematical techniques have been
developed that reduce the analysis of complex fluid flows, with
rapidly varying structures in space and time, to solving averaged or
\emph{homogenized} equations that do not have rapidly varying data,
but involve effective
parameters~\cite{McLaughlin:SIAM_JAM:780,Biferale:PF:2725,Fannjiang:1994:SIAM_JAM:333,Fannjiang:1997:1033,Pavliotis:PHD_Thesis,Majda:Kramer:1999:book,Majda:1994:10.1088,Xin:2009:Fronts:9780387876832}. Homogenization
of the advection diffusion equation for passive scalar transport by
stochastically stationary,
time-independent, \emph{mean-zero} fluid velocity fields was treated
in~\cite{McLaughlin:SIAM_JAM:780}. This reduced the analysis of
advective diffusion to solving a diffusion equation involving a
homogenized temperature field and a constant effective diffusivity
matrix~$\Dg^*$. 



An important consequence of this analysis is that the effective
diffusivity $\Dg^*$ is given in terms of a solution of the so-called
``cell problem''~\cite{McLaughlin:SIAM_JAM:780}, which can be written
as a steady state diffusion equation involving a skew-symmetric random
matrix $\Hm$~\cite{Avellaneda:PRL-753,Avellaneda:CMP-339,Fannjiang:1994:SIAM_JAM:333,Fannjiang:1997:1033}.  
By adapting the analytic continuation method of homogenization theory
for composite materials~\cite{Golden:CMP-473,Bergman:PRL-1285,Milton:APL-300}, it was
shown~\cite{Avellaneda:CMP-339,Avellaneda:PRL-753} that the cell
problem could be written in resolvent form involving a
self-adjoint random operator acting on the Hilbert space of
\emph{curl-free vector fields}. This, in turn, led to a Stieltjes
integral representation for the \emph{symmetric} part of $\Dg^*$,
involving the P{\'e}clet number $\Pen$ of the flow and a
\emph{spectral measure} of the operator.  A key feature of the
integral representation for $\Dg^*$ is that parameter information in
$\Pen$ is \emph{separated} from the geometry of the fluid velocity
field. The velocity field geometry 
is encoded in the spectral measure through its
moments. This parameter separation has
led~\cite{Avellaneda:CMP-339,Avellaneda:PRL-753,Fannjiang:1994:SIAM_JAM:333,Fannjiang:1997:1033}
to rigorous forward bounds for the diagonal components of $\Dg^*$. 




The mathematical framework developed in~\cite{McLaughlin:SIAM_JAM:780}
was adapted~\cite{Pavliotis:PHD_Thesis,Majda:Kramer:1999:book} to the
case of a periodic, time-dependent 
%incompressible 
fluid velocity field with \emph{non-zero mean}. It was
shown~\cite{Pavliotis:PHD_Thesis} that, depending 
on the strength of the fluctuations relative to the mean flow, the
effective diffusivity matrix $\Dg^*$ can be constant or a function of
both space and time. When $\Dg^*$ is constant, only its symmetric part
appears in the homogenized equation as an enhancement in the
diffusivity. However, when $\Dg^*$ is a function of space and time,
its antisymmetric part also plays a key role in the homogenized
equation. Based on~\cite{Bhattacharya:AAP:1999:951,Bhattacharya:1989:ASD}, the cell problem
associated with a \emph{time-independent} flow was
transformed~\cite{Pavliotis:PHD_Thesis} into a resolvent formula 
involving a self-adjoint operator, acting on a Sobolev
space~\cite{McOwen:2003:PDE,Folland:95:PDEs} of spatially periodic
\emph{scalar fields}, which is also a Hilbert space. This, in turn,
led to a discrete Stieltjes integral representation for the
\emph{antisymmetric} part of $\Dg^*$, involving the P{\'e}clet number
of the flow and a spectral measure of the operator.     



Such methods have been extended to steady flows where particles
diffuse according to linear collisions~\cite{Pavliotis:IMAJAM:951},
solute transport in porous media~\cite{Bhattacharya:AAP:1999:951,Bhattacharya:1989:ASD},
anelastic (weakly compressible)
flows~\cite{McLaughlin:Forest:PF:1999:880}, as well as to the
setting of a time-dependent fluid velocity
field~\cite{Murphy:ADSTPF-2017,Avellaneda:PRE:3249,Majda:Kramer:1999:book}. All
these approaches yield integral 
representations of the symmetric and, when appropriate, the
antisymmetric part of $\Dg^*$. Variational formulations of
the effective parameter problem for $\Dg^*$ are given
in~\cite{Fannjiang:1994:SIAM_JAM:333,Fannjiang:1997:1033,Avellaneda:CMP-339}. 





We now discuss the main contributions of the present work.
In~\cite{Avellaneda:PRL-753,Avellaneda:CMP-339}, a Stieltjes integral 
representation and associated rigorous bounds were developed for
the diagonal components of $\Dg^*$ (which are components of the
\emph{symmetric} part of $\Dg^*$); the off-diagonal components were
ignored. In~\cite{Pavliotis:PHD_Thesis}, using a different method, a
Stieltjes integral representation was developed for the off-diagonal
components of the \emph{antisymmetric} part of $\Dg^*$ (the diagonal
components are zero); no bounds were developed using the spectral 
representation. Here we adapt and extend the mathematical frameworks
developed in both~\cite{Avellaneda:PRL-753,Avellaneda:CMP-339}
and~\cite{Pavliotis:PHD_Thesis} to the case of a time-independent
randomly perturbed periodic flow. We obtain \emph{new} Stieltjes
integral representations and rigorous bounds for both the symmetric
and antisymmetric parts of $\Dg^*$ -- for both its diagonal and
off-diagonal components. These integral representations 
achieve a separation of the geometry of the fluid velocity -- 
through the spectral measures of self-adjoint operators -- 
from the relevant material property of the fluid, its 
molecular diffusivity.



%We now discuss why we considered both of these different approaches. 
%in the first place. 
For each approach, we provide a
mathematical foundation for the computation of Stieltjes integral
representations for $\Dg^*$, by developing
discrete, matrix formulations of the effective parameter problem. The
spectral measure in each of these two approaches is given explicitly
in terms of the eigenvalues and eigenvectors of either a standard or
generalized Hermitian eigenvalue problem. In general, the
numerical implementation of the generalized eigenvalue problem is more
computationally expensive than a standard eigenvalue 
problem~\cite{Parlett:1980}. However, the approach involving the
standard problem involves a matrix that is larger by a
factor of the dimension $d$ compared to the matrix of the other
approach involving the generalized problem. The numerical
%operation of
eigenvalue decomposition of a matrix is quite expensive,
and high resolution of the discretized domain results in very large
matrices. Therefore, it is important to develop an efficient way to
numerically compute the spectral measure. We provide a detailed matrix
analysis that demonstrates both approaches can be formulated in terms
of a common \emph{standard} eigenvalue problem involving a matrix with
the \emph{smaller size} encountered in the generalized eigenvalue problem,
thus combining beneficial numerical attributes of both
approaches. 









In the continuum setting, the self-adjoint operators used
in~\cite{Avellaneda:PRL-753,Avellaneda:CMP-339}
and~\cite{Pavliotis:PHD_Thesis} both involve the inverse of the negative
Laplacian $(-\Delta)^{-1}$, which is given in terms of convolution with
the Green's function for
$\Delta=\nabla^2$~\cite{Stakgold:BVP:2000}. In the discrete setting,
the negative matrix Laplacian is full-rank, hence invertible when
Dirichlet boundary conditions are considered, for example. However,
the matrix Laplacian is rank-deficient, hence non-invertible when
periodic boundary conditions are considered. The matrix analysis
discussed above in the full-rank setting 
%is elegant in the full-rank setting, 
reveals useful 
structure with minimal effort. However, the matrix analysis of the
rank-deficient setting is quite a bit more involved, but necessary,
since here we investigate advection enhanced diffusion by periodic
flows. This analysis demonstrates the two approaches considered yield
equivalent spectral representations for the effective diffusivity
matrix $\Dg^*$.



We utilize this unified mathematical framework to compute the
effective diffusivity matrix $\Dg^*$ for some model periodic and
randomly perturbed periodic flows, and describe the behavior of the
enhancement of diffusive transport in terms of the behavior of the
spectral measure. There are several approaches to computing $\Dg^*$,
including numerical solution of the underlying cell problem
PDE~\cite{Pavliotis:PHD_Thesis}, Monte Carlo
methods~\cite{Bonn:McLaughlin:JFM:2001:345}, and a method accurate 
for large {P}\'eclet numbers~\cite{Gorb:Nam:Novikov:2746477}. 
Our work here was not motivated by a goal of finding a faster
or more accurate method of computing $\Dg^*$, although our
approach does provide an alternative way of computing $\Dg^*$
and is quite robust.
%and generally applicable.

Instead, for randomly perturbed periodic flows, our spectral method for computing
$\Dg^*$ is set apart from more traditional methods in that it was
developed not only to study homogenized behavior but 
to investigate \textit{spectral statistics}.
Indeed, the effective behavior of the 
system, as encapsulated by $\Dg^*$, is closely connected to the statistical
behavior of the spectral measure which, in turn, is determined by the
statistics of the random eigenvalues and
eigenvectors. Consequently, the spectral method enables the
homogenization of advection diffusion processes 
to be viewed through the lens of
random matrix theory. In the theory of two-phase composite materials, this
approach has led to a new understanding of critical behavior of
transport in high contrast composite materials
as a percolation threshold is approached.
The focus on calculating spectral measures in Stieltjes representations
for composite materials
through computation of eigenvalues and eigenvectors of random matrices
led to the realization that critical behavior of classical transport
at a percolation threshold
can be viewed as a type of Anderson transition \cite{Murphy:PRL:118:036401}.
In the analysis enabled by the spectral approach for composites, 
we investigated the appearance, for example, of localization phenomena,
mobility edges, and universal Wigner-Dyson statistics
of the Gaussian Orthogonal Ensemble
for the eigenvalue spacing distribution.
The results in this
manuscript lay the mathematical and computational 
groundwork for such investigations in the
context of advection diffusion processes, where the geometry of the fluid velocity 
field plays the role of the microstructural properties of the composite.
Our results in
this direction are somewhat outside the scope of the current paper 
and will be published elsewhere, although we consider here
some spectral characteristics and their relations to 
flow geometry and transport properties.     

As another example of how the present work enables applications to geophysics,
we consider advection diffusion within sea ice.
The enhancement of the effective thermal conductivity of sea ice 
due to the presence of convective fluid flow has 
long been known from an observational perspective
\cite{Lytle:JGR-8853,trodahl2001thermal,Golden:GRL:L16501}.
The authors are unaware, though, of any predictive, theoretical works on 
this enhancement.
In \cite{Kraitzman_MMS_2019} the 
effective thermal conductivity of sea ice in the presence 
of {\it bulk} fluid convection is investigated, 
by applying the results developed here.
%, which is based on the approach initiated in
%\cite{Avellaneda:CMP-339,Avellaneda:PRL-753}. 
%in \cite{avellaneda1989stieltjes,murphy2018spectral,murphy2017spectral}. 
Using Stieltjes integral representations, a series of 
bounds on the effective thermal conductivity were obtained for model flows
using Pad$\acute{\text{e}}$ approximants and the analytic 
continuation method, in terms of the P$\acute{\text{e}}$clet number. 




Motivated by the theoretical findings in the current work,
in~\cite{Murphy:ADSTPF-2017} we generalized the results given here to
the setting of a time-dependent fluid velocity field
$\vecu=\vecu(t,\vecx)$. Furthermore we used different, abstract
methods of functional analysis to generalize the equivalence results
summarized in \thmref{thm:Equivalence}, \corref{cor:Equivalence},
\lemref{lem:Spectral_Equivalence}, and
\thmref{thm:Spectral_Equivalence_Rank_Def} of the current work to the
continuum, steady and dynamic settings.
% We are currently exploring the
% extension of these methods to the time-stochastic setting, which is
% relevant to atmospheric and oceanic flows. 



In order to streamline the presentation leading to
the numerical results, we have placed in an appendix the development
of integral representations for effective diffusivities using the approach
introduced in~\cite{Avellaneda:PRL-753,Avellaneda:CMP-339} --- for both
the continuum and discrete settings. The matrix analysis of the rank
deficient setting and the equivalence results discussed above have
also been placed in the appendix. Comments on the notation used
throughout this manuscript are given in \appref{app:Notation}.





% \subsection{Synopsis of the paper}
% %

% In \secref{sec:Homogenization}, we formulate the effective parameter
% problem for advection enhanced diffusion by random, incompressible
% flows. In particular, we review the problem of homogenizing the
% advection diffusion equation for such flows, developed
% in~\cite{McLaughlin:SIAM_JAM:780}, which yields a rigorous functional
% representation for the effective diffusivity matrix $\Dg^*$ in terms
% of the solution of a random ``cell problem'' which arises in the
% homogenization procedure.  










% In \appref{app:Functinals_curl-free}, we discuss how the
% incompressibility of the fluid velocity field allows it to be
% expressed in terms of the divergence of an 
% antisymmetric random matrix. This, in turn, allows the cell problem
% to be written as a steady state diffusion equation involving a
% curl-free random field. Moreover, we
% demonstrate how this leads to functional formulas for the
% symmetric and antisymmetric parts of the effective diffusivity matrix
% $\Dg^*$, involving the curl-free field and a self-adjoint random
% operator. In \appref{app:Integral_Reps_Curl_free}, we discuss how the
% cell problem can be transformed into a resolvent formula involving
% this self-adjoint operator, which acts on the Hilbert space of
% curl-free, random \emph{vector
%   fields}~\cite{Avellaneda:PRL-753,Avellaneda:CMP-339}. We then 
% discuss how this result and the spectral theorem~\cite{Stone:64,Reed-1980} 
% yields Stieltjes integral representations for both the
% symmetric~\cite{Avellaneda:PRL-753,Avellaneda:CMP-339} and
% antisymmetric parts of $\Dg^*$, involving a \emph{spectral measure} of  
% the operator. We utilize analytical properties of the Stieltjes
% integrals to provide rigorous bounds for the components
% of these homogenized matrixs.




% In \appref{app:Matrix_Formulation_Curl}, we provide a mathematical
% foundation for rigorous computation of $\Dg^*$ associated with this
% approach. In particular, a discrete representation of the cell 
% problem involving a Hermitian random matrix leads to Stieltjes integral
% representations for the symmetric and antisymmetric parts of $\Dg^*$,
% involving a discrete spectral measure which is given explicitly in
% terms of the eigenvalues and eigenvectors of the matrix. In
% \thmref{thm:Projection_Method}, we develop a projection method which
% allows spectral statistics to be obtained by diagonalizing a much
% smaller random matrix, greatly increasing the efficiency of numerical
% computations.  




% In \secref{sec:Integral_Reps_Sobolev}, we formulate another approach
% to the effective parameter problem~\cite{Pavliotis:PHD_Thesis}, which
% is different from the approach discussed in
% \appref{app:Matrix_Formulation_Curl}. We demonstrate this approach also
% provides Stieltjes integral representations for the symmetric and
% antisymmetric parts of $\Dg^*$. In particular, we transform the cell   
% problem into a resolvent formula involving a self-adjoint random 
% operator acting on a Sobolev space of random \emph{scalar fields},
% which is also a Hilbert space. This leads to functional formulas for
% $\Dg^*$ and a resolvent formula for the cell problem. This, in turn,
% leads to Stieltjes integral representations for the symmetric and
% antisymmetric parts of 
% $\Dg^*$, involving a spectral measure of the random operator.



% The symmetry of the random operator depends intimately
% on the Sobolev-type inner-product in this approach. Consequently, its
% matrix formulation is substantially different from that of
% \appref{app:Matrix_Formulation_Curl}. This technical 
% difficulty is overcome in \secref{sec:Matrix_Sobolev} by casting the
% effective parameter problem in terms of a \emph{generalized} eigenvalue
% problem, which has the Sobolev-type inner-product as a key
% feature. This leads to Stieltjes integral representations for the
% symmetric and antisymmetric parts of $\Dg^*$, involving a discrete
% spectral measure which is given explicitly in terms of the associated 
% generalized eigenvalues and eigenvectors.


% The inverse Laplacian operator is central to both of the continuum
% formulations of the 
% effective parameter problems described in \secref{sec:Integral_Reps_Sobolev} and~\appref{app:Curl_Free}. Consequently, the matrix
% Laplacian is also central to both of the discrete formulations
% described in \secref{sec:Matrix_Sobolev} and~\appref{app:Matrix_Formulation_Curl}, 
% which is assumed to be of full-rank so that it is invertible. Given
% this condition, we demonstrate in \appref{app:Discrete_Equivalence}
% that the two discrete formulations of the effective parameter problems
% given in \secref{sec:Matrix_Sobolev} and~\appref{app:Matrix_Formulation_Curl} produce equivalent spectral
% representations for the effective diffusivity matrix $\Dg^*$. 




% In \appref{app:Eigenvalue_method}, we generalize the mathematical
% frameworks formulated in \secref{sec:Matrix_Sobolev} and
% \appref{app:Matrix_Formulation_Curl} to the case of a rank-deficient matrix
% Laplacian. This is important in computations of the effective
% diffusivity matrix $\Dg^*$ for randomly perturbed periodic flows, as
% the matrix Laplacian with periodic boundary conditions is rank
% deficient. A detailed matrix analysis shows that both approaches yield
% equivalent spectral representations of $\Dg^*$ and that the spectral
% statistics can be obtained from a standard eigenvalue problem of a
% much smaller random matrix. When the matrix Laplacian is of full-rank,
% these generalized formulations reduce to the formulations in
% \secref{sec:Matrix_Sobolev} and \appref{app:Matrix_Formulation_Curl}. 


% In \secref{sec:Num_Results} we numerically compute the spectral
% measures and effective diffusivity matrixs for several non-random
% periodic flows as well as randomly perturbed periodic flows. Our
% computations are in excellent agreement with theoretical
% results~\cite{Avellaneda:CMP-339} for shear flow. We relate spectral
% characteristics of our computations to flow geometry and transport
% properties.      







\section{Homogenization of the advection diffusion equation}
\label{sec:Homogenization}
%
The dispersion of a passive scalar with density $\phi$ diffusing in a
fluid with molecular diffusivity $\varepsilon$ and being advected by a 
mean-zero incompressible velocity field $\vecu$ satisfying
$\bnabla\bcdot\vecu=0$ and $\langle\vecu\rangle=0$ is described by the
advection diffusion equation         
%
\begin{align}\label{eq:ADE}
  \phi_t(t,\vecx\,)=\vecu(\vecx)\bcdot\bnabla \phi(t,\vecx\,)+\varepsilon\Delta\phi(t,\vecx\,),
  \quad
  \phi(0,\vecx)=\phi_0(\vecx),
  \qquad 
  \vecx\in\mathbb{R}^d,
  \quad t>0,
\end{align}
%
with initial density $\phi_0(\vecx)$ given. Here,
$\langle\cdot\rangle$ denotes volume averaging over the period cell 
$\Vc$, $\phi_t$ denotes partial
differentiation of $\phi$ with respect to time $t$,
$\Delta=\bnabla\bcdot\bnabla =\nabla^2$ is the Laplacian, $\varepsilon>0$, $d$ is
the system dimension, and we denote by $0$ the null element in all
linear spaces in question. Moreover,
$\vecxi\bcdot\veczeta=\vecxi^{\,\dagger}\veczeta$ and $\dagger$ is the operation
of complex-conjugate-transpose, with
$\vecxi\bcdot\vecxi=|\vecxi|^2$. Later, we will extensively use this
form of the dot product over complex fields, with built in complex
conjugation. However, we emphasize that 
all quantities considered in this section are \emph{real-valued}. We
assume for now that the time-independent fluid velocity field $\vecu$
is spatially periodic on the region $\Vc\subset\mathbb{R}^d$. Later, we will
discuss the case of an array of randomly perturbed, periodic flows.
%
% The random paths of the particles are
% determined~\cite{Fannjiang:1997:1033} by the stochastic differential
% equation   
% %
% \begin{align}
%   \d\vecx(t)=\vecu(\vecx(t))\d t+\sqrt{2\varepsilon}\;\d\vecrmW(t),
%   \quad
%   \vecx(0)=\vecx_0,
% \end{align}
% %
% with the initial tracer particle location $\vecx_0$ given and
% $\vecrmW(t)$ is standard Brownian motion (the Wiener process).
%
% Moreover, it is assumed~\cite{McLaughlin:SIAM_JAM:780} that
% the fluid velocity field $\vecv(\vecx)$ is stationary and that the
% correlation functions $R_{ij}(z)=\langle v_i(\vex+\vecz)\,v_j(\vecx)\rangle$ have a
% Fourier transform that is continuous at the origin.
%




The long time, large scale dispersion of passive scalars can be
described~\cite{Taylor:PRSL:196} by an effective diffusivity
matrix $\Dg^*$. An explicit representation for $\Dg^*$ can be found
using methods of homogenization
theory~\cite{Bensoussan:Book:1978,Papanicolaou:RF-835}. These methods
have demonstrated that the averaged or \emph{homogenized} behavior of
the advection diffusion equation in~\eqref{eq:ADE} is determined by a
diffusion equation involving an averaged scalar density $\bar{\phi}$ and
a (constant) effective diffusivity matrix 
$\Dg^*$~\cite{McLaughlin:SIAM_JAM:780}   
%
\begin{align}\label{eq:phi_bar}
  \bar{\phi}_t(t,\vecx)=\bnabla\bcdot[\Dg^*\bnabla \bar{\phi}(t,\vecx)], \quad
  \bar{\phi}(0,\vecx)=\phi_0(\vecx).
\end{align}
%
The components $\Dg^*_{jk}$, $j,k=1,\ldots,d$, of $\Dg^*$ are given
by~\cite{McLaughlin:SIAM_JAM:780}   
%
\begin{align}\label{eq:Djk}
  \Dg^*_{jk}=\varepsilon\delta_{jk}+\langle u_j\chi_k\rangle,
\end{align}
%
The
%random
function $\chi_j$ in~\eqref{eq:Djk} satisfies a cell problem which is a
steady state advection diffusion equation with a forcing term
involving $u_j$, the $j$th component of the fluid velocity field
$\vecu$~\cite{McLaughlin:SIAM_JAM:780,Fannjiang:1997:1033},    
%
\begin{align}\label{eq:Random_Cell_Prob}
  \vecu\bcdot\bnabla\chi_j+\varepsilon\Delta\chi_j=-u_j\,,
  \quad
  \langle\bnabla \chi_j\rangle=0.
\end{align}
%
  




Equations~\eqref{eq:phi_bar}--\eqref{eq:Random_Cell_Prob}
follow from the assumption that the length scale associated with
spatial variations of the initial density $\phi_0$ is much larger than
the length scale of spatial variations associated with the fluid
velocity field
$\vecu$~\cite{McLaughlin:SIAM_JAM:780,Fannjiang:1997:1033} (separation
of scales). This  
information is incorporated into equation~\eqref{eq:ADE} by
introducing a small dimensionless parameter $\delta\ll1$ and
writing~\cite{McLaughlin:SIAM_JAM:780}    
%
\begin{align}
  \phi(0,\vecx)=\phi_0(\delta\vecx). 
\end{align}
%
Anticipating that $\phi$ will have diffusive dynamics as $t\to\infty$, space and 
time are rescaled by $\vecx\mapsto\vecx/\delta$ and $t\mapsto t/\delta^2$. As
$\delta\to0$, the associated solution $\phi^\delta(t,\vecx)=\phi(t/\delta^2,\vecx/\delta)$ of
equation \eqref{eq:ADE} (in the rescaled variables) converges to
$\bar{\phi}(t,\vecx)$ which satisfies equation~\eqref{eq:phi_bar}. The
convergence is in an $L^2$ sense that depends on the technical
assumptions made about the
%random
fluid velocity field
$\vecu$~\cite{McLaughlin:SIAM_JAM:780,Avellaneda:CMP-339,Fannjiang:1994:SIAM_JAM:333,Fannjiang:1997:1033,Pavliotis:PHD_Thesis,Majda:Kramer:1999:book}.   




We emphasize that the cell problem in~\eqref{eq:Random_Cell_Prob}
involves only the fast variable $\vecx/\delta$~\cite{McLaughlin:SIAM_JAM:780}.  
Other space-time scalings have also been considered, which have led
to space-time dependent $\Dg^*$~\cite{Pavliotis:PHD_Thesis} and even
anomalous diffusive
dynamics~\cite{Majda:Kramer:1999:book}. Homogenization theorems for
space-time dependent fluid velocity fields are treated 
in~\cite{Biferale:PF:2725,Majda:Kramer:1999:book,Pavliotis:PHD_Thesis}.     



In our analysis of the effective diffusivity matrix $\Dg^*$, it is
beneficial to use non-dimensional parameters. We therefore assume 
that equation~\eqref{eq:ADE} has been non-dimensionalized as
follows. Let $\ell$ and $\tau$ be typical length and time scales associated
with the problem of interest. Mapping to the non-dimensional variables
$t\mapsto t/\tau$ and $\vecx\mapsto \vecx/\ell$, one finds that $\phi$ satisfies the
advection diffusion equation in \eqref{eq:ADE} with a non-dimensional
molecular diffusivity and fluid velocity field,
%
\begin{align}\label{eq:Peclet_eps}
  \varepsilon\mapsto\tau\varepsilon/\ell^{\,2},
  \quad
  \vecu\mapsto\tau\,\vecu/\ell.    
\end{align}
%


This demonstrates by non-dimensionalizing equation \eqref{eq:ADE}, the
fluid velocity field $\vecu$ is, in turn, divided by a quantity with dimensions
of velocity and the molecular
diffusivity is divided by a quantity with dimensions of velocity
multiplied by spatial length. It is convenient to choose the rescaled
$\vecu$ and $\varepsilon$ in a way that captures information about the
fluid velocity field. However, it is also convenient to choose
these rescaled variables in a way that \emph{separates} the rescaled
$\varepsilon$ from the \emph{geometry} of the flow; this leads to
mathematically and physically meaningful 
properties of rigorous bounds for $\Dg^*$ which follow from the
analytic structure of Stieltjes integral representations for
$\Dg^*$~\cite{Avellaneda:CMP-339,Baker:1996:Book:Pade} --- discussed in 
\secref{sec:Integral_Reps_Sobolev} below.



We accomplish both of these goals as follows. Define the
dimensional fluid velocity field by $\vecu=u_0\vecv$, where the
parameter $u_0$ has dimensions of velocity and represents the
``\emph{flow strength}'' of $\vecu$ which is independent of the
geometry of the flow; the flow geometry is encapsulated in the
non-dimensional vector field $\vecv$. With these definitions,
we choose reference scales $\tau$ and $\ell$ in
equation~\eqref{eq:Peclet_eps} to satisfy $u_0=\ell/\tau$ so that
$\vecu\mapsto\vecv$ and $\varepsilon\mapsto\varepsilon/u_0\,\ell\,$. For
example, in \secref{sec:Num_Results} we compute $\Dg^*$ for 
$BC$-flow~\cite{Biferale:PF:2725} having dimensional fluid velocity field 
$\vecu=u_0\,(C\cos{y},B\cos{x})$, where the flow strength
$u_0\in(0,\infty)$ is independent of the non-dimensional parameters
$B,C\in[0,1]$ which determine the streamline geometry of
$\vecu$.


An example of a non-dimensional, parameter that compares the rate of
scalar advection to the rate of diffusion is the P\'{e}clet number.
%$\Pen=\ell u_0/\varepsilon$~\cite{McLaughlin:Forest:PF:1999:880,Majda:Kramer:1999:book,Majda:Kramer:1999:book}.
% $\Pen$.
We define it by the ratio $\Pen=\ell u_0/\varepsilon$, although
other definitions have been
used~\cite{McLaughlin:Forest:PF:1999:880,Majda:Kramer:1999:book,Majda:Kramer:1999:book}.
The
advection and diffusion dominated regimes are characterized by
$\Pen\gg1$ and $\Pen\ll1$, respectively.  Therefore, our choice of the 
rescaled $\varepsilon$ satisfies $\Pen=1/\varepsilon$.



The \emph{parameter separation} between $\Pen$ and the
geometry of the flow is important for  rigorous upper and
lower Pad\'{e} approximant bounds for
$\Dg^*$~\cite{Avellaneda:CMP-339,Murphy_Pade_Bounds}. Pad\'{e} 
approximants of $\Dg^*$ are given in terms of ratios of
polynomials~\cite{Baker:1996:Book:Pade} $P(z)/Q(z)$, where $z=\Pen^2$,
$0<z<\infty$, and the coefficients of these polynomials depend
on the moments of a \emph{spectral measure} that, in turn, depend on
the fluid velocity field
$\vecu$~\cite{Avellaneda:CMP-339,Murphy_Pade_Bounds}. For example,
when $\vecu$ is given by $BC$-flow the moments of the measure
depend~\cite{Murphy_Pade_Bounds} on the parameters $B$ and $C$. Our
numerical investigations have shown if the non-dimensionalization of
equation~\eqref{eq:ADE} is chosen in a way that the variable $z$ also
depends on the flow geometry through the ratio $B/C$, then this gives 
rise~\cite{Murphy_Pade_Bounds} to \emph{positive real} roots for the
polynomials $P(z)$ and $Q(z)$. This, in turn, gives rise to positive
real roots and
poles in the (rigorous) Pad\'{e} approximant bounds for
$\Dg^*$, which is not physically or mathematically consistent with the
known behavior of
$\Dg^*$~\cite{Fannjiang:1994:SIAM_JAM:333,Pavliotis:PHD_Thesis,Biferale:PF:2725,Majda:Kramer:1999:book}. 
This demonstrates the importance of \emph{parameter separation}
between $z$ and the flow geometry for Pad\'{e} 
approximant bounds for $\Dg^*$.





This way of non-dimensionalizing equation~\eqref{eq:ADE} is also
convenient in the case of a time-dependent fluid velocity
field~\cite{Murphy:ADSTPF-2017}, where the parameter $u_0$ again
represents the flow strength and the vector field $\vecv$ encapsulates 
the \emph{geometric and dynamical} properties of the flow. For example, the
space-time periodic flow with velocity field
$\vecu=u_0(\,(C\cos{y},B\cos{x})+\cos{t}\,(\gamma\sin{y},\beta\sin{x})\,)$
has dynamical behavior exhibiting Lagrangian
chaos~\cite{Biferale:PF:2725,Murphy:ADSTPF-2017}. Here, the flow 
strength $u_0\in(0,\infty)$ is independent of the parameters
$B,C,\gamma,\beta\in[0,1]$ which determine the geometric and dynamical
properties of $\vecu$. This choice of non-dimensionalization gives a
clearer interpretation of the advection and diffusion dominated
regimes in terms of $\Pen=1/\varepsilon$ than that given
in~\cite{Murphy:ADSTPF-2017}.  A detailed discussion of various
non-dimensionalizations of equation \eqref{eq:ADE}
% involving the Strouhal number, the P{\'e}clet number, and the
%periodic P{\'e}clet number
is given in~\cite{McLaughlin:Forest:PF:1999:880,Majda:Kramer:1999:book}.





% In the case of a time-independent, spatially periodic flow, a natural choice
% for $\ell$ and $\tau$ is, respectively, the maximum cell period and
% $\tau=\ell/\langle|\vecu|^2\rangle^{1/2}$, so that~\eqref{eq:Peclet_eps} is given by 
% $\varepsilon\mapsto\varepsilon/(\ell\,\langle|\vecu|^2\rangle^{1/2})$ and $\vecu\mapsto\vecu/\langle|\vecu|^2\rangle^{1/2}$. In
% this case, a natural definition of the P{\'e}clet number $\Pen$ is
% %$\Pen=\ell \langle|\vecu|^2\rangle^{1/2}/\varepsilon$,
% %
% \begin{align}\label{eq:Peclet}
%   \Pen=\frac{\ell \langle|\vecu|^2\rangle^{1/2}}{\varepsilon},
% \end{align}
% %
% so that the scaled molecular diffusivity satisfies
% $\varepsilon=\Pen^{-1}$. However, we will see the spectral measure
% $\mu$ at the heart of integral representations for $\Dg^*$ depends
% explicitly on the fluid velocity field $\vecu$. Therefore, if the
% rescaled $\varepsilon$ also depends on $\vecu$, then $\mu$ and
% $\varepsilon$ are \emph{coupled}. This limits the utility of bounds
% for $\Dg^*$ that follow from these integrals. For this reason, we
% define $\ell$ to be the maximum cell period and arbitrarily define
% $\tau=1$ second. A detailed discussion of various
% non-dimensionalizations 
% %involving the Strouhal number, the P{\'e}clet number, and the
% %periodic P{\'e}clet number
% is given in~\cite{McLaughlin:Forest:PF:1999:880,Majda:Kramer:1999:book}.
% In \secref{sec:Num_Results}, we will discuss the  
% scalings in equation~\eqref{eq:Peclet_eps} in more detail, where we
% discuss computation of the effective diffusivity matrix $\Dg^*$ for
% various fluid flows of interest and study the asymptotic behavior of
% $\Dg^*$ in the  advection dominated regime, where $\varepsilon\to0$.


% One may also introduce non-dimensional parameters by writing the
% advection diffusion equation in~\eqref{eq:ADE} in terms of the
% Strouhal $S$ and P{\'e}clet $\Pen$ numbers~\cite{Pavliotis:PHD_Thesis} 
% %
% \begin{align}\label{eq:ADE_S_Pe}
%   S\,\phi_t=\vecu\bcdot\bnabla \phi+\Pen^{-1}\Delta\phi.
%   % \quad
% %   \phi(0,\vecx)=\phi_0(\vecx),
% %   \qquad
% %   \vecx\in\mathbb{R}^d,
% %   \quad t>0,
% \end{align}
% %
% The cell problem in~\eqref{eq:Random_Cell_Prob}
% becomes~\cite{Avellaneda:PRL-753,Avellaneda:CMP-339,Pavliotis:PHD_Thesis}  
% %
% \begin{align}\label{eq:Random_Cell_Prob_S_Pe}
%   \vecu\bcdot\bnabla\chi_j+\Pen^{-1}\,\Delta\chi_j=-u_j.
%  %  \quad
% %   \langle\bnabla \chi_j\rangle=0.
% \end{align}
% %
% One can see from comparing equations~\eqref{eq:ADE} and
% \eqref{eq:Random_Cell_Prob} with~\eqref{eq:ADE_S_Pe} and
% \eqref{eq:Random_Cell_Prob_S_Pe}, respectively, that the advection dominated
% regime, i.e., $\varepsilon\to0$, corresponds to the large P{\'e}clet number limit
% $\Pen\to\infty$, with $\Pen\sim\varepsilon^{-1}$. In the following sections, we will focus
% on the formulation given by
% equations~\eqref{eq:ADE}--\eqref{eq:Random_Cell_Prob} and briefly
% comment on how the key results are modified when using the formulation
% given in
% equations~\eqref{eq:ADE_S_Pe} and \eqref{eq:Random_Cell_Prob_S_Pe}. 



\section{Hilbert space and integral representations} \label{sec:Integral_Reps_Sobolev}   
%
In this section, we adapt and extend a
method~\cite{Bhattacharya:AAP:1999:951,Bhattacharya:1989:ASD,Pavliotis:PHD_Thesis} which
provides Stieltjes integral representations for the effective
diffusivity matrix $\Dg^*$. We do so by providing functional formulas
for the symmetric $\Sg^*$ and antisymmetric $\Ag^*$ parts of $\Dg^*$,
involving the 
%random
scalar field $\chi_j$ in equation~\eqref{eq:Random_Cell_Prob}. We also
provide a Sobolev space formulation of the effective parameter
problem~\cite{Pavliotis:PHD_Thesis} which yields a resolvent formula
for $\chi_j$, involving a self-adjoint 
%random
operator that depends only on the fluid velocity field $\vecu$. This
and the spectral theorem~\cite{Stone:64,Reed-1980} yield Stieltjes
integral representations for $\Sg^*$ and $\Ag^*$ involving a spectral
measure of the operator.            
%




% As in \appref{app:Integral_Reps_Curl_free}, here we consider either a
% periodic flow or a $n\times n$ 
% periodic array of randomly perturbed periodic
% flows~\cite{Fannjiang:1997:1033}. In this case, averaging $\langle\cdot\rangle$ is 
% over the period cell $\Vc\subset\mathbb{R}^d$. and the effective diffusivity
% matrix $\Dg^*$ associated with the random fluid velocity field $\vecu$
% is obtained by taking an infinite volume limit,
% $\Dg^*=\lim_{n\to\infty}\Dg^*_n$, of the finite volume effective diffusivity
% matrix $\Dg^*_n$ and evoking an ergodic
% theorem~\cite{Fannjiang:1997:1033,Golden:CMP-473}. For notational
% simplicity, we will not distinguish between $\Dg^*_n$ and $\Dg^*$.  




% In contrast
% to \appref{app:Integral_Reps_Curl_free}, which defined the effective
% matrix $\Dg^*$ in terms of the \emph{vector field}
% $\bnabla\chi_j$, here we define $\Dg^*$ in terms of the
% \emph{scalar field} $\chi_j$.



Consider the Hilbert space $\Hs$, 
%
\begin{align}\label{eq:Hilbert_Sobolev}
  \Hs=\{f\in L^2(\Vc,\nu)\,
              :\, f(\vecx)~\text{is periodic in} \ \Vc
              \text{ and } \langle f \rangle=0\},
\end{align}
%
where $\nu$ is the Lebesgue measure on $\mathbb{R}^d$, restricted
to $\Vc$, and the 
$\sigma$-algebra associated with the underlying measure space is
generated by the Lebesgue measurable subsets of $\mathbb{R}^d$.
The Hilbert space $\Hs$ is equipped with a sesquilinear inner product
$\langle\cdot,\cdot\rangle$ defined by $\langle f,h\rangle=\langle\,\overline{f}\;h\,\rangle$, with
$\langle h,f\rangle=\overline{\langle f,h\rangle}$ for $\,f,h\in\Hs$, which induces a norm $\|\cdot\|$
defined by $\|f\|=\langle f,f\rangle^{1/2}$ and $f\in\Hs$ implies
$\|f\|<\infty$.




One could also consider a random fluid flow filling all of
$\mathbb{R}^d$, with a velocity field $\vecu$ determined by the
probability space $(\Omega,P)$ with $\sigma$-algebra generated by the sets
$\{\vecu(\vecx)\in B\}$, where $\vecx\in\mathbb{R}^d$ and $B$ is a Borel subset
of $\mathbb{R}^d$~\cite{Avellaneda:CMP-339}. Here, $\Omega$ is the set of
all geometric realizations of $\vecu$, which is indexed by the
parameter $\omega\in\Omega$ representing one particular geometric realization, and
$P$ is the associated probability measure. The underlying Hilbert
space in this case can be taken to be $\Hs=L^2(\Omega,P)$, i.e., the space
of all $P$-measurable complex-valued scalar functions $\xi$ satisfying
$\|\xi\|=\langle|\xi|^2\rangle^{1/2}<\infty$, where
$\langle\cdot\rangle$ denotes ensemble 
averaging and the underlying sesquilinear inner-product is defined by
$\langle\xi,\zeta\rangle=\langle\,\overline{\xi} \ \zeta\rangle$ . In this case, one could
consider a random fluid flow with a velocity field $\vecu$ that is
stationary~\cite{McLaughlin:SIAM_JAM:780} or
ergodic~\cite{Avellaneda:PRL-753,Avellaneda:CMP-339}, for example,
with regularity conditions at infinity, i.e., as $|\vecx|\to\infty$. In these
cases, one works with an infinite medium directly, which presents
substantial computational difficulties. 



A more computationally tractable random system is given by a $n\times n$
array of randomly perturbed periodic
flows~\cite{Fannjiang:1997:1033}. In this case, the $\sigma$-algebra 
associated with the underlying probability space is generated by the
Lebesgue measurable subsets of $\mathbb{R}^d$. Here, the Hilbert space
$\Hs$ is given by equation~\eqref{eq:Hilbert_Sobolev} and averaged
quantities depend on the realization of the random medium because
$\langle\cdot\rangle$ is given by volume averaging over the period cell
$\Vc$~\cite{Fannjiang:1997:1033}. The 
effective diffusivity matrix $\Dg^*$ is obtained by taking an infinite
volume limit, $\Dg^*=\lim_{n\to\infty}\Dg^*_n\,,$ of the finite volume effective
diffusivity matrix $\Dg^*_n$ and evoking an ergodic
theorem~\cite{Fannjiang:1997:1033,Golden:CMP-473}. Numerically, by the
law of large numbers~\cite{Durrett:Book:Probability}, it is
natural to spatially average each statistical trial and then ensemble
average over all of the sampled random realizations. This is the
approach we take here.





In any case, once the Hilbert space $\Hs$ is established, with
associated average $\langle\cdot\rangle$, inner-product
$\langle\cdot,\cdot\rangle$, and norm $\|\cdot\|$, the 
spectral theory presented in the remainder of this section progresses
independent of the underlying details, as it lays on an axiomatic
foundation~\cite{Stone:64}. For the sake of numerical 
tractability, we will assume that the Hilbert space
$\Hs$ is given by equation~\eqref{eq:Hilbert_Sobolev}. The fluid
velocity field $\vecu$ can be assumed to represent a periodic or
randomly perturbed periodic flow. Now, consider the Sobolev space
$\Hs^{1,2}$, which itself is a Hilbert 
space~\cite{Bhattacharya:AAP:1999:951,Bhattacharya:1989:ASD},      
%
\begin{align}\label{eq:Sobolev}
  \Hs^{1,2}=\{f\in\Hs\,:\,\|f\|_{1,2}<\infty,\; \langle f\rangle=0\},  \qquad
  \|f\|_{1,2}=\langle|\bnabla f|^2\rangle^{1/2},
\end{align}
%
where $\|\cdot\|_{1,2}$ is the norm induced by the underlying sesquilinear
inner-product $\langle\cdot,\cdot\rangle_{1,2}$ defined by
$\langle f,h\rangle_{1,2}=\langle\bnabla f\bcdot\bnabla h\rangle\,$, with
$\langle h,f\rangle_{1,2}=\overline{\langle f,h\rangle}_{1,2}\,$
(recall $\vecxi\bcdot\veczeta=\vecxi^\dagger\veczeta\,$ includes
complex conjugation).  





Recall the definition of the components
$\Dg^*_{jk}=\varepsilon\delta_{jk}+\langle u_j\chi_k\rangle$, $j,k=1,\ldots,d$, of the effective
diffusivity matrix $\Dg^*$ 
in~\eqref{eq:Djk}. Rewrite the functional $\langle u_j\chi_k\rangle$
as~\cite{Pavliotis:PHD_Thesis},      
%
\begin{align}\label{eq:uj_chik}
  \langle u_j\chi_k\rangle=\langle[\Delta\Delta^{-1}u_j]\chi_k\rangle
       =-\langle\bnabla \Delta^{-1}u_j\bcdot\bnabla \chi_k\rangle
       =-\langle\Delta^{-1}u_j,\chi_k\rangle_{1,2}\,,
%       =\varepsilon\langle\chi_j,\chi_k\rangle_{1,2}+\langle A\chi_j,\chi_k\rangle_{1,2}\,,
\end{align}
%
where we have integrated by parts and used the periodicity of the
functions $u_j$ and $\chi_k$. Here, $\Delta^{-1}$ is based on
convolution with the Green's function for the Laplacian
$\Delta$~\cite{Stakgold:BVP:2000}.
The symmetric $\Sg^*$ and antisymmetric $\Ag^*$ parts of the
effective diffusivity matrix $\Dg^*$ are defined by 
%
\begin{align}\label{eq:Symm_Anti-Symm}
  \Dg^*=\Sg^*+\Ag^*,\qquad
  \Sg^*=\frac{1}{2}\left(\Dg^*+[\Dg^*]^{\,T}\right), \quad
  \Ag^*=\frac{1}{2}\left(\Dg^*-[\Dg^*]^{\,T}\right).
\end{align}
%
Substituting into equation~\eqref{eq:uj_chik} the expression
$-u_j=\vecu\bcdot\bnabla\chi_j+\varepsilon\Delta\chi_j$ for $-u_j$
in~\eqref{eq:Random_Cell_Prob} yields the following functional 
formulas for the components $\Sg^*_{jk}$ and $\Ag^*_{jk}$,
$j,k=1,\ldots,d$,
of $\Sg^*$ and $\Ag^*$
%
\begin{align}\label{eq:Eff_Diffusivity_Sobolev}
  \Sg^*_{jk}=\varepsilon(\delta_{jk}+\langle\chi_j,\chi_k\rangle_{1,2})\,, \qquad
  \Ag^*_{jk}=\langle A\chi_j,\chi_k\rangle_{1,2}\,, \quad
  A=\Delta^{-1}[\vecu\bcdot\bnabla]\,.
\end{align}
%

Due to the incompressibility of
the fluid velocity field, $\bnabla\bcdot\vecu=0$, the
%random
operator
$A$ is antisymmetric on $\Hs^{1,2}$, i.e., $\langle
A\xi,\zeta\rangle_{1,2}=-\langle \xi,A\zeta\rangle_{1,2}$ for all 
$\xi,\zeta\in\Hs^{1,2}$ (see equation~\eqref{eq:Anti-sym_Sobolev}).
% We emphasize that 
% the operator $A$ is independent of $\varepsilon$.
Since the 
%random,
scalar fields $\chi_j$ and 
$A\chi_j$ are \emph{real-valued}, we have that $\langle\chi_j,\chi_k\rangle_{1,2}=\langle\chi_k,\chi_j\rangle_{1,2}$ and
$\Ag^*_{kj}=\langle A\chi_k,\chi_j\rangle_{1,2}=-\langle\chi_k,A\chi_j\rangle_{1,2}=-\langle A\chi_j,\chi_k\rangle_{1,2}=-\Ag^*_{jk}$, 
confirming that $\Sg^*$ is symmetric and $\Ag^*$ is antisymmetric,
with $\Ag^*_{kk}=0$. 



Applying the operator $\Delta^{-1}$ to both sides of the cell problem 
$\varepsilon\Delta\chi_j+\vecu\bcdot\bnabla\chi_j=-u_j$ in
equation~\eqref{eq:Random_Cell_Prob} yields the following resolvent
formula for $\chi_j$ involving the operator $A$
in~\eqref{eq:Eff_Diffusivity_Sobolev},      
%
\begin{align}\label{eq:Resolvent_Rep_Sobolev}
  \chi_j=(\varepsilon+A)^{-1}g_j\,, \qquad 
  %A=\Delta^{-1}(\vecu\bcdot\bnabla -\partial_t), \quad
  g_j=(-\Delta)^{-1}u_j\,.
\end{align}
%
Substituting the resolvent formula for $\chi_j$
in~\eqref{eq:Resolvent_Rep_Sobolev} into the bilinear functionals  in 
equation~\eqref{eq:Eff_Diffusivity_Sobolev} yields  
%
\begin{align}\label{eq:Eff_Diffusivity_Resolvent_Sobolev}
 \Sg^*_{jk}&=\varepsilon(\,\delta_{jk}+\langle(\varepsilon+A)^{-1}g_j,(\varepsilon+A)^{-1}g_k\rangle_{1,2}\,)\,, 
 \\
 \Ag^*_{jk}&=\langle A\,(\varepsilon+A)^{-1}g_j,(\varepsilon+A)^{-1}g_k\rangle_{1,2}\,.
 \notag
\end{align}
%




Since $\Vc$ is a bounded domain, the linear operator $\Delta^{-1}$ is
bounded on $\Hs$ with bounded operator norm
$\|\Delta^{-1}\|<\infty$~\cite{Stakgold:BVP:2000}. When $|\vecu|$ is
uniformly bounded on the period cell $\Vc$,
$\sup_{\vecx\in\Vc}|\vecu(\vecx)|<\infty$,
% %
% \begin{align}\label{eq:Bounded_u}  
%   \sup_{\vecx\in\Vc}|\vecu(\vecx)|^2<\infty,
% \end{align}
% %
the linear operator $A$ is bounded on $\Hs^{1,2}$,
with (see equations~\eqref{eq:First_Bound_A} and~\eqref{eq:Second_Bound_A})
%
\begin{align}\label{eq:Bounded_A}
  \|A\|_{1,2}\leq\big[\;\|\Delta^{-1}\|\,\sup_{\vecx\in\Vc}|\vecu(\vecx)|^2\;\big]^{1/2}<\infty\,,
\end{align}
%
where $\|A\|_{1,2}$ denotes the operator norm of $A$ induced by the norm
$\|\cdot\|_{1,2}$ in equation~\eqref{eq:Sobolev}.
All of the fluid velocity fields that we consider in our numerical
computations of $\Dg^*$ in \secref{sec:Num_Results} are uniformly bounded. 
More generally, for $u_k\in\Hs$, $k=1,\ldots,d$, the operator $A$ is
compact on
$\Hs^{1,2}$~\cite{Bhattacharya:AAP:1999:951,Bhattacharya:1989:ASD,Pavliotis:PHD_Thesis},
hence bounded~\cite{Stakgold:BVP:2000}. It follows that $M=-\imath A$,
where $\imath=\sqrt{-1}$, is a
bounded linear operator on $\Hs^{1,2}$ with
$\|M\|_{1,2}=\|A\|_{1,2}<\infty$. Moreover, the skew-symmetry of $A$ and the
sesquilinearity of the $\Hs^{1,2}$-inner-product imply that $M$ is also
symmetric,
$\langle M f,h\rangle_{1,2}=\langle f,M h\rangle_{1,2}\,.$ The bounded linear symmetric operator $M$ with domain
$\Hs^{1,2}$ is \emph{self-adjoint}~\cite{Stone:64,Reed-1980}.







The spectrum $\Sigma$ of the self-adjoint 
operator $M$ is real-valued, with spectral
radius equal to its operator norm~\cite{Reed-1980}, i.e.,
%$\Sigma\subseteq[-\|\Hm\|,\|\Hm\|\,]$.
%
\begin{align}\label{eq:Spectral_Radius_Sobolev}
  \Sigma\subseteq[-\|A\|_{1,2}\,,\|A\|_{1,2}\,].
\end{align}
%
The spectral theorem for bounded linear self-adjoint operators in
Hilbert space states that there is a one-to-one correspondence between
the operator $M$ and a family of self-adjoint projection operators
$\{Q(\lambda)\}_{\lambda\in\Sigma}$ --- the resolution of the identity ---
satisfying~\cite{Stone:64} 
%$\lim_{\lambda\to\,\inf{\Sigma}}Q(\lambda)=0$ and $\lim_{\lambda\to\,\sup{\Sigma}}Q(\lambda)=I$~\cite{Stone:64}.
%
\begin{align}\label{eq:Res_Identity}
  \lim_{\lambda\to\,\inf{\Sigma}}Q(\lambda)=0,
  \quad
  \lim_{\lambda\to\,\sup{\Sigma}}Q(\lambda)=I.
\end{align}
%
Furthermore, for all  
$\xi,\zeta\in\Hs^{1,2}$, the \emph{complex-valued} function of the
spectral variable $\lambda$
defined by $\mu_{\xi\zeta}(\lambda)=\langle
Q(\lambda)\xi,\zeta\,\rangle_{1,2}$ has real and 
imaginary parts that are of bounded variation~\cite{Stone:64}. Therefore, there
is a \emph{complex} Stieltjes measure $\mu_{\xi\zeta}$ associated with
$\mu_{\xi\zeta}(\lambda)$~\cite{Stieltjes:1995,Stone:64,Folland:99:RealAnalysis}.




The spectral theorem also states, for all complex-valued
functions $f\in L^2(\mu_{\xi\xi})$ and $h\in L^2(\mu_{\zeta\zeta})$,
there exist linear operators denoted by $f(M)$ and $h(M)$ which
are defined in terms of the bilinear functional
$\langle f(M)\,\xi,\,h(M)\,\zeta\,\rangle_{1,2}$~\cite{Stone:64}. In
particular, this functional has the following Radon--Stieltjes
integral representation involving the Stieltjes measure
$\mu_{\xi\zeta}$, for all $\xi,\zeta\in\Hs^{1,2}$,       
% 
\begin{align}\label{eq:Spectral_Theorem}  
  \langle f(M)\xi,\,h(M)\,\zeta\,\rangle_{1,2}=
  \int_{-\infty}^\infty\overline{f}(\lambda)\,h(\lambda)\,\d\mu_{\xi\zeta}(\lambda),
  \quad
  \mu_{\xi\zeta}(\lambda)=\langle Q(\lambda)\xi,\zeta\rangle_{1,2}\,,
  %\quad
  %f,h\in L^2(\mu_{\xi\zeta}),
\end{align}
%
where the integration is over the spectrum $\Sigma$ of
$M$~\cite{Reed-1980,Stone:64} and $\overline{f}$ denotes complex
conjugation of the scalar function $f$. Since the Stieltjes measure
$\mu_{\xi\zeta}$ has the property~\cite{Stone:64}
$\int_a^b\d\mu_{\xi\zeta}(\lambda)=\mu_{\xi\zeta}(b)-\mu_{\xi\zeta}(a)$,
equation~\eqref{eq:Res_Identity} implies that the mass
$\mu^0_{\xi\zeta}$ of the measure $\mu_{\xi\zeta}$ is given by
% 
\begin{align}\label{eq:Mass_General}
  \mu^0_{\xi\zeta}
  =\int_{-\infty}^\infty\d\mu_{\xi\zeta}(\lambda)
  =\int_{-\infty}^\infty\d\langle Q(\lambda)\xi,\zeta\,\rangle_{1,2}
  =\langle\xi,\zeta\rangle_{1,2}\,,
\end{align}
%
which is bounded in the sense that
$|\mu^0_{\xi\zeta}|\leq\|\xi\|_{1,2}\,\|\zeta\|_{1,2}<\infty$ for
all $\xi,\zeta\in\Hs^{1,2}$.
%, where $\|\cdot\|$ denotes the norm induced by the
%$\Hc$-inner-product~\cite{Stone:64}.
Due to the sesquilinearity of
the inner-product and the fact that the projection operator $Q(\lambda)$
is self-adjoint on $\Hs^{1,2}$, the complex-valued function $\mu_{\zeta\xi}(\lambda)$
satisfies $\mu_{\zeta\xi}(\lambda)=\overline{\mu}_{\xi\zeta}(\lambda)$ and 
$\mu_{\xi\xi}(\lambda)=\|Q(\lambda)\xi\|^2$, so $\mu_{\xi\xi}$ is a
\emph{positive} measure. The formulas in
equations~\eqref{eq:Res_Identity}--\eqref{eq:Mass_General} will be
used several times throughout this manuscript to clarify and
streamline the development of various Stieltjes integral
representations for
the effective diffusivity matrix $\Dg^*$. They will be used both in
the continuum setting, as well as the discrete setting which leads to
an efficient numerical algorithm for our computation of spectral
representations for $\Dg^*$ for various model fluid flows in
\secref{sec:Num_Results}.






We are now ready to present the main results of this section. For
notational simplicity, denote the complex-valued function
$\mu_{jk}(\lambda)=\langle Q(\lambda)g_j,g_k\rangle$, instead of
$\mu_{g_jg_k}(\lambda)$, where $g_j=(-\Delta)^{-1}u_j$ is defined
in~\eqref{eq:Resolvent_Rep_Sobolev}. Denote the real and imaginary
parts of the function $\mu_{jk}(\lambda)$ by  $\Real\mu_{jk}(\lambda)$
and $\Imag\mu_{jk}(\lambda)$ respectively.  
%
\begin{theorem}\label{thm:Int_Rep_Sobolev}
%
Let $Q(\lambda)$ denote the resolution of the identity corresponding
to the self-adjoint operator $M$. Then, the components $\Sg^*_{jk}$
and $\Ag^*_{jk}$, 
$j,k=1,\ldots,d$, of the symmetric $\Sg^*$ and antisymmetric $\Ag^*$
parts of the effective diffusivity matrix $\Dg^*$ have the following
Stieltjes-Radon integral representations 
%
\begin{align}\label{eq:Integral_Rep_kappa*}
  \Sg^*_{jk}=\varepsilon\left(\delta_{jk}
             +\int_{-\infty}^\infty
             \frac{\d\Real\mu_{jk}(\lambda)}{\varepsilon^2+\lambda^2}\right),
  \quad
  \Ag^*_{jk}=\int_{-\infty}^\infty\frac{\lambda\, \d\Imag\mu_{jk}(\lambda)}{\varepsilon^2+\lambda^2}\,.
\end{align}
%
For $j\ne k$, $\Real\mu_{jk}$ and $\Imag\mu_{jk}$ are signed
measures~\cite{Folland:99:RealAnalysis}. For $j=k$, 
$\Real\mu_{kk}=\mu_{kk}$ is a positive
measure~\cite{Folland:99:RealAnalysis}, which demonstrates that the
effective transport of the scalar density $\phi$ is enhanced by
the presence of an incompressible fluid velocity field $\vecu$, above
the bare diffusive value $\varepsilon$
%
\begin{align}\label{eq:Skk}
  \Sg^*_{kk}=\varepsilon\left(1
             +\int_{-\infty}^\infty
             \frac{\d\mu_{kk}(\lambda)}{\varepsilon^2+\lambda^2}\right)
           \ge\varepsilon\,.
\end{align}
%
The mass $\mu^0_{jk}$ of the measure $\mu_{jk}$ is real-valued and satisfies
%
\begin{align}\label{eq:Mass_Sobolev}
  \mu^0_{jk}  %=\langle g_j,g_k\rangle_{1,2}
        %=\langle\bnabla\Delta^{-1}u_j\bcdot\bnabla\Delta^{-1}u_k\rangle 
        =\langle[(-\Delta)^{-1}u_j]\,u_k\rangle,
        \quad
        |\mu^0_{jk}|\leq\|\Delta^{-1}u_j\|\,\|u_k\|<\infty.
%        |\mu^0_{jk}|\leq\|\Delta^{-1}\|\,\|u_j\|\,\|u_k\|<\infty.
\end{align}
%
%  
\end{theorem}




\noindent
\textbf{Proof of \thmref{thm:Int_Rep_Sobolev}}.
We first provide integral representations for the bilinear functional
formulas in equation~\eqref{eq:Eff_Diffusivity_Resolvent_Sobolev} for
$\Sg^*_{jk}$ and $\Ag^*_{jk}$. Since we have 
already established  the operator $M$ is self-adjoint, we only
need to identify the appropriate functions $f(\lambda)$ and
$h(\lambda)$ as well as the Hilbert space members
$\xi,\zeta\in\Hs^{1,2}$ in~\eqref{eq:Spectral_Theorem}.  Using $A=\imath M$, 
comparison of the functionals  in
equations~\eqref{eq:Eff_Diffusivity_Resolvent_Sobolev}
and~\eqref{eq:Spectral_Theorem} identifies
$f(\lambda)=h(\lambda)=(\varepsilon+\imath\lambda)^{-1}$ for the first
formula in~\eqref{eq:Eff_Diffusivity_Resolvent_Sobolev}, while
$f(\lambda)=\imath\lambda(\varepsilon+\imath\lambda)^{-1}$ and
$h(\lambda)=(\varepsilon+\imath\lambda)^{-1}$ for the 
second formula, with $\xi=g_j$ and $\zeta=g_k$ for both of these
formulas.



Now we just need to verify $f,h\in L^2(\mu_{kk})$ for all $k=1,\ldots,d$
and $0<\varepsilon<\infty$. From equation~\eqref{eq:Mass_General}, the
mass $\mu^0_{jk}$ of the measure $\mu_{jk}$ is
given by $\mu^0_{jk}=\langle g_j,g_k \rangle_{1,2}
=\langle\bnabla\Delta^{-1}u_j\bcdot\bnabla\Delta^{-1}u_k\rangle$. Integration
by parts and the Cauchy-Schwartz
inequality~\cite{Folland:99:RealAnalysis} then yield
equation~\eqref{eq:Mass_Sobolev}. In particular
$|\mu^0_{kk}|\le\|\Delta^{-1}\|\|u_k\|^2<\infty$, as $\Delta^{-1}$ has
bounded operator norm $\|\Delta^{-1}\|$ on
$L^2(\Vc)$~\cite{Stakgold:BVP:2000}.    
Consequently, since
$0<(\varepsilon^2+\lambda^2)^{-1}\leq1/\varepsilon^2<\infty$ and
$0<\lambda^2(\varepsilon^2+\lambda^2)^{-1}<1$ for all
$0<\varepsilon<\infty$, it is clear that $f,h\in
L^2(\mu_{kk})$. Consequently, the spectral theorem  
in~\eqref{eq:Spectral_Theorem} implies that the functional 
formulas for $\Sg^*_{jk}$ and $\Ag^*_{jk}$ in
equation~\eqref{eq:Eff_Diffusivity_Resolvent_Sobolev} have the 
following Stieltjes integral representations
%
\begin{align}\label{eq:Integral_Rep_Discrete}
  \Sg^*_{jk}=\varepsilon\left(\delta_{jk}+\int_{-\infty}^\infty\frac{\d\mu_{jk}(\lambda)}{\varepsilon^2+\lambda^2}\right),
  \quad
  \Ag^*_{jk}&=-\int_{-\infty}^\infty\frac{\imath\lambda\, \d\mu_{jk}(\lambda)}{\varepsilon^2+\lambda^2}\,,
  %\\  
  %\d\mu_{jk}(\lambda)=\sum_{n\in I_N}\langle\delta_{\lambda_n}(\d\lambda)\Qc_n\,&\vecg_j\bcdot\vecg_k\rangle.
  %\notag
\end{align}
%
which involve the \emph{complex measure} $\mu_{jk}$.





We now show how the integrals for $\Sg^*_{jk}$ and $\Ag^*_{jk}$
in~\eqref{eq:Integral_Rep_Discrete} can
be represented in terms of the 
\emph{signed measures} $\Real\mu_{jk}$ and $\Imag\mu_{jk}$. Since
$\varepsilon$, $\chi_k$, $u_k$, and $g_k$ 
in~\eqref{eq:Eff_Diffusivity_Resolvent_Sobolev} are \emph{real-valued}, we
have from~\eqref{eq:Resolvent_Rep_Sobolev} the following symmetry conditions 
%
\begin{align}\label{eq:Fnal_symm}
  \langle(\varepsilon I+A)^{-1} g_j,(\varepsilon I+A)^{-1}g_k\rangle_{1,2}
   &=\langle(\varepsilon I+A)^{-1}g_k,(\varepsilon I+A)^{-1}g_j\rangle_{1,2}\\
   \langle A\,(\varepsilon I+A)^{-1} g_j,(\varepsilon I+A)^{-1}g_k\rangle_{1,2}
   &=\langle(\varepsilon I+A)^{-1}g_k,A\,(\varepsilon I+A)^{-1}g_j\rangle_{1,2}\,.
   \notag
\end{align}
%
These symmetries, the sesquilinearity of the $\Hs$-inner-product, the
linearity~\cite{Stone:64} of the Stieltjes integrals
in equation~\eqref{eq:Integral_Rep_Discrete} with respect to the function
$\mu_{jk}(\lambda)$, and
the two identities
$\Real\mu_{jk}(\lambda)
=(\mu_{jk}(\lambda)+\overline{\mu}_{jk}(\lambda))/2$
and 
$\Imag\mu_{jk}(\lambda)
=(\mu_{jk}(\lambda)-\overline{\mu}_{jk}(\lambda))/(2\imath)$
yield
equation~\eqref{eq:Integral_Rep_kappa*}. 
This concludes our proof of \thmref{thm:Int_Rep_Sobolev} $\Box\,.$




\begin{theorem}\label{thm:Bounds}
  Let $\Sigma$ denote the spectrum of the operator $M$, denote
$\lambda_+=\sup_{\lambda\in\Sigma}|\lambda|$, and assume
$0<\varepsilon<\infty$. Then, the diagonal 
components $\Sg^*_{kk}=\Dg^*_{kk}$ of the effective diffusivity matrix
satisfy the following, rigorous upper~\cite{Avellaneda:CMP-339} and
lower bounds  
%
\begin{align}\label{eq:Lower_Upper_Bounds_Skk}
  \varepsilon\,[1+\mu_{kk}^0/(\varepsilon^2+\lambda_+^2)]\leq\Sg^*_{kk}\leq\varepsilon\,[1+\mu_{kk}^0/\varepsilon^2\,]\,.
\end{align}
%



For $j\ne k$, let $\Real\mu_{jk}=\Real\mu_{jk}^+-\Real\mu_{jk}^-$ and
$\Imag\mu_{jk}=\Imag\mu^+_{jk}-\Imag\mu^-_{jk}$ denote the Jordan 
decomposition of the signed measures $\Real\mu_{jk}$ and $\Imag\mu_{jk}$,
respectively, and denote the total variation of these measures by
$|\Real\mu|_{jk}=\Real\mu_{jk}^++\Real\mu_{jk}^-$ and
$|\Imag\mu|_{jk}=\Imag\mu^+_{jk}+\Imag\mu^-_{jk}$, 
respectively. Finally, denote the
masses of the measures $\Real\mu_{jk}^+$, $\Real\mu_{jk}^-$, and
$|\Imag\mu|_{jk}$ by $[\Real\mu_{jk}^+]^0$, $[\Real\mu_{jk}^-]^0$, and 
$|\Imag\mu|^0_{jk}$, respectively. Then, the off-diagonal components
$\Sg^*_{jk}$ and $\Ag^*_{jk}$, $j\ne k$, satisfy the following,
rigorous upper and lower bounds
%
\begin{align}\label{eq:Lower_Upper_Bounds_Sjk} 
  \varepsilon\,\frac{[\Real\mu_{jk}^+]^0}{\varepsilon^2+\lambda_+^2}-\frac{[\Real\mu_{jk}^-]^0}{\varepsilon}
  \leq\Sg^*_{jk}\leq
  \frac{[\Real\mu_{jk}^+]^0}{\varepsilon}-\varepsilon\,\frac{[\Real\mu_{jk}^-]^0}{\varepsilon^2+\lambda_+^2}\,,
  \qquad j\neq k,
\end{align}
%
%
\begin{align}\label{eq:Lower_Upper_Bounds_Ajk} 
  -\frac{\lambda_+\,|\Imag\mu|^0_{jk}}{\varepsilon^2}
  \leq \Ag^*_{jk} \leq
  \frac{\lambda_+\,|\Imag\mu|^0_{jk}}{\varepsilon^2},
  \qquad
  j\neq k.
\end{align}
%
\end{theorem}


\noindent
\textbf{Proof of \thmref{thm:Bounds}.}
%
% We now establish the rigorous bounds for $\Sg^*_{jk}$ and
% $\Ag^*_{jk}$, $j,k=1,\ldots,d$, in
% equations~\eqref{eq:Lower_Upper_Bounds_Skk}--\eqref{eq:Lower_Upper_Bounds_Ajk},
% which follow from the integral representations
% in~\eqref{eq:Integral_Rep_kappa*}.
Assume that $0<\varepsilon<\infty$. From
equation~\eqref{eq:Spectral_Radius_Sobolev}, the spectrum 
$\Sigma$ of the
\emph{compact}~\cite{Bhattacharya:AAP:1999:951,Bhattacharya:1989:ASD,Majda:Kramer:1999:book}
operator $M=-\imath A$ is a bounded subset of $\mathbb{R}$. The
spectrum is discrete~\cite{Stakgold:BVP:2000} away from the spectral origin
$\lambda=0$ and comes in $\pm$ 
pairs~\cite{Pavliotis:PHD_Thesis} with an accumulation
point at $\lambda=0$~\cite{Stakgold:BVP:2000}. Denote
$\lambda_+=\sup_{\lambda\in\Sigma}|\lambda|$ and note that
$\inf_{\lambda\in\Sigma}\lambda^2=0$. The inequalities 
$1/(\varepsilon^2+\lambda_+^2)\leq1/(\varepsilon^2+\lambda^2)\leq1/\varepsilon^2$,
hold for all
$\lambda\in\Sigma$. Consequently, since
$\mu_{kk}$ is a positive measure with finite mass $\mu_{kk}^0$, 
the inequalities in~\eqref{eq:Lower_Upper_Bounds_Skk}
hold~\cite{Folland:99:RealAnalysis}. It may happen that
$\mu_{kk}^0=0$, hence $\Sg^*_{kk}=\varepsilon$, e.g., shear flow
orthogonal to the $k$th
direction~\cite{Avellaneda:CMP-339,Fannjiang:1994:SIAM_JAM:333}.        




When $j\neq k$, $\Real\mu_{jk}$ is a signed measure. By the Jordan
decomposition of $\Real\mu_{jk}$, there are unique,
\emph{positive} measures $\Real\mu_{jk}^+$ and $\Real\mu_{jk}^-$ such that
$\Real\mu_{jk}=\Real\mu_{jk}^+-\Real\mu_{jk}^-\,.$ Moreover, 
associated with the signed measure $\Real\mu_{jk}$ is its \emph{total
  variation} $|\Real\mu|_{jk}$~\cite{Folland:99:RealAnalysis}    
%
\begin{align}\label{eq:Total_Variation}
  \Real\mu_{jk}=\Real\mu_{jk}^+-\Real\mu_{jk}^-\,,
  \qquad
  |\Real\mu|_{jk}=\Real\mu_{jk}^++\Real\mu_{jk}^-\,.
\end{align}
%
From equation~\eqref{eq:Mass_Sobolev} the measures $\Real\mu_{jk}^+$
and $\Real\mu_{jk}^-$ 
have bounded mass, which we denote $[\Real\mu_{jk}^+]^0$ and
$[\Real\mu_{jk}^-]^0$, 
respectively, thus the mass $|\Real\mu|_{jk}^0$ of the measure
$|\Real\mu|_{jk}$ is also bounded. Since we
have~\cite{Folland:99:RealAnalysis} that 
$|\Sg^*_{jk}|\leq\int\d|\Real\mu|_{jk}(\lambda)/(\varepsilon^2+\lambda^2)$, the
upper bound in equation~\eqref{eq:Lower_Upper_Bounds_Skk} with
$\mu^0_{kk}$ replaced by $|\Real\mu|_{jk}^0$ holds for the positive
quantity $|\Sg^*_{jk}|$.
% Our numerical results
% in \secref{sec:Num_Results} indicate that $\mu_{kk}=|\Real\mu|_{jk}$,
% $j\neq k$, for a 2D cat's eye flow that is symmetric about the line
% $y=x$, yielding the bound $|\Sg^*_{jk}|\leq\Sg^*_{kk}$
% (see 
% Figures~\numfigref{fig:Figure3_Transition_Away_From_CatsEye_Cell_Flow}
% and~\numfigref{fig:Figure4_Transition_Toward_CatsEye_Shear_Flow}
% below).
These bounds for $\Sg^*_{jk}$ can be improved upon by
separately considering the positive and negative contributions of the
integral representation for  $\Sg^*_{jk}$, yielding
equation~\eqref{eq:Lower_Upper_Bounds_Sjk}. In a similar way, we
obtain the bounds for $\Ag^*_{jk}$ in
equation~\eqref{eq:Lower_Upper_Bounds_Ajk}.  
% where $|\Imag\mu|^0_{jk}$ is the finite mass of the total variation
% $|\Imag\mu|_{jk}=\Imag\mu^+_{jk}+\Imag\mu^-_{jk}$ of the signed measure
% $\Imag\mu_{jk}=\Imag\mu^+_{jk}-\Imag\mu^-_{jk}$~\cite{Folland:99:RealAnalysis}.
This concludes our proof of \thmref{thm:Bounds} $\Box\,.$


We conclude this section by noting that bounds on $\Sg^*_{kk}$
can also be obtained using variational
methods~\cite{Fannjiang:1994:SIAM_JAM:333,Fannjiang:1997:1033,Avellaneda:CMP-339}
as well as Pad\'{e}
approximants~\cite{Baker:1996:Book:Pade,Avellaneda:CMP-339} of
Stieltjes functions. 

% One can also use the convexity of certain 
% measure spaces~\cite{Golden:CMP-473} to obtain a sequence of bounds
% for $\Sg^*_{kk}$ that incorporate progressively more moments
% $\mu_{kk}^n=\int\lambda^n\d\mu_{kk}(\lambda)$, $n=1,2,3,\ldots$, of the measure $\mu_{kk}$. This is a
% topic of current work.  


% We now discuss how the integral representations
% in~\eqref{eq:Integral_Rep_kappa*} change when the formulation of
% the effective parameter problem given in
% equations~\eqref{eq:ADE_S_Pe}--\eqref{eq:Djk_S_Pe} is
% used. Substituting into equation~\eqref{eq:Djk_S_Pe} the expression
% for $u_j$ in~\eqref{eq:Random_Cell_Prob_S_Pe} yields the following
% analogue of equation~\eqref{eq:Eff_Diffusivity} 
% %
% \begin{align}\label{eq:Eff_Diffusivity_S_Pe}
%   \Dg^*_{jk}=\Sg^*_{jk}+\Ag^*_{jk},
%   \qquad
%   \Sg^*_{jk}=\varepsilon\delta_{jk}+\Pen^{-1}\langle\bnabla\chi_j\bcdot\bnabla\chi_k\rangle,
%   \quad
%   \Ag^*_{jk}=\langle\Ab\bnabla\chi_j\bcdot\bnabla\chi_k\rangle.
% \end{align}
% %
% Applying the integro-differential operator $\bnabla\Delta^{-1}$ to the  
% cell problem in equation~\eqref{eq:Random_Cell_Prob_S_Pe} yields the
% following analogue of equation~\eqref{eq:Resolvent_Rep}
% % 
% \begin{align}\label{eq:Resolvent_Rep_S_Pe}
%   \bnabla\chi_k=\Pen\,(\Ib+\Pen\,\Ab)^{-1}\vecg_k,
%   \quad     
%   \vecg_k=-\bGamma\Hm\vece_k.
% \end{align}
% %
% The spectral theorem, and equations~\eqref{eq:Eff_Diffusivity_S_Pe}
% and~\eqref{eq:Resolvent_Rep_S_Pe} provide the following analogue of
% equation
% %
% \begin{align}\label{eq:Integral_Rep_kappa*}
%   \Sg^*_{jk}=\delta_{jk}+\Pen^2\int_{-\infty}^\infty\frac{\d\Real\mu_{jk}(\lambda)}{1+\Pen^2\lambda^2},
%   \quad
%   \Ag^*_{jk}=\Pen^3\int_{-\infty}^\infty\frac{\lambda\, \d\Imag\mu_{jk}(\lambda)}{1+\Pen^2\lambda^2}\,.
% \end{align}
% %
% As in~\eqref{eq:Assymptotics}, equation~\eqref{eq:Integral_Rep_kappa*}
% provides the following asymptotic behavior of the effective
% diffusivity matrix $\Dg^*$   
% %
% \begin{align}\label{eq:Assymptotics_Pe}
%   &\Sg^*_{jk}\sim\delta_{jk}+\Pen^2 \text{~as~} \Pen\to0,
%   \qquad
%   \Sg^*_{jk}\gtrsim O(1), \text{~as~} \Pen\to\infty,
%   \\
%   &\Ag^*_{jk}\sim\Pen^3, \text{~as~} \Pen\to0,
%   \qquad
%   \Ag^*_{jk}\gtrsim\Pen, \text{~as~} \Pen\to\infty,
%   \notag
% \end{align}
% %
% Since $\Pen\sim\varepsilon^{-1}$ equations~\eqref{eq:Assymptotics}
% and~\eqref{eq:Assymptotics_Pe} together give the refined asymptotic
% behavior of the effective diffusivity matrix $\Dg^*$ 
% %
% \begin{align}\label{eq:Assymptotics}
%   &\Sg^*_{jk}\sim\varepsilon, \text{~as~} \varepsilon\to\infty,
%   \qquad
%   \Sg^*_{jk}\gtrsim\varepsilon, \text{~as~} \varepsilon\to0,
%   \\
%   &\Ag^*_{jk}\sim\varepsilon^{-2}, \text{~as~} \varepsilon\to\infty,
%   \qquad
%   \Ag^*_{jk}\gtrsim O(1), \text{~as~} \varepsilon\to0
%   \notag
% \end{align}
% %




\section{Discrete setting: Sobolev space of scalar fields} 
\label{sec:Matrix_Sobolev}  
%
For our numerical computations of the effective diffusivity $\Dg^*$ in
\secref{sec:Num_Results}, we consider a discrete approximation of 
the cell problem in equation~\eqref{eq:Random_Cell_Prob} written as
$(\varepsilon+\imath M)\chi_j=g_j$. Here $M=-\imath A$,
$A=\Delta^{-1}[\vecu\bcdot\bnabla]$, and  $g_j=-\Delta^{-1}u_j$, as
defined in equations~\eqref{eq:Eff_Diffusivity_Sobolev}
and~\eqref{eq:Resolvent_Rep_Sobolev}. In this section, we manipulate
these formulas in order to develop a numerical algorithm which enables
numerical computations of $\Dg^*$ by directly computing a discrete
representation of the spectral measure $\mu_{jk}$ in
equation~\eqref{eq:Integral_Rep_kappa*}, in terms of the
eigenvalues and eigenvectors of a \emph{generalized} eigenvalue problem.
This is not a trivial extension of the spectral theory for the
continuum setting discussed in~\secref{sec:Integral_Reps_Sobolev}, as
the matrix 
representation of the operator $(-\Delta)^{-1}[\vecu\bcdot\bnabla]$ is
\emph{not} Hermitian.
% nor are the projection matrices
% %$\Qm_n$
% associated with the discrete resolution of the identity analogous
% to~\eqref{eq:Res_Identity}.
The special structure of the generalized 
eigenvalue problem itself makes these matrix operators Hermitian only with
respect to a discrete Sobolev-like inner-product analogous to the
inner-product $\langle\cdot,\cdot\rangle_{1,2}$ introduced
after equation~\eqref{eq:Sobolev}. Moreover, the eigenvector orthogonality is
only with respect to this inner-product.  



Towards this goal, we begin by noting that since $\vecu(\vecx)$ is
incompressible, $\bnabla\bcdot\vecu=0$, there is a real
(non-dimensional) \emph{antisymmetric} 
%random
matrix
$\Hm(\vecx)$~\cite{Avellaneda:PRL-753,Avellaneda:CMP-339} such that 
% 
\begin{align}\label{eq:u_DH}
 \vecu=\bnabla\bcdot\Hm, \qquad  \Hm^{\,T}=-\Hm,
\end{align}
% 
where $\Hm^{\,T}$ denotes transposition of the matrix $\Hm$.
Due to the anti-symmetry of the matrix $\Hm$ in equation~\eqref{eq:u_DH} and
the symmetry of the 
Hessian operator $\bnabla\bnabla$ when acting on a sufficiently smooth
space of functions, we have $\Hm\bcolon\bnabla\bnabla\varphi=0$ for all
such smooth functions $\varphi$, where $\bcolon$ denotes matrix
contraction. Consequently, 
$\bnabla\bcdot[\Hm\bnabla\varphi]=[\bnabla\bcdot\Hm]\bcdot\bnabla\varphi+\Hm\bcolon\bnabla\bnabla\varphi=[\bnabla\bcdot\Hm]\bcdot\bnabla\varphi$,
or
%$\bnabla\bcdot[\Hm\bnabla]=[\bnabla\bcdot\Hm]\bcdot\bnabla$
%
\begin{align}\label{eq:H_Hessian}
  \bnabla\bcdot[\Hm\bnabla]=[\bnabla\bcdot\Hm]\bcdot\bnabla,
\end{align}
%
as operators acting on such functions. With~\eqref{eq:H_Hessian} we
can write the operator $M=\imath 
(-\Delta)^{-1}[\vecu\bcdot\bnabla]$ as
$M=\imath(-\Delta)^{-1}\bnabla\bcdot[\Hm\bnabla]\,.$


We now discuss our discrete formulation of the effective parameter problem,
which leads to Stieltjes integral representations for the symmetric
$\Sg^*$ and antisymmetric $\Ag^*$ components of the effective
diffusivity matrix $\Dg^*$, involving a discrete spectral measure. For
notational brevity, assume the period cell $\Vc$ is square. In the discrete
setting~\cite{Murphy:2015:CMS:13:4:825}, $\Vc$ is represented by a
square grid of size $L$, which is bijectively mapped to a vector with
$L^d$ components. The functions $u_j(\vecx)$ and
$\chi_j(\vecx)$ are mapped to vectors $\vecu_j$ and $\vecchi_j$
with $L^d$ components, respectively, and the matrix $\Hm(\vecx)$ is mapped
to a square banded antisymmetric matrix of size $N=L^dd$ (see
\secref{sec:Numerical_Methods} for details). For
simplicity, we will not make a notational distinction for $\Hm$
between the discrete and continuum settings, as the context will be
clear. 




The differential operator $\bnabla$ is
represented by a finite difference matrix
$\nabla$~\cite{Murphy:2015:CMS:13:4:825,Demmel:1997}, where 
$\nabla^T=(\nabla_1^T,\ldots,\nabla_d^T)$ and $\nabla_j$, $j=1,\ldots,d$, are also finite
difference matrices. Moreover, the divergence operator $\bnabla\bcdot$ 
is given by $-\nabla^T$ and the matrix representation of the negative
Laplacian $-\Delta$ is given by $\nabla^T\nabla$.
Consequently, we may write the  
discrete, matrix representation $\Mm$ of the self-adjoint operator
$M=\imath(-\Delta)^{-1}\bnabla\bcdot[\Hm\bnabla]$
as $\Mm=(\nabla^T\nabla)^{-1}[-\imath\nabla^T\Hm\nabla$].  This
composition of the Hermitian matrix $-\imath\nabla^T\Hm\nabla$ and the 
real-symmetric matrix $(\nabla^T\nabla)^{-1}$ is neither real-symmetric nor
Hermitian symmetric. From equation~\eqref{eq:Anti-sym_Sobolev} we see that the
symmetry properties of the integro-differential operator $M$ depend
intimately on the inner-product $\langle
f,h\rangle_{1,2}=\langle\bnabla f\bcdot\bnabla h\rangle$ of the
underlying Sobolev space $\Hs^{1,2}$ defined
in equation~\eqref{eq:Sobolev}. Therefore, we anticipate the properties of this 
inner-product must be incorporated into the discrete formulation of integral
representations for $\Dg^*$.



We now provide a matrix formulation of the effective parameter problem
introduced in \secref{sec:Integral_Reps_Sobolev}, which involves a
\emph{generalized eigenvalue problem} that has the Sobolev-type
inner-product as a central feature.  In particular, this formulation
retains the key properties of the weak form of the eigenvalue problem 
$\langle M\varphi_n,\varphi_n\rangle_{1,2}=\lambda_n$. Namely, the operator $M$ is self-adjoint
\emph{with respect to the inner-product} $\langle\cdot,\cdot\rangle_{1,2}$, its eigenfunctions $\varphi_n\in\Hs^{1,2}$ are
orthonormal $\langle\varphi_n,\varphi_m\rangle_{1,2}=\delta_{nm}$, $n,m=1,2,3,\ldots$, with respect to the
inner-product $\langle\cdot,\cdot\rangle_{1,2}$, and the spectrum $\Sigma$ of $M$ is real valued,
$\Sigma\subset\mathbb{R}$. Towards this goal, consider the eigenvalue problem
$M\varphi_n=\lambda_n\varphi_n$ in the following ``strong'' form,    
%
\begin{align}\label{eq:Strong_eval_prob}
  \imath\bnabla\bcdot[\Hm\bnabla\varphi_n]=\lambda_n(-\Delta)\varphi_n.
  %\qquad
  %\langle\varphi_n,\varphi_m\rangle_{1,2}=\delta_{nm}.
\end{align}
%
% Equation~\eqref{eq:Strong_eval_prob} is well defined for
% $\Hm_{jk}\in C^1(\Vc)$ and $\varphi_n\in C^2(\Vc)$, where $C^r(\Vc)$
% the space of continuously differentiable functions of order $r$ with domain
% $\Vc$.
Using a discrete version of equation
\eqref{eq:Strong_eval_prob}, we will establish the integral
representations in~\eqref{eq:Integral_Rep_kappa*} for discrete
versions of the functionals
$\Sg^*_{jk}=\varepsilon(\delta_{jk}+\langle\chi_j,\chi_k\rangle_{1,2})$ 
and $\Ag^*_{jk}=\langle\imath M\chi_j,\chi_k\rangle_{1,2}$ 
in~\eqref{eq:Eff_Diffusivity_Sobolev}, involving a discrete spectral
measure.     
 



By our discussion in this section, the matrix  
representation of the eigenvalue problem in
\eqref{eq:Strong_eval_prob} is 
%
\begin{align}\label{eq:Strong_eval_prob_matrix}  
  \Bm\vecz_n=\lambda_n\Cm\vecz_n,
  % \quad
%   \langle\nabla\vecz_n\bcdot\nabla\vecz_m\rangle=\delta_{nm},
  \qquad
  \Bm=-\imath\nabla^T\Hm\nabla,
  \quad
  \Cm=\nabla^T\nabla.
\end{align}
%
The first formula in equation~\eqref{eq:Strong_eval_prob_matrix} is a
\emph{generalized eigenvalue problem}~\cite{Parlett:1980} associated
with the pencil $\Bm-\lambda\Cm$, where $\Bm$ and $\Cm$ are
Hermitian and real-symmetric matrices, respectively, of size
$K=L^d$. The $\lambda_n$ and $\vecz_n$, $n=1,\ldots,K$, are known as
generalized eigenvalues and eigenvectors, respectively. The matrix
$\Cm=\nabla^T\nabla$ is clearly positive semi-definite. In this
section, we will assume that $\Cm$ is positive definite, hence
invertible. We will discuss the case where  $\Cm$ is positive
semi-definite in \appref{app:Eigenvalue_method}, which is necessary
for the setting where $\nabla$ incorporates \emph{periodic} boundary
conditions --- needed for our study of advection enhanced diffusion by
a spatially periodic fluid velocity field $\vecu$. 






Since $\Bm$ and $\Cm$ are Hermitian and real-symmetric, respectively,
and $\Cm$ is 
positive definite, the generalized eigenvalue problem
in~\eqref{eq:Strong_eval_prob_matrix} has properties  
which are similar to the properties of the standard symmetric
eigenvalue problem \cite{Parlett:1980}. In particular, the generalized 
eigenvalues $\lambda_n$ are all real, the generalized eigenvectors
$\vecz_n$ form a basis for $\mathbb{C}^K$, and the $\vecz_n$ are
orthonormal in the following sense $\vecz_n^{\,\dagger}\Cm\vecz_m=\delta_{nm}$,
$n,m=1,\ldots,K$ \cite{Parlett:1980}. Since $\Cm=\nabla^T\nabla$ is
real-valued, this is equivalent to the Sobolev-type orthogonality
condition   
%
\begin{align}\label{eq:Sobolev_orthogonality}
  \nabla\vecz_n\bcdot\nabla\vecz_m=\delta_{nm}\,.  
\end{align}
%
In other words, the generalized eigenvectors $\vecz_n$ are orthonormal
with respect to the ``discrete inner-product'' $\langle\cdot,\cdot\rangle_{1,2}$ defined by
$\langle\vecxi,\veczeta\rangle_{1,2}=\langle \nabla\vecxi\bcdot\nabla\veczeta\rangle$, for 
$\vecxi,\veczeta\in\mathbb{C}^K$ and $\langle\cdot\rangle$ denotes
ensemble averaging. 
Denoting by $\Zm$ the matrix with columns consisting of the generalized
eigenvectors $\vecz_n$, equation \eqref{eq:Sobolev_orthogonality} is
seen to be equivalent to $[\nabla\Zm]^\dagger[\nabla\Zm]=\Ib$, or
$\Zm^\dagger\Cm\Zm=\Ib$. A key feature of the generalized eigenvalue
problem is that the matrix $\Zm$ simultaneously
diagonalizes $\Bm$ and $\Cm$. Specifically, if $\Lambda$ is the diagonal
matrix whose elements on the main diagonal are the generalized
eigenvalues $\lambda_n$, then \cite{Parlett:1980}
% 
\begin{align}\label{eq:Simultaneous_Diag}
  \Zm^\dagger\Bm\Zm=\Lambda, \qquad
  \Zm^\dagger\Cm\Zm=\Ib.
\end{align}
%



We now derive the discrete version of
equations~\eqref{eq:Res_Identity}--\eqref{eq:Mass_General} comprising
the key results of the spectral theorem, for the generalized
eigenvalue problem setting. These derivations will  
streamline and clarify our results for the current section, showing how the
derived series representations of the symmetric $\Sg^*$ and
antisymmetric $\Ag^*$ parts of the effective diffusivity matrix
$\Dg^*$ can be written as the Stieltjes integrals in
equation~\eqref{eq:Integral_Rep_kappa*}, involving a discrete spectral
measure. This derivation will also 
clarify and streamline our results in the appendix, which lead to the
numerical algorithm used in \secref{sec:Num_Results} for spectral
computations of $\Dg^*$ for  periodic fluid flows. The numerical
algorithm used in \secref{sec:Num_Results} is analogous to the
algorithm that we develop in the current section, which is elegant for 
the full-rank setting --- revealing a great deal of structure with
minimal effort. The matrix analysis of the rank-deficient setting
developed in \appref{app:Eigenvalue_method} is quite a bit more
involved. Developing the full-rank setting in the current section
first, makes the results of the rank-deficient setting more transparent
and the final results more anticipated.  





Since the $\vecz_n$, $n=1,\ldots,K$, form a basis for $\mathbb{C}^K$ and
satisfy the orthogonality relation
in~\eqref{eq:Sobolev_orthogonality}, for all 
$\vecxi\in\mathbb{C}^K$ we have 
$\vecxi=\sum_n(\nabla\vecz_n\bcdot\nabla\vecxi)\vecz_n
=\sum_n(\vecz_n[\nabla\vecz_n]^{\,\dagger}\nabla)\vecxi$, which
implies
%the following analogue of equation~\eqref{eq:Matrix_Rep_Spec_Theorem}
%
\begin{align}\label{eq:Matrix_Rep_Spec_Theorem_Sobolev}
  \sum_{n=1}^K\Qm_n=\Ib, \qquad
  \Qm_n=\vecz_n[\nabla\vecz_n]^{\,\dagger}\nabla,  \qquad
  \Qm_l\,\Qm_m=\Qm_l\,\delta_{l m},
\end{align}
%
where the matrix operators $\Qm_n$, $n=1,\ldots,K$, are self-adjoint with
respect to the \emph{discrete} inner-product
$\langle\cdot,\cdot\rangle_{1,2}$ defined above, i.e.,
$\langle\Qm_n\vecxi,\veczeta\rangle_{1,2}=\langle\vecxi,\Qm_n\veczeta\rangle_{1,2}$
for all $\vecxi,\veczeta\in\mathbb{C}^K$. 
% We now use equation~\eqref{eq:Matrix_Rep_Spec_Theorem_Sobolev} to prove the
% spectral theorem in~\eqref{eq:Spectral_Theorem} for this generalized
% eigenvalue problem setting.
From $\Bm\vecz_n=\lambda_n\Cm\vecz_n$ and
equation~\eqref{eq:Matrix_Rep_Spec_Theorem_Sobolev} we have that
$\Bm\Qm_n=\lambda_n\Cm\Qm_n$ which is equivalent to
$\Cm^{-1}\Bm\Qm_n=\lambda_n\Qm_n$, since the matrix $\Cm$ is
invertible. This formula
and~\eqref{eq:Matrix_Rep_Spec_Theorem_Sobolev} then imply that the matrix 
$\Cm^{-1}\Bm$ has the spectral decomposition
$\Cm^{-1}\Bm=\sum_n\lambda_n\Qm_n$. By the mutual 
orthogonality of the projection matrices $\Qm_n$ and by induction, we
have that  
$[\Cm^{-1}\Bm]^m=\sum_n\lambda_n^m\Qm_n$ for all $m\in\mathbb{N}$. This, in turn,
implies that $f(\Cm^{-1}\Bm)=\sum_nf(\lambda_n)\Qm_n$ for
any polynomial $f:\mathbb{R}\mapsto\mathbb{C}$.






From the mutual orthogonality of
the projection matrices $\Qm_n$ and their symmetry with
respect to the discrete inner-product $\langle\cdot,\cdot\rangle_{1,2}$ it follows that, for
all $\vecxi,\veczeta\in\mathbb{C}^K$ and all complex-valued polynomials
$f(\lambda)$ and $h(\lambda)$, the bilinear functional $\langle
f(\Cm^{-1}\Bm)\vecxi,h(\Cm^{-1}\Bm)\veczeta\rangle_{1,2}$ has the
integral representation in~\eqref{eq:Spectral_Theorem}, with $M$
substituted by $\Cm^{-1}\Bm$ and other appropriate notational
changes. Moreover, the complex-valued function
$\mu_{\xi\zeta}(\lambda)=\langle
 Q(\lambda)\xi,\zeta\rangle_{1,2}$ in~\eqref{eq:Spectral_Theorem} is 
now given by
$\mu_{\xi\zeta}(\lambda)=\langle\Qm(\lambda)\vecxi,\veczeta\rangle_{1,2}\,$,
where the associated 
matrix representation $\Qm(\lambda)$ of the projection operator
$Q(\lambda)$ and the discrete spectral measure
$\d\mu_{\xi\zeta}(\lambda)$ are given by 
%the following analogue of equation~\eqref{eq:Disc_Spec_Measure_Matrix} 
% 
\begin{align}\label{eq:Disc_Spec_Measure_Matrix_Sobolev}
  \Qm(\lambda)=\sum_{n:\;\lambda_n\leq\lambda}\theta(\lambda-\lambda_n)\Qm_n,
  \qquad
  \d\mu_{\xi\zeta}(\lambda)=\sum_{n:\;\lambda_n\leq \lambda}\langle\delta_{\lambda_n}(\d\lambda)[\nabla\Qm_n\vecxi\bcdot\nabla\veczeta]\rangle.
\end{align}
%
Here, $\theta(\lambda)$ is the Heaviside function, satisfying
$\theta(\lambda)=0$ for $\lambda<0$ and $\theta(\lambda)=1$ for
$\lambda\geq0$, and
$\delta_{\lambda_n}(\d\lambda)=\d\theta(\lambda-\lambda_n)$ is the
$\delta$-measure centered at $\lambda_n$.  From
equation~\eqref{eq:Matrix_Rep_Spec_Theorem_Sobolev} and well known
properties of $\theta(\lambda)$, we have that $\Qm(\lambda)$ satisfies
equation~\eqref{eq:Res_Identity}. From
equation~\eqref{eq:Matrix_Rep_Spec_Theorem_Sobolev} 
and well known properties of the $\delta$-measure, the mass
$\mu^0_{\xi\zeta}$ of the spectral measure
in~\eqref{eq:Disc_Spec_Measure_Matrix_Sobolev} satisfies
equation~\eqref{eq:Mass_General} with appropriate notational changes. 
We are now ready to present the main result of this section.


%
\begin{theorem}\label{thm:Int_Rep_Sobolev_Matrix}
%
Consider the generalized eigenvalue problem
$\Bm\vecz_n=\lambda_n\Cm\vecz_n$
in~\eqref{eq:Strong_eval_prob_matrix}. Let $\Zm$ be the matrix with
columns consisting of the eigenvectors $\vecz_n$ and $\Lambda$ be the
diagonal matrix with eigenvalues $\lambda_n$ on the diagonal, which
satisfy equation~\eqref{eq:Simultaneous_Diag}. 
The discrete, matrix representations of the bilinear functional
formulas for $\Sg^*_{jk}$ and $\Ag^*_{jk}$ in
equation~\eqref{eq:Eff_Diffusivity_Sobolev} are given by  
%
\begin{align}\label{eq:Matrix_Functionals_Sobolev}
  \Sg^*_{jk}
  =\varepsilon(\delta_{jk}
  +\langle\nabla\vecchi_j\bcdot\nabla\vecchi_k\rangle),
  \quad
  \Ag^*_{jk}
  =\langle\nabla\Cm^{-1}[\imath\Bm]\vecchi_j\bcdot\nabla\vecchi_k\rangle\,.   
\end{align}
%
Also, the discrete representation of the resolvent formula for $\chi_j$ in
equation~\eqref{eq:Resolvent_Rep_Sobolev} is given by
%
\begin{align}\label{eq:Matrix_chij}
  \vecchi_j=\Zm(\varepsilon\Ib+\imath\Lambda)^{-1}\Zm^{\,\dagger}\vecu_j\,.
\end{align}
%
The discrete representations of the bilinear functional
formulas for $\Sg^*_{jk}$ and $\Ag^*_{jk}$
in~\eqref{eq:Eff_Diffusivity_Resolvent_Sobolev} are given by 
% 
\begin{align}\label{eq:Functionals_kappa_alpha_Sobolev}
\Sg^*_{jk}&=
\varepsilon(\delta_{jk}+
\langle(\varepsilon\Ib+\imath\Lambda)^{-1}\Zm^{\,\dagger}\vecu_j
\bcdot(\varepsilon\Ib+\imath\Lambda)^{-1}\Zm^{\,\dagger}\vecu_k
\rangle)\,,
\\
\Ag^*_{jk}&=
\langle\imath\Lambda(\varepsilon\Ib+\imath\Lambda)^{-1}\Zm^{\,\dagger}\vecu_j
\bcdot(\varepsilon\Ib+\imath\Lambda)^{-1}\Zm^{\,\dagger}\vecu_k
\rangle\,.
\notag
\end{align}
%
Consequently, from the discrete analogue of
equation~\eqref{eq:Fnal_symm} and the formulas for $\Sg^*_{jk}$ and
$\Ag^*_{jk}$ in~\eqref{eq:Functionals_kappa_alpha_Sobolev}, we have the
following series representations  
%
\begin{align}\label{eq:Discrete_Integrals_full_rank_Sobolev}
  \Sg^*_{jk}/\varepsilon-\delta_{jk}=\sum_{n=1}^{K}
      \frac{\Real[\,\overline{(\vecz_n^\dagger\vecu_j)}
                              (\vecz_n^\dagger\vecu_k)\,]}
           {\varepsilon^2+\lambda_n^2}\,,
  \qquad
  \Ag^*_{jk}=\sum_{n=1}^{K}
      \frac{\lambda_n\,\Imag[\,\overline{(\vecz_n^\dagger\vecu_j)}
                                         (\vecz_n^\dagger\vecu_k)\,]}
           {\varepsilon^2+\lambda_n^2}\,.
\end{align}
%
Finally, defining $\vecg_j=(\nabla^T\nabla)^{-1}\vecu_j$ and recalling
the projection matrix $\Qm_n$ 
in~\eqref{eq:Matrix_Rep_Spec_Theorem_Sobolev}, we have 
%
%
\begin{align}\label{eq:Sobolev_weights}
  \overline{(\vecz_n^\dagger\vecu_j)}(\vecz_n^\dagger\vecu_k)
  %=\vecz_n\vecz_n^{\,\dagger}\vecu_j\bcdot\vecu_k
  %=\vecz_n\vecz_n^{\,\dagger}[\nabla^T\nabla]\vecg_j
    %\bcdot[\nabla^T\nabla]\vecg_k
  %=[\nabla\vecz_n][\nabla\vecz_n]^{\,\dagger}\nabla\vecg_j
   %\bcdot\nabla\vecg_k
  =\nabla\Qm_n\vecg_j\bcdot\nabla\vecg_k\,.
\end{align}
%
It follows from equations
\eqref{eq:Discrete_Integrals_full_rank_Sobolev} and \eqref{eq:Sobolev_weights}
that the series representations for $\Sg^*_{jk}$ and $\Ag^*_{jk}$
in~\eqref{eq:Discrete_Integrals_full_rank_Sobolev} have the Stieltjes   
integral representations in equation~\eqref{eq:Integral_Rep_kappa*},
involving the discrete spectral measure $\mu_{jk}$ 
in~\eqref{eq:Disc_Spec_Measure_Matrix_Sobolev} with 
$\vecxi=\vecg_j$ and $\veczeta=\vecg_k\,$.
%
\end{theorem}
%


\noindent
\textbf{Proof of \thmref{thm:Int_Rep_Sobolev_Matrix}}.
From the matrix
representation $\Am=(\nabla^T\nabla)^{-1}(\nabla^T\Hm\nabla)$ of the
operator $A=\Delta^{-1}\bnabla\bcdot[\Hm\bnabla]$ 
and equation~\eqref{eq:Strong_eval_prob_matrix}, the matrix representation     
of the functional formulas $\Sg^*_{jk}=\varepsilon(\delta_{jk}+\langle\chi_j,\chi_k\rangle_{1,2})$ and
$\Ag^*_{jk}=\langle A\chi_j,\chi_k\rangle_{1,2}$ in
equation~\eqref{eq:Eff_Diffusivity_Sobolev} are given
by~\eqref{eq:Matrix_Functionals_Sobolev}.  
Moreover,  the matrix representation of the cell problem
$\varepsilon\Delta\chi_j+\bnabla\bcdot[\Hm\bnabla]\chi_j=-u_j$ 
in~\eqref{eq:Random_Cell_Prob} is given by   
% 
\begin{align}\label{eq:Cell_Problem_Gen}
  (\varepsilon\Cm+\imath\Bm)\vecchi_j=\vecu_j\,.
\end{align}
%
The matrix $\Zm$ is invertible, as the generalized eigenvectors
$\vecz_n$ form a basis for $\mathbb{C}^K$. Consequently, by equation
\eqref{eq:Simultaneous_Diag} we have that
$\Bm=\Zm^{-\dagger}\Lambda\Zm^{-1}$, $\Cm=\Zm^{-\dagger}\Zm^{-1}$. It now follows from
equation~\eqref{eq:Cell_Problem_Gen} that
$\Zm^{-\dagger}(\varepsilon\Ib+\imath\Lambda)\Zm^{-1}\vecchi_j=\vecu_j$,
which implies the resolvent formula for $\vecchi_j$ in
equation~\eqref{eq:Matrix_chij}. 




Substituting the resolvent formula for $\vecchi_j$
in~\eqref{eq:Matrix_chij} into
equation~\eqref{eq:Matrix_Functionals_Sobolev}  yields
equation~\eqref{eq:Functionals_kappa_alpha_Sobolev}, 
%that is a direct analogue of equation~\eqref{eq:Functionals_kappa_alpha} 
where we have used that $[\nabla\Zm]^\dagger=[\nabla\Zm]^{-1}$. The
quadratic form $\Zm^{\,\dagger}\vecu_j\bcdot\Zm^{\,\dagger}\vecu_k$ arising
in~\eqref{eq:Functionals_kappa_alpha_Sobolev} can be written in terms
of the projection matrices $\Qm_n$ defined
in~\eqref{eq:Matrix_Rep_Spec_Theorem_Sobolev} as follows
%Analogous to equation~\eqref{eq:Quadratic_W}, we have      
%
\begin{align}\label{eq:Quadratic_Z}
  \Zm^\dagger\vecu_j\bcdot\Zm^\dagger\vecu_k
  =\sum_{n=1}^K\overline{(\vecz_n^\dagger\vecu_j)}(\vecz_n^\dagger\vecu_k)
  =\sum_{n=1}^K\vecz_n\vecz_n^{\,\dagger}\vecu_j\bcdot\vecu_k.
\end{align}
%
We now demonstrate that
$\vecz_n\vecz_n^{\,\dagger}\vecu_j\bcdot\vecu_k
=\nabla\Qm_n\vecg_j\bcdot\nabla\vecg_k$, 
where $\vecg_j=(\nabla^T\nabla)^{-1}\vecu_j$,   
%
\begin{align}\label{eq:Sobolev_weights_tfm}
  \vecz_n\vecz_n^{\,\dagger}\vecu_j\bcdot\vecu_k
  =\vecz_n\vecz_n^{\,\dagger}[\nabla^T\nabla]\vecg_j\bcdot[\nabla^T\nabla]\vecg_k
  =[\nabla\vecz_n][\nabla\vecz_n]^{\,\dagger}\nabla\vecg_j\bcdot\nabla\vecg_k
  =\nabla\Qm_n\vecg_j\bcdot\nabla\vecg_k,
\end{align}
%
which establishes equation~\eqref{eq:Sobolev_weights}. This concludes
our proof of \thmref{thm:Int_Rep_Sobolev_Matrix} $\Box\,.$



In \secref{sec:Num_Results} we use a generalization of
equation~\eqref{eq:Discrete_Integrals_full_rank_Sobolev} to compute Stieltjes
integral representations for the symmetric $\Sg^*$ and antisymmetric
$\Ag^*$ parts of the effective diffusivity matrix $\Dg^*$ for some
model, periodic fluid velocity fields. This generalization,
discussed in~\thmref{thm:Spectral_Equivalence_Rank_Def} below,
%, shown in equation~\eqref{eq:Discrete_Integrals_rank_deficient},
is a direct analogue of
equation~\eqref{eq:Discrete_Integrals_full_rank_Sobolev} and holds 
for the setting where the matrix gradient $\nabla$ with periodic
boundary conditions is rank-deficient, so the negative matrix
Laplacian $\nabla^T\nabla$ is \emph{non-invertible}.



We conclude this
section with a discussion that helps reduce the amount of memory
required to store the eigenvalues and spectral weights comprising the
discrete spectral measure, which is
useful when computing a large number of statistical realizations
associated with randomly perturbed periodic flows.
From the formula for $\mu_{\xi\zeta}(\lambda)$ above
equation~\eqref{eq:Disc_Spec_Measure_Matrix_Sobolev}, the fact that
the matrix $\nabla$ and vectors $\vecg_k$ are real-valued, and the two 
identities
$\Real\mu_{jk}(\lambda)
=(\mu_{jk}(\lambda)+\overline{\mu}_{jk}(\lambda))/2$
and 
$\Imag\mu_{jk}(\lambda)
=(\mu_{jk}(\lambda)-\overline{\mu}_{jk}(\lambda))/(2\imath)\,,$
we have 
%
\begin{align}\label{eq:Signed_mu_Dis}
\Real\mu_{jk}(\lambda)&=
 \frac{1}{2}\sum_{n:\;\lambda_n\leq\lambda}
 \langle\theta(\lambda-\lambda_n)[\nabla(\Qm_n+\overline{\Qm}_n)\vecg_j\bcdot\nabla\vecg_k\,]\rangle
\\
\Imag\mu_{jk}(\lambda)&=
 \frac{1}{2\imath}\sum_{n:\;\lambda_n\leq\lambda}
 \langle\theta(\lambda-\lambda_n)[\nabla(\Qm_n-\overline{\Qm}_n)\vecg_j\bcdot\nabla\vecg_k\,]\rangle\,.
\notag
\end{align}
%
with
$[\nabla(\Qm_n+\overline{\Qm}_n)\vecg_j\bcdot\nabla\vecg_k\,]
=2\Real[\nabla\Qm_n\,\vecg_j\bcdot\nabla\vecg_k]$
and $[\nabla(\Qm_n-\overline{\Qm}_n)\vecg_j\bcdot\nabla\vecg_k\,]
=2\imath\Imag[\nabla\Qm_n\,\vecg_j\bcdot\nabla\vecg_k]$.







Consider the generalized eigenvalue problem in
equation~\eqref{eq:Strong_eval_prob_matrix} written as 
$[\imath\Bm]\vecz_n=\imath\lambda_n\Cm\vecz_n\,,$ with 
$\Bm=-\imath\nabla^T\Hm\nabla$ and $\Cm=\nabla^T\nabla$. Since the
matrices $\imath\Bm$ and $\Cm$ are real-valued we have
$[\imath\Bm]\overline{\vecz}_n=-\imath\lambda_n\Cm\,\overline{\vecz}_n$.
Consequently, the (generalized) eigenvectors $\vecz_n$ come in complex
conjugate pairs and the $\lambda_n$ come in $\pm$ pairs. Therefore, if the size 
$K$ of these matrices is even, then we may re-number the index set
$\mathcal{I}_K$ as $\mathcal{I}_K=\{-K/2,\ldots,-1,1,\ldots ,K/2\}$ such that
$\lambda_{-n}=-\lambda_n$ and
$\vecz_{-n}=\overline{\vecz_n}$. If $K$ is odd then $\lambda_0=0$ is
also an eigenvalue with a \emph{real-valued} eigenvector
$\vecz_0$. Consequently, 
the representations of the measures $\Real\mu_{jk}$ and $\Imag\mu_{jk}$,
following from the functions in equation~\eqref{eq:Signed_mu_Dis},
can be simplified \cite{Pavliotis:PHD_Thesis} to depend only
on the restricted index set $\{n\geq0:\lambda_n\leq\lambda\}$. This is clear from
equations~\eqref{eq:Integral_Rep_kappa*} and~\eqref{eq:Signed_mu_Dis}, 
since for $n\geq0$ we have $\lambda_{-n}^2=(-\lambda_n)^2=\lambda_n^2$ and
$\vecz_{-n}=\overline{\vecz_n}$, thus
$\Qm_{-n}=\overline{\Qm}_n$. It follows that
%
\begin{align}\label{eq:Real_Spectral}
  &\Real[\nabla\Qm_n\vecg_j\bcdot\nabla\vecg_k]
  +\Real[\nabla\Qm_{-n}\vecg_j\bcdot\nabla\vecg_k] 
  =2\Real[\nabla\Qm_n\vecg_j\bcdot\nabla\vecg_k]
  \\
  &\lambda_n\Imag[\nabla\Qm_n\vecg_j\bcdot\nabla\vecg_k]
  +\lambda_{-n}\Imag[\nabla\Qm_{-n}\vecg_j\bcdot\nabla\vecg_k]
  =2\lambda_n\Imag[\nabla\Qm_n\vecg_j\bcdot\nabla\vecg_k]\,,
  \notag
\end{align}
%
with $\lambda_0\text{Im}[\Qm_0\vecg_j\bcdot\vecg_k]\equiv0\,$. For
numerical computations of statistical properties of advection enhanced
diffusion by randomly perturbed periodic flows, a useful consequence
of this is only \emph{half} of the eigenvalues and spectral weights
need to be held in memory, as the other half are redundant, which
saves memory consumption.



% As the projection matrix $\Qm_n$ is self-adjoint with respect to the
% discrete inner-product $\langle\cdot,\cdot\rangle_{1,2}$ defined
% above, we have $[\nabla\Qm_n\vecg_k\bcdot\nabla\vecg_j]
% =[\nabla\overline{\Qm_n}\vecg_j\bcdot\nabla\vecg_k]$.







\section{Spectral computations of the effective diffusivity matrix} \label{sec:Num_Results}
%
In \secref{sec:Matrix_Sobolev}, we developed a mathematical framework
that provides discrete Stieltjes integral representations for the  
symmetric $\Sg^*$ and antisymmetric $\Ag^*$ parts of the effective
diffusivity matrix $\Dg^*$. These discrete integral representations
can be written as the series shown in
equation~\eqref{eq:Discrete_Integrals_full_rank_Sobolev} or the integrals
shown in~\eqref{eq:Integral_Rep_kappa*}, involving the discrete
spectral measure in~\eqref{eq:Disc_Spec_Measure_Matrix_Sobolev}
with the appropriate notational changes described in
\thmref{thm:Int_Rep_Sobolev_Matrix}. This framework assumes that the
matrix gradient $\nabla$ has Dirichlet boundary conditions, for
example, so the matrix is \emph{full-rank} and the negative matrix
Laplacian $\nabla^T\nabla$ is invertible. However, for our studies of
advection enhanced diffusion by spatially periodic fluid velocity
fields, we need to use a matrix gradient $\nabla$ with periodic boundary
conditions and this matrix is
\emph{rank-deficient}. In order to streamline our presentation leading
to the numerical results of the current section,
in~\appref{app:Eigenvalue_method} we extend the discrete mathematical
framework developed \secref{sec:Matrix_Sobolev} to the rank-deficient
setting. This analysis, in the proof
of~\thmref{thm:Spectral_Equivalence_Rank_Def} below, 
shows the discrete Stieltjes integral 
representations for $\Sg^*$ and $\Ag^*$ are given by
%the formulas in equation~\eqref{eq:Discrete_Integrals_rank_deficient} of
%\thmref{thm:Spectral_Equivalence_Rank_Def}, which are
direct analogues
of the formulas in equation~\eqref{eq:Discrete_Integrals_full_rank_Sobolev}. 




In this section, we use
these direct analogues of the formulas in
equation~\eqref{eq:Discrete_Integrals_full_rank_Sobolev} 
%equation~\eqref{eq:Discrete_Integrals_rank_deficient}
to compute
$\Dg^*$ for various model periodic flows and randomly perturbed
periodic flows. As a benchmark test, we compute the spectral measure
and $\Dg^*$ for a shear flow, which are known in closed
form~\cite{Avellaneda:CMP-339}, and a cell flow with closed
streamlines, for which it is
known~\cite{Fannjiang:1994:SIAM_JAM:333,Fannjiang:1997:1033} that
$\Dg^*\sim\varepsilon^{1/2}$ for $\varepsilon\ll1$. Our numerical
results are in good 
agreement with the theoretical results. We also consider  
a fluid velocity field that has a free parameter. As the parameter
varies, the flow transitions from cell flow with closed streamlines to
shear flow in the diagonal $x$-$y$ direction. This gives rise to
 transitional behavior in the spectral measure, which
governs transitional behavior in the effective diffusivity
matrix~$\Dg^*$. For the sake of brevity, we will focus our attention
on the $\varepsilon$-behavior of the components $\Sg^*_{jk}$,
$j,k=1,\ldots,d$, of $\Sg^*$. Also, for computational simplicity,
we have restricted our computations to dimension $d=2$.





\subsection{Numerical Methods}\label{sec:Numerical_Methods}
%
By equation~\eqref{eq:u_DH}, the time-independent fluid velocity field
$\vecu=\vecu(\vecx)$ is given in terms of an antisymmetric matrix
$\Hm=\Hm(\vecx)$, $\vecu=\bnabla\bcdot\Hm$. For dimension $d=2$, the
matrix $\Hm$ is determined by a stream function $\Psi=\Psi(\vecx)$,   
%
\begin{align}\label{eq:H_2D}
  \Hm=
  \left[
  \begin{array}{cc}
    0&\Psi\\
    -\Psi&0  
    \end{array}
\right],
\end{align}
%
yielding $\vecu=[-\partial_y\Psi,\partial_x\Psi]$, where $\partial_x$ denotes
partial differentiation in the variable $x$, for example. In this
section we consider two flows with free parameters which transition
from cell flow to shear flow as parameters vary. In particular, we
consider  BC-flow~\cite{Biferale:PF:2725} and ``cat's eye''
flow~\cite{Fannjiang:1994:SIAM_JAM:333}, which are defined by the
following stream functions $\Psi_{BC}$ and $\Psi_{CE}$, respectively,    
%
\begin{align}\label{eq:Stream_Functions}
  \Psi_{BC}(\vecx)=B\sin{x}-C\sin{y},
  \qquad
  \Psi_{CE}(\vecx)=\sin{x}\sin{y}+\alpha\cos{x}\cos{y},
\end{align}
%
where we have denoted $\vecx=(x,y)$. The corresponding fluid velocity
fields are 
%
\begin{align}\label{eq:BC_CE_velocity_fields}
  \vecu_{BC}(\vecx)&=(C\cos{y},B\cos{x})\,,
  \\
  \vecu_{CE}(\vecx)&=(-\sin{x}\cos{y}+\alpha\cos{x}\sin{y}\,,
            \cos{x}\sin{y}-\alpha\sin{x}\cos{y})\,.
  \notag
\end{align}
%
The flow geometry of these fluid velocity fields transition from shear
flow to cell flow structure as the system parameters $\alpha,B,C\in[0,1]$
vary.


%
%
\begin{figure}[t]
  \centerline{\includegraphics[scale=0.85]{Figure1_Stream_Function_Contours}} 
\caption{%
  Transitions in flow structure. The streamlines
  for BC-flow and cat's eye flow are displayed for various
  parameter values, transitioning from shear to cell flow
  structure. From left to right and top to bottom, the parameter
  values associated with BC-flow are $B=1$ fixed
  and $C=0,\;0.01,\;0.1,\;0.3,\;0.5$ and $1$, while those for cat's
  eye flow are $\alpha=0,\;0.1,\;0.3,\;0.5,\,0.7$, and $1$. 
        }
\label{fig:Figure1_Stream_Function_Contours}
\end{figure}
%



Streamlines of a 2D flow are level sets of the stream function
$\Psi$, which define a family of curves that are instantaneously tangent
to the fluid velocity field $\vecu$, since $\vecu=[-\partial_y\Psi,\partial_x\Psi]$
implies that
$\vecu\bcdot\bnabla\Psi=0$. In~\figref{fig:Figure1_Stream_Function_Contours},   
we display streamlines of the flows
in equation~\eqref{eq:Stream_Functions} for various parameter values.
The streamlines for cat's eye flow are symmetric
about the line $y=x$, which follows from the symmetry of the stream
function $\Psi_{CE}(x,y)=\Psi_{CE}(y,x)$. The stream function
for BC-flow has a more complicated symmetry
$\Psi(x,y;B,C)=-\Psi(y,x;C,B)$. This symmetry indicates that if the values
of $B$ and $C$ are interchanged, $B\longleftrightarrow C$, then the original flow is
recovered from a $90^\circ$ rotation $(x\to y,\;y\to-x)$ followed by a
reflection about the $x$-axis $(y\to-y)$. Consequently, flows elongated
in the $y$-direction become flows elongated in the $x$-direction under
the interchange $B\longleftrightarrow C$. Consistently, our numerical
computations of $\mu_{jk}$ and $\Sg_{jk}^*$, $j,k=1,2$, exhibit these
symmetries. For $BC$-flow, these symmetries allow us to restrict our
attention to the behavior of $\mu_{jk}$ and $\Sg_{jk}^*$ as only one
parameter varies. Here we discuss our results for cat's eye flow for
various (deterministic) values of the parameter $\alpha$ as well as $\alpha$ uniformly
distributed on the interval $[0,p]$ for various values of $p$.
%, having second moment $p^2/3$.
For the sake of brevity, we consider $BC$-shear flow in the $x$ and
$y$-directions only, which are obtained for parameter values
$(B,C)=(0,1)$ and $(B,C)=(1,0)$, respectively, and do not display our
results for the transitional behavior from one extreme to the
other. The spectral measure and $\Dg^*$ were computed for $BC$-cell
flow in~\cite{Murphy:ADSTPF-2017}.  


% In equation~\eqref{eq:Peclet_eps} and the paragraph therein we
% discussed our non-dimensionalization the
% advection diffusion equation  
% in~\eqref{eq:ADE}. In particular, we mapped $\vecu$ to the
% non-dimensional fluid velocity field $\vecu\mapsto\vecu/\|\vecu\|$ and $\varepsilon$ to
% the non-dimensional molecular diffusivity $\varepsilon\mapsto\varepsilon/(\ell\|\vecu\|)$, 
% where $\ell$ is the maximum cell period and $\|\vecu\|=\langle|\vecu|^2\rangle^{1/2}$
% is the Hilbert space norm of $\vecu$. For the fluid velocity fields
% in equation~\eqref{eq:BC_CE_velocity_fields}, $\Vc=[0,2\pi]^2$ and
% $\ell=2\pi$.
% When $\vecu$ in equation~\eqref{eq:BC_CE_velocity_fields} is
% non-random then the underlying Hilbert space is $\Hc=L^2(\Vc,\text{m})$,
% where $\nu$ denotes the normalized Lebesgue measure (uniform
% distribution) on $\mathbb{R}^d$, restricted to $\Vc$, and $\|\cdot\|$
% denotes the $\Hc$-norm, with 
% % 
% \begin{align}\label{eq:Norm_u2}
%   \|\vecu_{BC}\|^2=\frac{B^2+C^2}{2}\,,
%   \qquad
%   \|\vecu_{CE}\|^2=\frac{1+\alpha^2}{2}\,.
% \end{align}
% %



In \secref{sec:Integral_Reps_Sobolev}, we gave an overview of the effective
parameter problem for the setting of randomly perturbed, $\Vc$-periodic
flows and introduced the Hilbert space $\Hs$
in equation~\eqref{eq:Hilbert_Sobolev}. Numerically, it 
is natural to set $\Hs$ to be the space of randomly perturbed 
$\Vc$-periodic functions, $\Hs=L^2(\text{m}\times
P)$, where $P$ is the probability measure associated with the random 
variable $\alpha$~\cite{Khoshnevisan:2007:Prob}. In this case, by Fubini's
theorem~\cite{Folland:99:RealAnalysis}, $\langle\cdot\rangle$ can be
considered to denote spatial averaging followed by statistical
averaging.
% and 
% the formula for $\|\vecu_{CE}\|^2$ in~\eqref{eq:Norm_u2} is
% replaced by    
% %
% \begin{align}\label{eq:Norm_u2_Unif0p}  
%   \|\vecu_{CE}\|^2=\frac{3+p^2}{6}\,.  
% \end{align}
% %




We now discuss in more detail our discrete, matrix formulation of the
effective parameter problem. To illustrate 
how to generalize these ideas to dimension $d$ larger than $d=2$, we
will maintain aspects of our general 
notation. In this discrete setting, the spatial region $\Vc=[0,2\pi]^d$,
for example, is replaced by a square lattice $\Vc_L^d$ of size $L$
containing $L^d$ equally spaced points in $\Vc$. As discussed
in \secref{sec:Matrix_Sobolev}, 
the differential operators $\bnabla$ and $\bnabla\bcdot$ are replaced
by finite difference, matrix operators $\nabla$ and $-\nabla^T$, respectively,
with suitable boundary conditions. Periodic boundary conditions will
be assumed throughout this section. Since these matrices operate on
vectors, the $d$-dimensional lattice $\Vc_L^d$ must be bijectively
mapped to a one dimensional lattice $\Vc_N$ of size
$N=L^dd$. The specific structure of
$\Vc_N$ and $\nabla$ depend on the bijective mapping $\Theta$ chosen. In our
computations discussed in this section, we used the mapping $\Theta$
described in~\cite{Murphy:2015:CMS:13:4:825}.





The spatially dependent $d$-dimensional
vector field $\vecv(\vecx)=(v_1(\vecx),\ldots,v_d(\vecx))$, say, is
bijectively mapped by $\Theta$ to a discretized \emph{constant} vector 
$(\vecv_1,\ldots,\vecv_2)$ with $N$ elements, where the vectors $\vecv_i$,
$i=1,\ldots,d$, each have $L^d$ elements and contain all of
the spatial information about the $v_i(\vecx)$ for $\vecx\in\Vc_L^d$.
Similarly, the $d$-dimensional standard basis vector 
$\vece_1=(1,0,\ldots,0)$ is mapped to the $N$-dimensional vector
$(\mathbf{1},\mathbf{0},\ldots,\mathbf{0})$, where $\mathbf{1}$ and
$\mathbf{0}$ are vectors of ones and zeros with $L^d$ elements, and
similarly for the $\vece_j$ for $j=2,\ldots,d$. Therefore, the vectors
$\hate_j$, $j=1,\ldots,d$, satisfying      
%
\begin{align}\label{eq:Lattice_Basis_e}
  \hate_j=\Theta(\vece_j)/L^{d/2},
  \qquad
  \hate_j\cdot\hate_k=\delta_{jk}\,,
\end{align}
%
serve as ``lattice standard basis vectors.''
% In previous sections, we
% deferred the description of these vectors to the present section and, for
% simplicity, used the notation $\vece_j$. Now that the specific nature
% of these vectors has been discussed, we will henceforth use the
% notation in equation~\eqref{eq:Lattice_Basis_e}.
With this convention, the division by $L^{d/2}$
in~\eqref{eq:Lattice_Basis_e} takes care of the uniform $L$-scaling in
discrete approximations of spatial integrals; instead of
the $(2\pi)^d$ normalized 
Lebesgue measure $\d\vecx$ we have $\Delta\vecx=(2\pi/L)^d/(2\pi)^d$
when $\Vc=[0,2\pi]^d$ and the spatial 
average $\langle\xi(\vecx)\,\zeta(\vecx)\rangle_{\Vc}$ becomes
$\vecxi\bcdot\veczeta/L^d$. In a similar way, the $2\times2$ matrix
$\Hm(\vecx)$ in~\eqref{eq:H_2D} becomes a $N\times N$ antisymmetric
banded matrix, where the stream function 
$\Psi(\vecx)$ is represented by a diagonal $L^d\times L^d$ matrix and
the zero element $0$ is represented by a matrix of zeros.
% In higher dimensions $d>2$ the discrete representation of
% the matrix $\Hm$ is also banded.
As in previous sections, for
notational simplicity, we will not make a notational distinction for
the matrix $\Hm$ between the continuum and discrete settings as the
context will be clear.  





 





In \thmref{thm:Spectral_Equivalence_Rank_Def} of
\appref{app:Eigenvalue_method} below we extend our results developed 
in \thmref{thm:Int_Rep_Sobolev} of \secref{sec:Integral_Reps_Sobolev} to the
setting where the matrix gradient $\nabla$ has periodic boundary
conditions and is rank-deficient. This is accomplished by considering
the singular value decomposition (SVD) $\nabla=\Um\Sigma\Vm^T$.
Here, $\nabla$ is of size $N\times K$, say, where $K=L^d$ and
$N=Kd$. Also, 
$\Sigma=\text{diag}(\sigma_1,\ldots,\sigma_K)$, where
$0\leq\sigma_1\leq\cdots\leq\sigma_K$. The matrices $\Um$ and
$\Vm$ are of size $N\times K$ and $K\times K$, respectively, and
satisfy $\Um^T\Um=\Ib$ and
$\Vm^T\Vm=\Vm\Vm^T=\Ib$, where $\Ib$ is the
identity matrix of size $K$~\cite{Demmel:1997}. When $\nabla$ is of full
rank, then the singular values satisfy $\sigma_j>0$ for all
$j=1,\ldots,K$ and the matrix Laplacian
$\nabla^T\nabla=\Vm\Sigma^2\Vm^T$ is invertible. 
When $\nabla$ is rank-deficient, then there are $K_1$ non-zero
singular values and $K_0=K-K_1$ zero singular values, say. For
example, when $d=2$ the nullity of $\nabla$ is 1 so $K_1=K-1$. In
this case, we write $\Um=[\Um_0\;\Um_1]$,
$\Sigma=\text{diag}(\Om,\Sigma_1)$, and $\Vm=[\Vm_0\;\Vm_1]$, where
$\Om$ is a matrix of zeros, so that
$\nabla=\Um_1\Sigma_1\Vm_1^T$ and the negative matrix Laplacian
$\nabla^T\nabla=\Vm_1\Sigma_1^2\Vm_1^T$ is non-invertible, since
$\Vm_1\Vm_1^T\ne\Ib\,$.   



The associated matrix analysis in~\appref{app:Eigenvalue_method}
demonstrates that the spectral measure $\mu_{jk}$ underlying the Stieltjes
integral representation of $\Sg^*_{jk}$ is given by
%
\begin{align}\label{eq:Measure_weights_mjk}
   \d\mu_{jk}(\lambda)=\sum_{n:\;\lambda^1_n\leq\lambda}\langle m_{jk}(n)\,\delta_{\lambda^1_n}(\d\lambda)\rangle,
 \end{align}
%
where $\lambda_n^1$, $n=1,\ldots,K_1$, are the eigenvalues of the matrix
$-\imath\Um_1^T\Hm\Um_1$,
%in~\eqref{eq:Eig_Decomp_PHP_U1},
while various
equivalent representations of the 
spectral weights $m_{jk}(n)$, $j,k=1,\ldots,d$, are given in
equation~\eqref{eq:Measure_Weight_Equivalence_Gen}
of~\thmref{thm:Spectral_Equivalence_Rank_Def} below. For notational 
simplicity, in this section we denote $\Real\mu_{jk}$ by $\mu_{jk}$.
In our computations of $\mu_{jk}$ we used 
%
\begin{align}\label{eq:Spectral_Weights}
  m_{jk}(n)=\Real[\,\overline{([\vecr_n^1]^\dagger\Um_1^T\Hm\hate_j)}\,
                         ([\vecr_n^1]^\dagger\Um_1^T\Hm\hate_k)\,],
  \quad
  n=1,\ldots,K_1,
\end{align}
%
which follows from
equations~\eqref{eq:Sobolev_weights_tfm},~\eqref{eq:Signed_mu_Dis},~\eqref{eq:Evector_relation_U1},  
and~\eqref{eq:Measure_Weight_Equivalence_Gen}. Here 
$\vecr_n^1$, $n=1,\ldots,K_1$, are the \emph{complex} eigenvectors of the
matrix $-\imath\Um_1^T\Hm\Um_1$. Consequently $m_{kk}(n)\ge0$, so
$\mu_{kk}$ is a positive measure and $\mu_{jk}$, $j\neq k$, is a
signed measure, where $m_{jk}(n)$ in~\eqref{eq:Spectral_Weights} can
take positive or negative values.




To reveal the structure of $\mu_{12}$ and $\Sg^*_{12}$ in our
numerical computations discussed in \secref{sec:Numerical_RESULTS}
below, we denote the spectral weights $m_{jk}(n)$ associated 
with the Jordan decomposition $\mu_{jk}=\mu_{jk}^+-\mu_{jk}^-$
in~\eqref{eq:Total_Variation} by $m_{jk}^+(n)$ and $m_{jk}^-(n)$,
where $m_{jk}^\pm(n)\geq0$. Also, we define the
functions
%~\cite{Folland:99:RealAnalysis} 
%of $\varepsilon$, 
%$[\Sg^*_{12}]^+=\max\{\Sg^*_{12}(\varepsilon),0\}$ and
%$[\Sg^*_{12}]^-=\max\{-\Sg^*_{12}(\varepsilon),0\}$     
$[\Sg^*_{12}]^+$ and $[\Sg^*_{12}]^-$    
%
\begin{align}
  [\Sg^*_{12}]^+(\varepsilon)=\max\{\Sg^*_{12}(\varepsilon),0\},
  \quad
  [\Sg^*_{12}]^-(\varepsilon)=\max\{-\Sg^*_{12}(\varepsilon),0\},
\end{align}
%
for each $0<\varepsilon<\infty$, so that 
$\Sg^*_{12}=[\Sg^*_{12}]^+-[\Sg^*_{12}]^-$,
$[\Sg^*_{12}]^\pm(\varepsilon)=\Sg^*_{12}(\varepsilon;\mu_{12}^\pm)$,  and $[\Sg^*_{12}]^\pm\geq0$. 




In the case of a non-random fluid velocity field $\vecu$, we used
$L=200$ so that $K_1=39,999$. The eigenvalues $\lambda_n^1$ and eigenvectors
$\vecr_n^1$ of the non-random Hermitian matrix $-\imath\Um_1^T\Hm\Um_1$
were computed using the Matlab function \emph{eig()}. In this case,
the averaging $\langle\cdot\rangle$ in~\eqref{eq:Measure_weights_mjk} is interpreted
as spatial averaging over the period cell $\Vc$. In the setting of a
randomly perturbed flow, we
used $L=100$ so that $K_1=9,999$. In this case, the averaging $\langle\cdot\rangle$
in~\eqref{eq:Measure_weights_mjk} is interpreted as spatial averaging
followed by ensemble averaging over $\sim10^3$ statistical trials.



The numerical accuracy of the eigenvalue problem
is determined by the \emph{eigenvalue condition numbers}
$\mathcal{K}(\lambda_n^1)$, $n=1,\ldots,K_1$, which are the reciprocals of the cosines
of the angles between the left and right eigenvectors. Large
eigenvalue condition numbers of a Hermitian matrix implies that it is
near a matrix with multiple eigenvalues, while eigenvalue condition
numbers close to 1 imply that the eigenvalue problem is
well-conditioned. The eigenvalue problem for the matrix
$-\imath\Um_1^T\Hm\Um_1$ is extremely well conditioned with
$\max_n|1-\mathcal{K}(\lambda_n^1)|\sim10^{-14}$ typical, computed
using Matlab's function \emph{condeig()}.



To our knowledge, Matlab does  not provide a function that describes
the accuracy of the computed SVD of the matrix
$\nabla=\Um\Sigma\Vm^T$. In order to better understand the numerical
accuracy in the entries of the matrix $\Um$, which is central to our
computational method, we performed the following tests. For the case
of Dirichlet boundary conditions, the matrix $\nabla$ is full-rank,
hence the matrix Laplacian $\nabla^T\nabla$ is invertible. We computed
the matrix $\Gamma=\nabla(\nabla^T\nabla)^{-1}\nabla^T$ directly using
Matlab's \emph{mldivide} function, i.e.,
$\Gamma=\nabla((\nabla^T\nabla)\backslash \nabla^T)$, and also using
the SVD of the matrix $\nabla$, with $\Gamma=\Um\Um^T$. We then
computed the component-wise maximum difference $\max_{l
  m}|[\nabla((\nabla^T\nabla)\backslash\nabla^T)-\Um\Um^T]_{l
  m}|$. When $L=100$ and $L=200$ this difference is~$\sim10^{-15}$,
which gives an idea of the accuracy of the SVD of $\nabla$ for the
rank-deficient, periodic setting. The matrix $\Gamma$ is used
extensively throughout \appref{app:Matrix_Formulation_Curl}. In 
all of our computations, Matlab's sparse architecture was employed
wherever possible to reduce  roundoff error.




\subsection{Numerical Results}\label{sec:Numerical_RESULTS}
%
Before we discuss our numerical results in this section, it is
helpful to first describe the relationship between the spectral
measure $\mu_{jk}$ and the $\varepsilon$-behavior of
$\Sg_{jk}^*$. This relationship is easiest to understand when $j=k$,
in terms of the enhancement in scalar transport above the
bare diffusive
value $\varepsilon$, as shown in
equation~\eqref{eq:Skk}. Consider  $\mu_{kk}$ and $\Sg^*_{kk}$, for
some $k=1,\ldots,d$, and write the formula in~\eqref{eq:Skk} as   
%
\begin{align}\label{eq:Enhancement}
\Sg^*_{kk}(\varepsilon)=\varepsilon  + \mathcal{E}_{kk}(\varepsilon),
\quad
\mathcal{E}_{kk}(\varepsilon)
=\varepsilon\int_{-\infty}^\infty
\frac{\d\mu_{kk}(\lambda)}{\varepsilon^2+\lambda^2}\,, 
\end{align}
%
where $\mathcal{E}_{kk}(\varepsilon)\ge0$ denotes the \emph{enhancement} above
$\varepsilon$. 
By equation~\eqref{eq:Spectral_Radius_Sobolev}, the spectrum $\Sigma$
of the self-adjoint operator $M$ is a bounded subset of
$\mathbb{R}$. Consequently, in the \emph{diffusion dominated regime}
where $\varepsilon\gg|\lambda|$ for all $\lambda\in\Sigma$, we
have~\cite{Folland:99:RealAnalysis} 
$\Sg^*_{kk}\sim \varepsilon +\mu_{kk}^0/\varepsilon\,.$ The enhancement
$\mathcal{E}_{kk}(\varepsilon)\sim\mu_{kk}^0/\varepsilon$ is only nominal
in this regime where $\varepsilon\gg1$ and is independent of the
distribution of measure mass --- dependent only the total mass.
%somewhat independent of the details of the spectral measure
%$\mu_{kk}$ --- depending only on the associated eigenvectors.
However, in the
\emph{advection dominated regime} where  $\varepsilon\ll1$, if the
spectral measure $\mu_{kk}$ has significant mass near the spectral
origin $\lambda=0\,$, e.g., if the spectral weights $m_{kk}(n)$
in~\eqref{eq:Measure_weights_mjk} have values significantly greater
than zero for
$|\lambda_n|\ll1\,$, then the integrand associated with
$\mathcal{E}_{kk}(\varepsilon)$ can introduce singular behavior that
competes with the small $\varepsilon$ prefactor in front of the
integral, giving rise to a significant
enhancement $\mathcal{E}_{kk}(\varepsilon)$ for
$0<\varepsilon\ll1$. Although, if the 
mass of the measure is zero or extremely small in a neighborhood of
$\lambda=0$, then the enhancement $\mathcal{E}_{kk}(\varepsilon)$ can be less
pronounced, since the small $\varepsilon$ prefactor dominates the
(lack of) singular behavior in the integrand near $\varepsilon=0$. This
illustrates that in the advection dominated regime, where
$0<\varepsilon\ll1$, the $\varepsilon$-behavior of the enhancement
$\mathcal{E}_{kk}(\varepsilon)$ depends strongly on the details of the
spectral measure 
$\mu_{kk}$ near $\lambda=0$.



We emphasize that, due to roundoff error and finite resolution
($L<\infty$) effects, our numerical approximations of $\mu_{jk}$ and 
the $\varepsilon$-behavior of $\Sg_{jk}$ breakdown for extremely
small values of $\varepsilon$. For example, in the discrete setting,
it is highly unlikely that $\lambda_n=0$ is (exactly) a numerical
eigenvalue solution of equation~\eqref{eq:Strong_eval_prob}, even
though in the continuum setting $\lambda=0$ \emph{is an accumulation
point}~\cite{Keener-2000,Stakgold:BVP:2000} of the discrete spectra for
the compact~\cite{Bhattacharya:AAP:1999:951,Bhattacharya:1989:ASD} operator $M$. Therefore,
our numerical simulations can probe the $\varepsilon$-behavior of
$\Sg_{jk}$ for moderately small values of $\varepsilon$ but the approximation
ultimately breaks down in the limit $\varepsilon\to0$.
However, for moderately small values of $\varepsilon$, our description
above regarding  the 
relationship between $\mu_{jk}$ and the $\varepsilon$-behavior of
$\Sg_{jk}$ is still valid --- illustrating that the details of the
spectral measure $\mu_{jk}$ near the spectral origin $\lambda=0$
strongly influence the $\varepsilon$-behavior of
$\Sg_{jk}$ when $\varepsilon\ll1$.




These concepts are illustrated in our computations of $\mu_{jk}$ and
$\Sg_{jk}$ for ``cat's eye'' flow displayed
in~\secref{sec:Cats_Eye_Flow} below. As the free
parameter $\alpha$ increases from 0 to 1, the flow transitions from
cell flow with closed streamlines to shear 
flow in the $y=x$ direction, as shown
in~\figref{fig:Figure1_Stream_Function_Contours}. Our computations of
$\mu_{jk}$ display a 
 transitional behavior near $\lambda=0$ 
which gives rise to a pronounced change in the $\varepsilon$-behavior
of $\Sg_{jk}$ near $\varepsilon=0$, as well as a significant
enhancement in $\Sg_{kk}$ above the bare diffusive value $\varepsilon$. 







As a benchmark result, we demonstrate in~\secref{sec:BC_Shear_Flow}
that our computations accurately capture the known behavior of
$\mu_{kk}$ and $\Dg_{kk}^*$ for shear
flow in the $k$th direction~\cite{Avellaneda:CMP-339}, where
$\mu_{kk}=\mu_{kk}^0\,\delta_0(\d\lambda)$ and
$\Dg_{kk}^*=\varepsilon+\mu_{kk}^0/\varepsilon$. As another benchmark result, we
demonstrate in~\secref{sec:Cats_Eye_Flow} that our computations
accurately capture the
known~\cite{Fannjiang:1994:SIAM_JAM:333,Fannjiang:1997:1033}
asymptotic behavior 
$\Dg_{kk}^*\sim\varepsilon^a$ with critical exponent $a=1/2$, for
$\varepsilon\ll1$. In particular, the numerical methods developed in
this manuscript 
compute $a\approx0.54$, with a $8\%$ error relative to the true
value. For the sake of comparison, our Fourier approach to computing
the spectral measure $\mu_{kk}$ discussed in~\cite{Murphy:ADSTPF-2017}
computes $a\approx0.52$, with a $4\%$ relative error, and our
implementation of the linear systems approach to computing $\Dg_{kk}^*$
discussed in~\cite{Pavliotis:PHD_Thesis} computes $a\approx0.49$ with a
$2\%$ relative error. This marked increase in the accuracy of the linear
systems approach is largely due to the ability to handle larger matrix
sizes which, in turn, is due to the $O(N^2)$ numerical complexity of
the method compared to the $O(N^3)$ numerical complexity of the
spectral measure method. 
   

   







\subsubsection{$BC$-shear flow}\label{sec:BC_Shear_Flow}
%
In the continuum setting, it is known~\cite{Avellaneda:CMP-339} for
shear flow in the $x$-direction that the measure $\mu_{11}$ is given by 
a $\delta$-measure concentrated at the spectral origin, while
$\mu_{22}\equiv0$, and similarly for shear flow in the $y$-direction.
As a baseline result, we computed the spectral measures and effective
diffusivities for BC-shear-flow in both the $x$ and $y$-directions,
which are obtained for parameter values $(B,C)=(0,1)$ and
$(B,C)=(1,0)$, respectively. Our computations for the components
$\mu_{jk}$, $j,k=1,2$, of the spectral measure for BC-shear-flow 
displayed in~\figref{fig:Figure2_Shear_Flow_Baseline} are in good agreement
with this theoretical prediction in~\cite{Avellaneda:CMP-339}. 




%
\begin{figure}[t]
  \centerline{\includegraphics[scale=0.81]{Figure2_Shear_Flow_Baseline}} 
\caption{%
  Shear flow baseline result. The streamlines of BC-shear-flow in (a) the
  $x$-direction and (b) the $y$-direction. (c) The $\varepsilon$
  behavior of  $\Sg^*_{jk}$, $j,k=1,2$, for shear flow in the
  $x$-direction. The weights $m_{jk}$ of the spectral measure
  $\Real\mu_{jk}$ for shear flow in (d) the $x$-direction and (e) the
  $y$-direction as a function of $\lambda$. Consistent with theoretical 
  predictions, the measure associated with the direction of the flow
  resembles a delta measure centered at the origin, while the other
  two components have spectral weights $m_{jk}$ with very small
  magnitudes.    
        }
\label{fig:Figure2_Shear_Flow_Baseline}
\end{figure}
%


\figref{fig:Figure2_Shear_Flow_Baseline} displays the streamlines for
$BC$-shear-flow in (a) the $x$-direction and (b) the 
$y$-direction. In  \figref{fig:Figure2_Shear_Flow_Baseline}c the components
$\Sg_{jk}^*$, $j,k=1,2$, of the effective diffusivity matrix are
displayed for $BC$-shear-flow in the $x$-direction. The analogous
result for $BC$-shear-flow in the $y$-direction is visually identical
to \figref{fig:Figure2_Shear_Flow_Baseline}c under the mapping
$\Sg_{11}^*\leftrightarrow\Sg_{22}^*$, i.e., under the exchange of the colors
black $\leftrightarrow$ blue. The components $\mu_{jk}$, $j,k=1,2$, of the 
spectral measure are displayed for $BC$-shear-flow in (d) the
$x$-direction and (e) the $y$-direction. 




We focus our discussion on the results for $BC$-shear-flow in the
$x$-direction, as the discussion regarding $BC$-shear-flow in the
$y$-direction is analogous. 
For all $n=1,\ldots,K_1$, the spectral weights $m_{22}(n)$ in
\figref{fig:Figure2_Shear_Flow_Baseline}d
%associated with the $y$-direction
satisfy $m_{22}(n)\lesssim10^{-29}$. With the effects of finite
resolution ($L<\infty$) and roundoff error associated with a machine
epsilon of $\sim10^{-16}$, these spectral weights can be considered
``numerically zero.'' The spectral weights $m^\pm_{12}(n)$ satisfy
$m^\pm_{12}(n)\lesssim10^{-28}$ for $\lambda_n$ away from the spectral
origin $\lambda=0$ with a peak near $\lambda=0$ having magnitudes
$m^\pm_{12}(n)\lesssim10^{-16}$. The spectral weights for 
the $x$-direction satisfy 
$m_{11}(n)\lesssim10^{-28}$ away from $\lambda=0$, while the weights
near $\lambda=0$ satisfy $10^{-9}\lesssim m_{11}\lesssim10^{-1}$, resembling
a $\delta$-measure with virtually all of its mass concentrated near
$\lambda=0$. This is consistent with theoretical
predictions~\cite{Avellaneda:CMP-339}. Due to the anti-symmetry of the 
\emph{real-valued} matrix 
$\Um_1^T\Hm\Um_1$, its complex eigenvectors and purely imaginary eigenvalues
come in complex conjugate
pairs~\cite{Horn_Johnson-1990}. Consequently, the eigenvalues of the 
Hermitian matrix $-\imath\Um_1^T\Hm\Um_1$ come in
%positive-negative
$\pm$
pairs
with identical spectral weights, resulting in the symmetry about 
$\lambda=0$ displayed by the spectral measures
in~\figref{fig:Figure2_Shear_Flow_Baseline}.   



Due to the high concentration of measure mass in $\mu_{11}$ very near the
spectral origin, our computation of $\Sg^*_{11}$ displayed in
\figref{fig:Figure2_Shear_Flow_Baseline}c behaves like it's being
governed by a delta function concentrated at the origin. In
particular, \figref{fig:Figure2_Shear_Flow_Baseline}c shows that the
computed $\varepsilon$-behavior of $\Sg^*_{11}$, displayed in black
color with solid line-style, lays right on top of the graph
of its upper bound $\varepsilon\,[1+\mu_{11}^0/\varepsilon^2]$ given
in~\eqref{eq:Lower_Upper_Bounds_Skk}, with
$\mu_{11}^0\approx4.975\times10^{-1}$,  displayed in black color and
dash-dot line-style. (We had to increase the line-width of  the upper
bound to be able to distinguish between the two black lines.) Due to
the extremely small magnitudes of the spectral weights $m_{22}$ and
$m_{12}^\pm$, with measure masses
$\mu_{22}^0\approx5.33\times10^{-29}$, 
$[\mu_{12}^0]^+\approx1.03\times10^{-16}$, and
$[\mu_{12}^0]^-\approx3.34\times10^{-15}$, 
both the upper and lower bounds for $\Sg_{22}^*$ and $\Sg_{12}^*$ in
equations~\eqref{eq:Lower_Upper_Bounds_Skk}
and~\eqref{eq:Lower_Upper_Bounds_Sjk} are very close to
$\varepsilon$ and 0, respectively; the graph of $\Sg_{22}^*$ is virtually right on top
of the lower bound $\varepsilon$ in cyan color and solid line-style, and the
magnitudes of $[\Sg_{12}^*]^+$ and $[\Sg_{12}^*]^-$ are so small 
they don't even appear. Since the support of the spectral
measure is contained in the interval $[-1,1]$, the components of the
effective diffusivity approach their bare diffusive value
$\varepsilon\,\delta_{jk}$ for large $\varepsilon$, as discussed above. 





In~\cite{Murphy:ADSTPF-2017} we developed Fourier
methods for the computation of the spectral measure $\mu_{jk}$ for
$BC$-cell flow, with $B=C=1$. In particular, the eigenvalue problem
$M\varphi_n=\lambda_n\varphi_n$ associated with the operator
$M=-\imath\Delta^{-1}[\vecu\bcdot\bnabla]$ was 
transformed into an infinite algebraic system of equations defining a
discrete, generalized eigenvalue problem. The Fourier 
coefficients of the eigenfunctions $\varphi_n$, $n=1,2,3,\ldots$, for the
continuum setting comprise the components of the generalized
eigenvectors in the discrete setting.
% Moreover, motivated by the theoretical findings in the current work,
% we provided an extension of the results given here to the setting of a
% time-dependent fluid velocity field, where
% $A=\Delta^{-1}[\partial_t+\vecu\bcdot\bnabla]$ and $\partial_t$ 
% denotes partial differentiation in time. Furthermore, we used abstract
% methods of functional analysis to generalize
% \lemref{lem:Spectral_Equivalence} to the continuum, steady and dynamic
% settings.
% The Fourier methods in~\cite{Murphy:ADSTPF-2017} were also generalized
% to the setting of a time-dependent fluid velocity field.
Since we
already treated $BC$-cell flow in~\cite{Murphy:ADSTPF-2017}, and for
the sake of brevity, we now turn our attention to a discussion
regarding our numerical results for  ``cat's eye flow'' displayed
in~\figref{fig:Figure3_Transition_Away_From_CatsEye_Cell_Flow}.       



\subsubsection{Cat's eye flow}\label{sec:Cats_Eye_Flow}
%
Since the streamlines for cat's eye flow in
\figref{fig:Figure1_Stream_Function_Contours} are symmetric about the
line $y=x$ for all $\alpha\in[0,1]$, as discussed above, we anticipate
that $\mu_{11}=\mu_{22}$. Our computations of the
components $\mu_{jk}$, $j,k=1,2$, of the spectral measure shown in
Figures \numfigref{fig:Figure3_Transition_Away_From_CatsEye_Cell_Flow} and
\numfigref{fig:Figure4_Transition_Toward_CatsEye_Shear_Flow}
%(for deterministic $\alpha$)
display this symmetry. A closer look
at these figures reveals a deeper symmetry, namely that
$\mu_{11}=\mu_{22}=|\mu_{12}|$, where $|\mu_{12}|=\mu_{12}^++\mu_{12}^-$ is the
total variation of the signed measure $\mu_{12}$ introduced in
equation~\eqref{eq:Total_Variation}, i.e., superimposing the 
panels for $m_{12}^+$ and $m_{12}^-$ in Figures \numfigref{fig:Figure3_Transition_Away_From_CatsEye_Cell_Flow} and
\numfigref{fig:Figure4_Transition_Toward_CatsEye_Shear_Flow}, yield
the figure panels for 
$m_{11}$ and $m_{22}$. We have also numerically verified the behavior
$\mu_{11}=\mu_{22}=|\mu_{12}|$. 

%
\begin{figure}[t]
  \centerline{\includegraphics[scale=0.85]{Figure3_Transition_Away_From_CatsEye_Cell_Flow}} 
\caption{%
  Transition away from cat's eye cell flow. The spectral weights
  $m_{jk}$ for the components $\Real\mu_{jk}$, $j,k=1,2$, of the
  spectral measure are displayed with increasing values of the free
  parameter $\alpha$ from left to right. As the parameter $\alpha$ 
  increases, the streamlines of the flow transition away from
  cell structures to open channels. This is reflected in the
  measure by a dramatic increase in the magnitude of the spectral
  weights $m_{jk}$ associated with the ``accumulation point'' of the measure at
  $\lambda=0$, while the other weights change only slightly.   
        }
\label{fig:Figure3_Transition_Away_From_CatsEye_Cell_Flow}
\end{figure}
%

Since the operator $A=\Delta^{-1}[\bnabla\bcdot\vecu]$ is
compact~\cite{Bhattacharya:AAP:1999:951,Bhattacharya:1989:ASD}, its 
spectra is discrete with an accumulation point at the spectral
origin $\lambda=0$~\cite{Stakgold:BVP:2000}. This accumulation point behavior of
the measures $\mu_{jk}$, $j,k=1,2$, can be seen in all of the panels of
\figref{fig:Figure3_Transition_Away_From_CatsEye_Cell_Flow}. When the
parameter $\alpha=0$, the streamlines of cat's eye flow are closed cell
structures, as shown in \figref{fig:Figure1_Stream_Function_Contours},
so that large scale transport occurs only when
$\varepsilon>0$~\cite{Fannjiang:1994:SIAM_JAM:333}. In this case, the
computed ``accumulation point'' has eigenvalues $\lambda_n$ with very small magnitude
$10^{-19}\lesssim|\lambda_n|\lesssim10^{-14}$. (It is clear that a
finite number of eigenvalues does not constitute an accumulation point, but we
will use this terminology to identify the concentration of eigenvalues
near $\lambda=0$ shown in Figures
\numfigref{fig:Figure3_Transition_Away_From_CatsEye_Cell_Flow} --
\numfigref{fig:Figure4_Transition_Toward_CatsEye_Shear_Flow} 
and to set ideas.) However, the spectral measure 
weights $m_{kk}(n)$ and $m^\pm_{jk}(n)$ of the accumulation point have even
smaller magnitudes with 
$10^{-35}\lesssim m_{kk}(n)\lesssim10^{-29}$, and similarly for
$m^\pm_{jk}(n)$, as shown in
\figref{fig:Figure3_Transition_Away_From_CatsEye_Cell_Flow}. Consequently, 
this component of the spectral measure does not contribute
significantly to the enhancement
$\mathcal{E}_{jk}(\varepsilon)=\varepsilon\sum_n\,m_{jk}(n)/(\varepsilon^2+\lambda_n^2)$
of $\Sg_{jk}^*(\varepsilon)=\varepsilon\delta_{jk} +
\mathcal{E}_{jk}(\varepsilon)$. Plotting the panels of
\figref{fig:Figure3_Transition_Away_From_CatsEye_Cell_Flow} with log
scale $x$-axis reveals that there is a \emph{gap} in the
spectral measure with no spectra in the interval
$10^{-14}\lesssim|\lambda_n|\lesssim10^{-7}$, as shown
in~\figref{fig:Figure6_CatsEye_Cell_Flow_mu12_LogLog}. The other
component of 
the spectral measure has spectra in the interval
$10^{-7}\lesssim|\lambda_n|\lesssim10^{\,0}$
and weights
$10^{-37}\lesssim m_{kk}(n)\lesssim10^{-1}$. 
However, the part of this component of the spectral measure with
spectral weights having more significant magnitudes
$10^{-4}\lesssim m_{kk}(n)\lesssim10^{-1}$ are associated with
eigenvalues with magnitude $|\lambda_n|\gtrsim10^{-1}$ and consequently
also do not contribute significantly to the enhancement
$\mathcal{E}_{jk}$. 



%
\begin{figure}[t]
  \centerline{\includegraphics[scale=1]{Figure4_Transition_Away_From_CatsEye_Cell_Flow_mu12_LogLog}} 
\caption{%
  Migration of positive measure mass from the computed ``accumulation point''
  near the spectral origin. As the free parameter $\alpha$ of cat's
  eye flow increases from zero, the magnitude of the spectral weights 
  $m_{12}^\pm$ comprising the accumulation point increase
  dramatically. Moreover, the spectra associated with the positive 
  weights $m_{12}^+$ migrate away from the spectral origin until the
  accumulation point is comprised only of negative valued mass. The
  corresponding behavior for $m_{kk}$, $k=1,2$, is determined by the
  relation $\mu_{kk}=|\mu_{12}|$ between the components of the
  spectral measure $\d\mu_{jk}(\lambda)=\sum_n\langle
  m_{jk}(n)\,\delta_{\lambda^1_n}(\d\lambda)\rangle$.   
        }
\label{fig:Figure6_CatsEye_Cell_Flow_mu12_LogLog}
\end{figure}
%




When  $0<\alpha\ll1$, open channels connect neighboring cells and
large scale transport takes place both in thin boundary layers and
within the channels~\cite{Fannjiang:1994:SIAM_JAM:333}. This is
reflected in the spectral measure
$\d\mu_{jk}(\lambda)=\sum_n\langle
m_{jk}(n)\,\delta_{\lambda^1_n}(\d\lambda)\rangle$ in 
equation~\eqref{eq:Measure_weights_mjk} by a dramatic 
increase in the magnitude of the 
spectral weights $m_{kk}(n)$ and $m^\pm_{12}(n)$ associated with the
accumulation point  --- by more than \emph{14 orders of magnitude}
with $10^{-19}\lesssim m_{kk}(n)\lesssim10^{-15}$ ---
corresponding to a change of only $10^{-6}$ in the magnitude of
$\alpha$, as shown in Figures
\numfigref{fig:Figure3_Transition_Away_From_CatsEye_Cell_Flow} and
\numfigref{fig:Figure6_CatsEye_Cell_Flow_mu12_LogLog}. The 
associated change in the spectral weights away from $\lambda=0$ is
nominal. However, \figref{fig:Figure6_CatsEye_Cell_Flow_mu12_LogLog} 
reveals that as $\alpha$ increases from $0$
to $10^{-6}$ a localized portion of the accumulation point begins to migrate
away from $\lambda=0$ to the
other component of the spectral measure with spectra in the interval
$10^{-7}\lesssim|\lambda_n|\lesssim10^{\,0}$ --- decreasing the mass of
the accumulation point for $\mu_{kk}$. Moreover, all of the positive masses
$m^+_{12}(n)$ migrate away from the accumulation 
point so the accumulation point of $\mu_{12}$ becomes comprised with  purely
negative weights.  From this discussion, it
is clear that the increased magnitudes of the masses comprising the accumulation
point provide an increased contribution to the enhancement 
$\mathcal{E}_{jk}(\varepsilon)$ for $\varepsilon\sim10^{-14}$, for
example, but only a moderate increase, and the increase in
$\mathcal{E}_{jk}(\varepsilon)$ for
$10^{-7}\lesssim\varepsilon\lesssim10^{\,0}$ is also not significant.  


%
\begin{figure}[t]
  \centerline{\includegraphics[scale=0.85]{Figure5_Transition_Toward_CatsEye_Shear_Flow}} 
\caption{%
  Transition toward cat's eye shear flow. The spectral weights
  $m_{jk}$ for the components $\Real\mu_{jk}$, $j,k=1,2$, of the
  spectral measure are displayed with increasing values of the free
  parameter $\alpha$ from left to right. As the value of the parameter $\alpha$
  increases, the streamlines become more elongated in the $x$-$y$
  diagonal direction, becoming shear flow when $\alpha=1$. This is
  reflected in the spectral measure by an increase in the breadth of
  the spectral region with significant measure mass.
        }
\label{fig:Figure4_Transition_Toward_CatsEye_Shear_Flow}
\end{figure}
%




As the value of $\alpha$ increases to  $\alpha=1$,
the magnitudes of the masses comprising the accumulation point 
increase until they reach maximum values with
$10^{-11}\lesssim m_{kk}(n)\lesssim10^{-5}$,
associated with eigenvalues
with magnitudes $10^{-19}\lesssim|\lambda_n|\lesssim10^{-14}$, as
shown in \figref{fig:Figure6_CatsEye_Cell_Flow_mu12_LogLog}. This
marked increase in the magnitudes of the spectral weights provide a
significant contribution to the enhancement 
$\mathcal{E}_{jk}(\varepsilon)$ for $\varepsilon\sim10^{-14}$, for
example. Moreover, as 
$\alpha$ increases in the range $(10^{-1},10^{\,0})$, a significant
transitional behavior arises in the other component of the spectral
measure away from $\lambda=0$, as shown in Figures
\numfigref{fig:Figure6_CatsEye_Cell_Flow_mu12_LogLog} and
\numfigref{fig:Figure4_Transition_Toward_CatsEye_Shear_Flow}. In 
particular,
% plotting the panels of
% \figref{fig:Figure3_Transition_Away_From_CatsEye_Cell_Flow} with log
% scale $x$-axis (not shown),demonstrates a plateau
a bulge of measure weights with significant magnitude forms for
eigenvalues in the range
$10^{-4}\lesssim|\lambda_n|\lesssim10^{-1}$,
with a marked increase in weight magnitude from
$10^{-7}\lesssim m_{kk}(n)\lesssim10^{-4}$
to
$10^{-7}\lesssim m_{kk}(n)\lesssim10^{-2}$. This provides a
significant contribution to the enhancement
$\mathcal{E}_{jk}(\varepsilon)$ even for
$10^{-4}\lesssim\varepsilon\lesssim10^{-1}$. 




%
\begin{figure}[t]
  \centerline{\includegraphics[scale=1]{Figure6_Effective_Diffusivity_CatsEye}} 
\caption{%
  Transitional behavior of the effective diffusivity from cat's eye
  cell flow to shear flow. The behavior of the components
  $\Sg^*_{jk}$, $j,k=1,2$, of the effective diffusivity as a function
  of the non-dimensionalized molecular diffusivity $\varepsilon$ and
  increasing values of 
  the free parameter $\alpha$ from left to right and top to bottom. The
  upper and lower bounds corresponding to $\Sg^*_{jk}$ are in dash-dot
  and dashed line-style, respectively, and are the same line colors as the line
  colors for $\Sg^*_{kk}$, $k=1,2$, and red for $\Sg^*_{12}$. The
  trivial lower bound $\varepsilon$ for $\Sg^*_{kk}$ 
  is in cyan color and solid line-style. When the lower bound for
  $\Sg^*_{12}$ becomes negative it tends to $-\infty$ in this
  log-scale. For $\alpha=0$, a polynomial
  fit to $\Sg_{kk}^*$ is also displayed in magenta color and dashed
  line-style. As the value of the parameter $\alpha$
  increases and the flow transitions from cell to shear structure,
  there is a substantial enhancement in the effective diffusivity for
  small values of $\varepsilon$. 
        }
\label{fig:Figure5_Effective_Diffusivity_CatsEye}
\end{figure}
%




The behavior of $\Sg_{jk}^*$ that we deduced from the behavior of the
spectral measure $\mu_{jk}$ is consistent with our computations of the
components $\Sg^*_{jk}$, $j,k=1,2$, of the matrix $\Sg^*$, which are
displayed in \figref{fig:Figure5_Effective_Diffusivity_CatsEye}.
%along with their upper and lower bounds given in the same color with
%dash-dot and dashed line-style, respectively.
Since the support of the
spectral measure $\mu_{jk}$ is contained in the interval $[-1,1]$, the
components $\Sg^*_{jk}$ of the effective diffusivity approach their
bare diffusive value $\varepsilon\,\delta_{jk}$ as $\varepsilon$
surpasses $\varepsilon=1$, as discussed in the beginning of
\secref{sec:Numerical_RESULTS}. 




For cat's eye cell-flow, when $\alpha=0$ the log-log plot of
$\Sg_{kk}^*$ displays a linear trend for
$10^{-3}\lesssim\varepsilon\lesssim10^{-1}$, capturing the
known~\cite{Fannjiang:1994:SIAM_JAM:333,Fannjiang:1997:1033} power law 
behavior $\Sg_{kk}^*\sim\varepsilon^{a}$ as $\varepsilon\to0$. The
polynomial fit $P(\varepsilon)$ to this line is given by
$P(\varepsilon)=0.54\varepsilon+0.08$. This calculation of the
critical exponent $a=0.54$ has a $8\%$ error relative to the
value of the theoretical result
$a=1/2$~\cite{Fannjiang:1994:SIAM_JAM:333,Fannjiang:1997:1033}.
% Plotting the panel of
% \figref{fig:Figure3_Transition_Away_From_CatsEye_Cell_Flow} for
% $\alpha=0$ with log scale $x$-axis (not shown) reveals that the
% measure weights $m_{kk}(n)$ have linearly increasing maximum from
% $\sim10^{-8}$ to $\sim10^{-5}$ for 
% $10^{-4}\lesssim|\lambda_n|\lesssim10^{-1}$
% which gives rise to this linear behavior of $\Sg_{kk}^*$.
The presence  
of spectral weights with magnitudes
$10^{-7}\lesssim m_{kk}(n)\lesssim10^{-4}$
and associated eigenvalues 
$10^{-4}\lesssim|\lambda_n|\lesssim10^{-1}$
give rise to a moderate enhancement in $\Sg_{kk}^*$. This enhancement
increases from a fraction of an order of magnitude to one and a half
orders of magnitude above its bare diffusive value $\varepsilon$, as
$\varepsilon$ decreases from $10^{-1}$ to $10^{-3}$, as shown in
\figref{fig:Figure3_Transition_Away_From_CatsEye_Cell_Flow} for
$\alpha=0$.    


We deduced from 
\figref{fig:Figure3_Transition_Away_From_CatsEye_Cell_Flow} that for
$\alpha\in(0,10^{-2})$ the behavior of the spectral measure gives rise
to only a moderate enhancement  
$\mathcal{E}_{jk}(\varepsilon)=\varepsilon\sum_n\,m_{jk}(n)/(\varepsilon^2+\lambda_n^2)$
of the effective diffusivity $\Sg_{jk}^*(\varepsilon)=\varepsilon\delta_{jk} +
\mathcal{E}_{jk}(\varepsilon)$ both for the advection dominated regime
where $\varepsilon\sim10^{-14}$, for example, and for the transitional regime
$10^{-3}\lesssim\varepsilon\lesssim10^{-1}$.
%and instead provides a significant enhancement only for $\varepsilon\ll1$.
This is consistent with the behavior of $\Sg^*_{jk}$ 
shown in \figref{fig:Figure5_Effective_Diffusivity_CatsEye}
for $\alpha=0$ and $\alpha=0.05$. We further deduced
from Figures~\numfigref{fig:Figure6_CatsEye_Cell_Flow_mu12_LogLog} and
\numfigref{fig:Figure4_Transition_Toward_CatsEye_Shear_Flow} that for
$\alpha\in(10^{-2},1)$ the behavior of the spectral measure
gives rise to a significant enhancement for both $\varepsilon\sim10^{-14}$
and
$10^{-3}\lesssim\varepsilon\lesssim10^{-1}$ where there is 
 a marked increase in spectral weight magnitude from
$10^{-7}\lesssim m_{kk}(n)\lesssim10^{-4}$
to
$10^{-7}\lesssim m_{kk}(n)\lesssim10^{-2}$
 for
eigenvalues satisfying
$10^{-4}\lesssim|\lambda_n|\lesssim10^{-1}$.
This is consistent with the panels of
\figref{fig:Figure5_Effective_Diffusivity_CatsEye} corresponding to
$0.1\le\alpha\le1$, with $\Sg^*_{jk}$ enhanced many orders of
magnitude above its bare diffusive value $\delta_{jk}\,\varepsilon$,
and $\Sg^*_{kk}$ as well as $\Sg^*_{12}$ closely following their upper
bounds for $\varepsilon\gtrsim10^{-0.5}$ when $\alpha=1$.  



We conclude this section with a description of various symmetries
arising in our numerical computations and their consequences.
We discussed above that our computations of $\mu_{jk}$, 
$j=1,2$, display the symmetry $\mu_{11}=\mu_{22}=|\mu_{12}|$.
%Since the behavior of the $\mu_{jk}$ govern
%the behavior of the corresponding components of the effective
%diffusivity $\Sg^*_{jk}$, the symmetry $\mu_{11}=\mu_{22}=|\mu_{12}|$ between
%the measures
This gives rise to the symmetry
$\Sg^*_{11}(\varepsilon)=\Sg^*_{22}(\varepsilon)=\varepsilon
+\Ec_{12}(\varepsilon;\mu_{12}^+)+\Ec_{12}(\varepsilon;\mu_{12}^-)$ between
the components of the effective diffusivity, where we have denoted 
$\Ec_{jk}(\varepsilon;\mu_{jk})=\varepsilon\int\d\mu_{jk}(\lambda)/(\varepsilon^2+\lambda^2)$, 
e.g.,
$\Sg^*_{11}(\varepsilon)=\varepsilon+\Ec(\varepsilon;\mu_{11})$. The 
symmetry $\Sg^*_{11}=\Sg^*_{22}$ can be clearly seen in our
computations of $\Sg^*_{jk}$, $j,k=1,2$, displayed in
\figref{fig:Figure5_Effective_Diffusivity_CatsEye}; the two curves lay 
right on top of one another, as do their upper and lower bounds as
$\mu^0_{11}=\mu^0_{22}$. We have also numerically explored the 
empirical relationship
$\Sg^*_{11}\approx\varepsilon+[\Sg^*_{12}]^++[\Sg^*_{12}]^-$ 
by plotting $\Sg^*_{11}$
%, $\Sg^*_{22}$,
and $\varepsilon+[\Sg^*_{12}]^++[\Sg^*_{12}]^-$ on
one graph. For most values of $\alpha$ and $\varepsilon$ considered, the three curves
lay virtually on top of each other (not shown), and when there is
a deviation of $\varepsilon+[\Sg^*_{12}]^++[\Sg^*_{12}]^-$ from $\Sg^*_{11}$,
it is slight. This property also leads to the inequalities
$\Sg^*_{11}\geq\varepsilon+[\Sg^*_{12}]^+$ and
$\Sg^*_{11}\geq\varepsilon+[\Sg^*_{12}]^-$, with
$\Sg^*_{11}=\Sg^*_{22}$, which is consistent with the behavior of
$\Sg^*_{jk}$  shown in
\figref{fig:Figure5_Effective_Diffusivity_CatsEye}. 







% When the figure
% panels associated with $\mu_{12}$ in
% \figref{fig:Figure3_Transition_Away_From_CatsEye_Cell_Flow} are
% plotted in log-log scale as shown in
% \figref{fig:Figure6_CatsEye_Cell_Flow_mu12_LogLog}, the following is
% revealed. As $\alpha$ increases from zero, the spectra of the accumulation point
% with positive measure mass migrates away from $\lambda=0$, so that the accumulation
% point eventually consists of spectra with only negative measure
% mass. Consequently, as $\varepsilon$ decreases below $10^{-3}$, this influence
% of $\mu_{12}$ on $\Sg^*_{12}$ becomes more dominant and $\Sg^*_{12}$ changes
% sign, becoming negative, as shown in
% \figref{fig:Figure5_Effective_Diffusivity_CatsEye}. As the value 
% of $\varepsilon$ approaches the location of this accumulation point, the numerical
% approximation breaks down due to finite size effects $L<\infty$, as
% discussed at the beginning of \secref{sec:Numerical_RESULTS}.



%
\begin{figure}[t]
  \centerline{\includegraphics[scale=1]{Figure7_Random_CatsEye_Flow}} 
\caption{%
  Spectral functions and effective diffusivities for randomly
  perturbed cat's eye flow. The random parameter $\alpha$ is uniformly
  distributed on the interval $[0,p]$. The spectral functions
  $\mu_{jk}(\lambda)$ are displayed with corresponding effective diffusivities
  $\Sg^*_{jk}$ directly below for various values of $p$, increasing
  from left to right. As $p$ increases and the streamlines of the flow
  become more elongated in the $x$-$y$ direction, on average, the
  region about the spectral origin $\lambda=0$ with substantial measure mass
  increases in breadth and magnitude. This gives rise to a substantial
  enhancement in the components $\Sg^*_{jk}$ of the effective
  diffusivity for larger values of the non-dimensionalized molecular
  diffusivity $\varepsilon$. The 
  color scheme of the panels for $\Sg^*_{jk}$ is the same as that in
  \figref{fig:Figure5_Effective_Diffusivity_CatsEye}. 
        }
\label{fig:Figure7_Random_CatsEye_Flow}
\end{figure}
%


\subsubsection{Randomly perturbed cat's eye flow}
%
We now discuss our computations of the components $\mu_{jk}$ and
$\Sg^*_{jk}$, $j,k=1,2$, of the spectral measure and effective
diffusivity matrix, respectively, for randomly perturbed cat's eye
flow with $\alpha$ uniformly distributed on the interval $[0,p]$. For each
statistical trial of a sample space $\Omega_0$, with $|\Omega_0|
\sim10^{\,3}$
%statistical trials and a system resolution
and $L=100$, we computed \emph{every}
eigenvalue $\lambda_n^1$ and eigenvector $\vecr_n^1$,
$n=1,\ldots,K_1$, of the matrix 
$-\imath\Um_1^T\Hm\Um_1$ to form the spectral measure $\mu_{jk}$
in equation~\eqref{eq:Measure_weights_mjk}. In order to visually
determine the behavior of the function
$\mu_{jk}(\lambda)=\langle\Qm(\lambda)\hat{\vece}_j,\hat{\vece}_k\rangle$ underlying the
spectral measure $\mu_{jk}$, we plot a histogram representation of
$\mu_{jk}(\lambda)$ called the \emph{spectral function}, which we will also
denote by $\mu_{jk}(\lambda)$. We now describe how we computed this graphical
representation of the measure $\mu_{jk}$. First, 
the spectral interval $I\supseteq\Sigma$ was divided into $V$ sub-intervals $I_v$,
$v=1,\ldots,V$, of equal length. In our
computations of the spectral functions we typically used $V\sim10^{\,2}$.
%$I/V$.
Second, for fixed $v$, we identified
all of the eigenvalues that satisfy $\lambda_n^1(\omega)\in I_v$, for
$n=1,\ldots,K_1$ and $\omega\in\Omega_0$. The assigned value of $\mu_{jk}(\lambda)$ at the
midpoint $\lambda$ of the interval $I_v$, is the sum of the spectral weights 
$m_{jk}(\omega)$ associated with all such $\lambda_n^1(\omega)\in
I_v$, normalized by $|\Omega_0|$. In this way, the area underneath the
curve is the measure mass $\mu_{jk}^0$. 




Consistent with the symmetries of the randomly perturbed flow, our
computations of 
the spectral function satisfy $\mu_{11}(\lambda)=\mu_{22}(\lambda)$, hence the
ensemble averaged components $\Sg^*_{jk}$ of the effective diffusivity
also satisfy $\Sg^*_{11}=\Sg^*_{22}$, as shown in
\figref{fig:Figure7_Random_CatsEye_Flow}. Similar to our computations
for non-random $\alpha$, when $p=0.1$ the measure mass of $\mu_{jk}$,
$j,k=1,2$, near the spectral origin $\lambda=0$ is quite small and, on average,
the region with significant magnitude increases in breadth as $p$
increases, with the formation of a high concentration of measure mass at
the spectral origin $\lambda=0$ as $p\to1$ --- due to the
incorporation of statistical realizations with near shear-flow
characteristics associated with $\alpha\approx1$. This average increase in
the breadth of 
the region with  
significant mass and the formation of the high concentration of measure mass at
$\lambda=0$ gives rise to a substantial enhancement of the components
$\Sg^*_{jk}$ of the effective diffusivity above the bare molecular
diffusivity values $\varepsilon\,\delta_{jk}$. The sign changes in
$\mu_{12}(\lambda)$ give rise to sign
changes in $\Sg^*_{12}=[\Sg^*_{12}]^+-[\Sg^*_{12}]^-$. In the
log-log plot of $\Sg^*_{12}$ a negative singularity forms in
$\Sg^*_{12}$ at the location of sign changes.
% Our results for cat's eye
% flow with random $\alpha$ also demonstrate the influence of resolution $L$
% in the numerical computations. In particular, with a decrease in
% resolution $L$ from $L=200$ in Figures
% \numfigref{fig:Figure3_Transition_Away_From_CatsEye_Cell_Flow} and
% \numfigref{fig:Figure4_Transition_Toward_CatsEye_Shear_Flow} to
% $L=100$ in \figref{fig:Figure7_Random_CatsEye_Flow}, we see that the 
% accuracy of the numerical computations break down for $\varepsilon\sim10^{-3}$ with
% $L=100$ instead of $\varepsilon\sim10^{-4}$ with $L=200$, indicated by a $1/\varepsilon$
% divergence. For the continuum setting, the accumulation point of the spectrum
% at $\lambda=0$ can be discrete with finite or infinite multiplicity, and can
% even be continuous~\cite{Stone:64}. 




\section{Conclusions} \label{sec:Conclusions}
%
% In \secref{sec:Homogenization}, we reviewed the homogenization of the
% advection diffusion equation~\cite{McLaughlin:SIAM_JAM:780}, leading
% to a diffusion equation 
% involving an effective diffusivity matrix $\Dg^*$. The components
% $\Dg_{jk}^*$, $j,k=1,\ldots,d$, of the matrix $\Dg^*$ are given in terms of a
% bilinear functional involving the $j$th component $u_j$ of the fluid
% velocity field $\vecu$ and the $k$th solution $\chi_k$ of a cell
% problem.
%In \secref{sec:Integral_Reps_Sobolev},
We adapted and extended two methods previously introduced
in~\cite{Avellaneda:PRL-753,Avellaneda:CMP-339}
and~\cite{Bhattacharya:AAP:1999:951,Bhattacharya:1989:ASD,Pavliotis:PHD_Thesis} to provide
new
% provide
% a resolvent formula for $\chi_k$, involving a
% self-adjoint operator $M$ that acts on a Sobolev space of
% scalar-fields. The spectral theorem then yielded the
Stieltjes integral representations for the symmetric $\Sg^*$ and
antisymmetric $\Ag^*$ parts of effective diffusivity matrix $\Dg^*$
in~\eqref{eq:Integral_Rep_kappa*}
% shown in 
% equation~\eqref{eq:Integral_Rep_kappa*} of
% \thmref{thm:Int_Rep_Sobolev}
--- for all components of these homogenized matrices. Each integral
representation involves the non-dimensionalized molecular diffusivity
$\varepsilon$ and a 
spectral measure of a self-adjoint operator acting on an
appropriate Hilbert space.
% In \thmref{thm:Bounds} of
% \secref{sec:Integral_Reps_Sobolev},
We utilized these integral representations to derive rigorous
bounds for the off-diagonal components of the matrices $\Sg^*$ and
$\Ag^*$. We also proved the spectral measures of both methods are
identical, establishing that the two approaches yield equivalent
spectral representations for $\Dg^*$.


%In \secref{sec:Matrix_Sobolev},
We developed discrete formulations of these two mathematical frameworks
%in~\secref{sec:Integral_Reps_Sobolev}, 
involving matrix representations of the self-adjoint operators and
developed a standard and \emph{non-standard} spectral theorem in terms
of a standard and \emph{generalized} eigenvalue problem, respectively.
% which
% incorporates important properties 
% of a discrete version of the Sobolev-type inner-product introduced
% in~\secref{sec:Integral_Reps_Sobolev}. In
% \thmref{thm:Int_Rep_Sobolev_Matrix} of \secref{sec:Matrix_Sobolev}, we
This matrix analysis provided the Stieltjes integral representations
for $\Sg^*$ and $\Ag^*$ in~\eqref{eq:Integral_Rep_kappa*}, involving
discrete spectral measures given explicitly in terms of the
%generalized
eigenvalues and eigenvectors of the matrices.
We developed these discrete frameworks for both of the settings where the 
matrix gradient $\nabla$ has Dirichlet boundary conditions, for example, and is
therefore \emph{full-rank}, and where $\nabla$ has periodic boundary
conditions and is therefore \emph{rank-deficient}. In our studies of
advection enhanced diffusion by a \emph{periodic} fluid velocity field
$\vecu$ here, it is necessary to use a matrix gradient $\nabla$ with periodic
boundary conditions.
% , which is \emph{rank-deficient}. In \thmref{thm:Spectral_Equivalence_Rank_Def}
% of \appref{app:Eigenvalue_method}, we generalize the discrete
% mathematical framework developed in \secref{sec:Matrix_Sobolev} to the
% setting that the matrix gradient has periodic boundary conditions and
% is rank-deficient. 
We also proved, in both the full-rank and rank-deficient settings,
that the spectral measures of both methods are identical, establishing
that the two approaches yield equivalent spectral representations for
the effective diffusivity matrix $\Dg^*$. More specifically, this
matrix analysis demonstrates both approaches can be formulated in
terms of a common \emph{standard} eigenvalue problem involving a
matrix with the \emph{smaller size} encountered in the generalized
eigenvalue problem, thus combining the beneficial numerical attributes
of both approaches. 




%In \secref{sec:Num_Results},
We employed these discrete formulations
%developed in \thmref{thm:Spectral_Equivalence_Rank_Def} of
%\appref{app:Eigenvalue_method}
to compute the components 
$\Sg_{jk}^*$, $j,k=1,\ldots,d$, of the matrix $\Sg^*$ for
some model 2D ($\,d=2\,$) periodic flows and randomly perturbed
periodic flows, by directly computing the associated discrete spectral
measure $\Real\mu_{jk}$. As a baseline result, we computed $\Sg_{jk}^*$ and
$\Real\mu_{jk}$ for $BC$-shear-flow, for which the spectral measure is
known~\cite{Avellaneda:CMP-339}. Our numerical results are in good agreement
with the theoretical result. We also computed $\Sg_{jk}^*$ and
$\Real\mu_{jk}$ for the ``cat's eye'' flow, for both the non-random and
randomly perturbed settings, as a function of a free parameter. As the
parameter varies, the flow transitions from cell-flow to shear-flow in
the diagonal direction $y=x$. For cat's eye cell-flow, our
computations capture the
known~\cite{Fannjiang:1994:SIAM_JAM:333,Fannjiang:1997:1033} power law
behavior $\Sg_{kk}^*\sim\varepsilon^{1/2}$ for $\varepsilon\ll1$. 
The spectral measure $\Real\mu_{jk}$ and
$\Sg_{jk}^*$ have transitional behavior as the parameter
varies. This reveals how the details of $\Real\mu_{kk}$ near the
spectral origin govern the enhancement of $\Sg_{kk}^*$ above
its bare diffusive value $\varepsilon$ in the advection dominated
regime where $\varepsilon\ll1$, and similarly for $\Sg_{12}^*$. Consistent
with the symmetries of the flow, our computations indicate that
$\Real\mu_{11}=\Real\mu_{22}$. Our computations of $\Real\mu_{12}$ for
cat's eye flow also suggest a deeper symmetry, namely  
$|\Real\mu_{12}|=\Real\mu_{11}=\Real\mu_{22}$, where $|\Real\mu_{12}|$
is the total   
variation of the signed measure $\Real\mu_{12}$. Our computations of
$\Sg_{jk}^*$ are consistent with these symmetries and rigorous
bounds derived in \thmref{thm:Bounds}. In order to streamline the
presentation of the main theoretical and numerical results in the body
of the manuscript, we have placed more detailed developmental material
in an appendix.



 
In almost 30 years since the initial 
formulation~\cite{Avellaneda:CMP-339,Avellaneda:PRL-753} of Stieltjes
integral representations for the effective diffusivity matrix $\Dg^*$, 
analytical calculations of $\Dg^*$ have been obtained for only a few
simple flows, such as shear flow. Our results
%of \appref{app:Eigenvalue_method}
help further advance the applicability of this
%powerful
spectral measure
approach 
by providing a mathematical foundation
for computation of spectral representations of $\Dg^*$. For randomly
perturbed periodic flows, the spectral method differs from more
traditional methods in that it enables statistical investigation of
the random eigenvalues and eigenvectors, thus connecting homogenization
of advection diffusion processes to random matrix 
theory~\cite{Murphy:PRL:118:036401}. The results in this manuscript
lay the groundwork for such investigations.      


 
% Here is one more suggested take on the conclusion section:
% ********************************************************************************************

% We adapted and extended two methods previously developed to provide
% a new Stieltjes integral representations for the effective diffusivity matrix in the advection diffusion
% equation with a time-independent incompressible velocity field.
% Both the symmetric and antisymmetric representations involve the molecular diffusivity and
% a spectral measure of a self-adjoint operator acting on an appropriate Hilbert space.
% We utilized these integral representations to derive new, rigorous
% bounds for the off-diagonal components of the matrices, and we also proved the
% spectral measures of both methods are identical, yielding equivalent
% spectral representations for the effective diffusivity.
% We developed discrete formulations of these two mathematical frameworks
% and a standard and \emph{non-standard} spectral theorem in terms
% of the eigenvalue problem, respectively.
% We employed these discrete formulations
% to compute the symmetric components of the effective diffusivity matrix for
% several two-dimensional periodic flows with parameters randomly perturbed.
% Our numerical results are in good agreement with the theoretical result. As the
% parameters vary, the flow transitions from cell-flow to shear-flow in
% the diagonal direction, and our calculations reveal how the details of the spectral measure near the
% spectral origin govern the enhancement of effective diffusivity in the advection dominated
% regime where molecular diffusivity is small.

% In almost 30 years since the initial formulation of Stieltjes
% integral representations for the effective diffusivity matrix,
% analytical calculations have been obtained for only a few
% simple flows, such as the shear flow. Our results
% help overcome this limitation by
% providing a mathematical foundation for computation of the spectral
% representations of the effective diffusivity. Moreover, the spectral method is set apart
% from more traditional methods in that it
% allows us to characterize the statistical behavior of the spectral
% measure that governs the statistical properties of the effective diffusivity. Therefore, deeper
% insights into the properties can be gleaned from studies of
% the statistical properties of the eigenvalues and eigenvectors
% themselves.

% *************************************************************************************





\section{Acknowledgments}
We gratefully acknowledge support from the Applied and Computational Analysis Program
and the Arctic and Global Prediction Program  
at the US Office of Naval Research through grants 
N00014-12-10861,
N00014-13-10291, 
%N00014-15-1-2455, 
and N00014-18-1-2552. 
We are also grateful for support from the Division
of Mathematical Sciences at the U.S. National
Science Foundation (NSF) through Grants
DMS-0940249, 
DMS-1413454,
DMS-1715680, 
DMS-1211179, and
DMS-1522383.
Finally, we would like to thank the NSF Math
Climate Research Network (MCRN) for their support of this work.

\appendix

\section{Appendix overview}\label{app:Appendix_overview}
%
In order to streamline the presentation of the main theoretical and
numerical results in the body of the manuscript, we have placed more
detailed developmental material in a series of appendices here. We
now give an overview of the topics covered in this appendix. In
\appref{app:Notation}, we comment on the notation used throughout 
this manuscript. In \appref{app:Bounded_A}, we derive
important properties of the linear operator
$A=\Delta^{-1}[\vecu\bcdot\bnabla]$ defined in
equation~\eqref{eq:Eff_Diffusivity_Sobolev}.




% In \appref{app:Curl_Free} 
% we adapt and extend an alternative
% method~\cite{Avellaneda:PRL-753,Avellaneda:CMP-339} to the method
% discussed in \secref{sec:Integral_Reps_Sobolev}, which leads to the
% integral representations for the symmetric $\Sg^*$ and antisymmetric
% $\Ag^*$ parts of the effective diffusivity matrix $\Dg^*$ displayed in
% equation \eqref{eq:Integral_Rep_kappa*}. In
% \appref{app:Matrix_Formulation_Curl} we provide a discrete formulation of the
% effective parameter problem discussed in \appref{app:Curl_Free} that
% is analogous to the discrete framework presented in
% \secref{sec:Matrix_Sobolev}, yielding a discrete integral
% representation of $\Dg^*$. In \appref{app:Discrete_Equivalence}, we
% demonstrate that the discrete frameworks formulated in
% \secref{sec:Matrix_Sobolev} and \appref{app:Matrix_Formulation_Curl} yield
% equivalent spectral representations of $\Dg^*$ when the matrix gradient
% $\nabla$ is of full-rank, so that the matrix Laplacian
% $-\nabla^T\nabla$ is
% invertible. In \appref{app:Eigenvalue_method}, we generalize the
% result of \appref{app:Discrete_Equivalence} to the setting where the
% matrix $\nabla$ is rank-deficient so that the matrix $-\nabla^T\nabla$ is
% non-invertible, establishing the equivalence of these two discrete
% approaches in the rank-deficient setting. The continuum version of
% this result, holding for both spatially periodic and space-time
% periodic velocity fields $\vecu$, is established
% in~\cite{Murphy:ADSTPF-2017}. 



In \thmref{thm:Int_Rep_Sobolev} of \secref{sec:Integral_Reps_Sobolev}
we adapted and extended a
method~\cite{Bhattacharya:AAP:1999:951,Bhattacharya:1989:ASD,Pavliotis:PHD_Thesis} involving
a self-adjoint operator $M$ acting on a Sobolev space of scalar-valued
functions which provides the Stieltjes integral representations for
$\Sg^*$ and $\Ag^*$ in equation~\eqref{eq:Integral_Rep_kappa*}. In
\appref{app:Curl_Free} below, we adapt and extend a different
method~\cite{Avellaneda:PRL-753,Avellaneda:CMP-339} involving a
self-adjoint operator $\Mb$ acting on the Hilbert space of curl-free
vector-valued functions which also leads to the Stieltjes integral
representations for $\Sg^*$ and $\Ag^*$ in
equation~\eqref{eq:Integral_Rep_kappa*}. These results are summarized
in a corollary (\corref{cor:Int_Rep}) of
\thmref{thm:Int_Rep_Sobolev}. 





In \thmref{thm:Equivalence} of \appref{app:Curl_Free}, we prove that
the spectral measures arising in  \thmref{thm:Int_Rep_Sobolev} and
\corref{cor:Int_Rep} are identical, establishing that the two
approaches yield equivalent spectral representations of $\Dg^*$. This
is accomplished by proving that the masses and all the moments of the
two spectral measures are equal and citing the Hausdorff moment
problem for measures with bounded
support~\cite{Stone:64,Akhiezer:Book:1965}. In a corollary
(\corref{cor:Equivalence}) of \thmref{thm:Equivalence} we utilize a
one-to-one isometric 
correspondence~\cite{Murphy:ADSTPF-2017} between the Sobolev space
arising in \thmref{thm:Int_Rep_Sobolev} and the Hilbert  
space arising in \corref{cor:Int_Rep}, to extend the results of
\thmref{thm:Equivalence} to every spectral measure associated with the
two self-adjoint operators $M$ and $\Mb$. This corollary also proves
that the masses 
and all the moments of the two spectral measures are equal in the
generalized setting of a space-time periodic flow, possibly associated
with chaotic dynamics and unbounded spectrum~\cite{Murphy:ADSTPF-2017}. 
In the setting of unbounded spectrum the Hamburger or Stieltjes moment
problems are instead relevant, and more conditions must be met beyond
the equality of the masses and moments to ensure the measures are
identical, such as Carleman's
criterion~\cite{Akhiezer:Book:1965}. In~\cite{Murphy:ADSTPF-2017} an
alternate method was used to determine that the spectral measures
arising in the two different approaches are indeed identical in the
time-dependent setting.    


In \appref{app:Matrix_Formulation_Curl}, we develop a discrete
formulation of the mathematical framework given in \appref{app:Curl_Free},
involving a Hermitian matrix representation for the self-adjoint
operator $\Mb$. We also briefly review the standard spectral theorem for
Hermitian matrices. This provides the Stieltjes integral
representations for $\Sg^*$ and $\Ag^*$
in~\eqref{eq:Integral_Rep_kappa*} involving a discrete spectral
measure, given explicitly in terms of the eigenvalues and
eigenvectors of the matrix. This discrete framework holds for the
setting where the matrix gradient has Dirichlet boundary conditions,
for example, and is therefore \emph{full-rank}. These results are
analogues of those in \thmref{thm:Int_Rep_Sobolev_Matrix} and are
summarized in \corref{cor:Int_Rep_Matrix}. In
\thmref{thm:Projection_Method}, we develop a \emph{projection
method} that is used to generalize the results in
\thmref{thm:Int_Rep_Sobolev_Matrix} and \corref{cor:Int_Rep_Matrix} to
the setting where the matrix gradient has periodic boundary
conditions, for example,
and is therefore \emph{rank-deficient}. The results of
\thmref{thm:Projection_Method} are also used to prove the discrete
spectral representations arising in
\thmref{thm:Int_Rep_Sobolev_Matrix} and \corref{cor:Int_Rep_Matrix}
are equivalent in this rank-deficient setting.




Specifically, in \lemref{lem:Spectral_Equivalence} of
\appref{app:Discrete_Equivalence}, we use properties of the singular
value decomposition of the matrix gradient $\nabla$ to reveal
symmetries between the two discrete approaches formulated in
Section~\ref{sec:Matrix_Sobolev} and \appref{app:Matrix_Formulation_Curl},
establishing  that the two approaches yield equivalent spectral
representations of the effective diffusivity  matrix $\Dg^*$ when
$\nabla$ is full-rank. In particular, we establish in the 
proof of \lemref{lem:Spectral_Equivalence} that the eigenvalues and
generalized eigenvalues underlying the spectral measures for each
method are in fact eigenvalues of a Hermitian matrix arising in both 
methods. Moreover, the eigenvectors $\vecw_n$ and generalized
eigenvectors $\vecz_n$ of the two methods are related by
$\vecw_n=\nabla\vecz_n$, which leads to the equivalence of the
discrete spectral measures of the two approaches.



In \thmref{thm:Spectral_Equivalence_Rank_Def} of
\appref{app:Eigenvalue_method}, we generalize
\lemref{lem:Spectral_Equivalence} to the rank-deficient setting. This
generalizes both the numerical algorithms developed in 
\secref{sec:Matrix_Sobolev} 
and \appref{app:Matrix_Formulation_Curl} to the setting of periodic boundary  
conditions and combines beneficial numerical attributes of each
algorithm. In \secref{sec:Num_Results}, this common method is used to
compute $\Sg^*$ for model flows and relate spectral
characteristics to flow geometry and transport 
properties.




\section{Notation}\label{app:Notation}
%
We now briefly discuss the notation used throughout the
manuscript. Operators are denoted by capital letters while functions
comprising the domains of these operators are denoted by lowercase
letters. (Capital letters are also used to denote the size of
matrices.) Furthermore, vector-valued 
functions are denoted by lowercase bold font, e.g., $\vecxi\,,$ while
scalar-valued functions are denoted by lowercase non-bold font, e.g., 
$\xi\,.$ Matrices denoted by capital Latin letters are in math-serif
font, e.g., $\Hm$, $\Bm$, $\Cm$, $\Zm$, $\Qm$, $\Am$, $\Um$, $\Vm$,
$\Wm$, etc. Integro-differential operators mapping scalar-valued 
functions to scalar-valued functions are in standard math-italic font,
e.g., $A$ and $M$. Integro-differential  operators mapping
vector-valued functions to vector-valued functions are in
math-boldface font, e.g., $\Ab$ and $\Mb$. We use a similar notation
for differential operators, e.g., $\bnabla\xi$ and $-\Delta\xi$ and
their discrete, matrix counterparts $\nabla\vecxi$ and
$\nabla^T\nabla\vecxi$, respectively. All homogenized matrices are in
Gothic font and have a
super-script asterisk, e.g., $\Dg^*$, $\Sg^*$, and
$\Ag^*$. 


\section{Properties of the linear operator $A$}\label{app:Bounded_A}
%
In this section we derive various properties of the linear operator
$A=\Delta^{-1}[\vecu\bcdot\bnabla]$ defined
in equation~\eqref{eq:Eff_Diffusivity_Sobolev}. In particular, we
demonstrate that $A$ is antisymmetric on the Hilbert space $\Hs^{1,2}$
defined in~\eqref{eq:Sobolev}. Moreover, we show that $A$ is
bounded on $\Hs^{1,2}$ and we provide an upper bound for $\|A\|_{1,2}$ when
$\vecu$ is uniformly bounded on the period cell $\Vc$.  


We first show that the incompressibility condition
$\bnabla\bcdot\vecu=0$ implies that the operator $A$ is
antisymmetric on $\Hs^{1,2}$ \cite{Bhattacharya:AAP:1999:951,Bhattacharya:1989:ASD}, i.e.,
$\langle Af,h\rangle_{1,2}=-\langle f,A h\rangle_{1,2}\,.$ On the 
Hilbert space $\Hs$ defined in equation~\eqref{eq:Hilbert_Sobolev}, the linear
operator $\Delta^{-1}$ satisfies $\langle\Delta\Delta^{-1}f,h\rangle=\langle f,h\rangle$ in a distributional
sense, for all
$f,h\in\Hs$~\cite{McOwen:2003:PDE,Folland:95:PDEs,Murphy:ADSTPF-2017}.
%Moreover, the operator $\Delta^{-1}$ is bounded and
%symmetric~\cite{Stakgold:BVP:2000} on $\Hs$, hence
%self-adjoint~\cite{Reed-1980}.
Consequently, integration by parts and $\bnabla\bcdot\vecu=0$
yields~\cite{Bhattacharya:AAP:1999:951,Bhattacharya:1989:ASD,Murphy:ADSTPF-2017,Pavliotis:PHD_Thesis} 
%
\begin{align}\label{eq:Anti-sym_Sobolev}
 \langle Af,h\rangle_{1,2}%= \langle(\Delta^{-1})(\vecu\bcdot\bnabla)f,h\rangle_{1,2}
       &=\langle[\bnabla(\Delta^{-1})(\vecu\bcdot\bnabla)f]\bcdot\bnabla h\rangle
       \\                              
       &=-\langle[(\vecu\bcdot\bnabla)f]\,,h\rangle
       \notag\\
       &=-\langle[\bnabla\bcdot(\vecu f)]\,,h\rangle
       \notag\\     
       &=\langle f\,,[(\vecu\bcdot\bnabla)h]\rangle
       \notag\\
       &=\langle f\,,[\Delta(\Delta^{-1})(\vecu\bcdot\bnabla)h]\rangle
       \notag\\
       &=-\langle\bnabla f\bcdot[\bnabla(\Delta^{-1})(\vecu\bcdot\bnabla)h]\rangle
      %  \notag\\                              
%        &=-\langle f,(\Delta^{-1})(\vecu\bcdot\bnabla)h\rangle_{1,2}
       \notag\\                              
       &=-\langle f,Ah\rangle_{1,2}\,,
       \notag
\end{align}
%
for all $f,h\in\Hs^{1,2}$ and real-valued incompressible $\vecu$
(see~\cite{Murphy:ADSTPF-2017} for more details).




Now, we derive the bound for $\|A\|_{1,2}$ given in
equation~\eqref{eq:Bounded_A}. From the Cauchy-Schwartz inequality
$|\langle f,h\rangle|\leq\|f\|\,\|h\|$ we have 
%
\begin{align}\label{eq:First_Bound_A}
  \|Af\|_{1,2}^2&=|\langle\bnabla[\Delta^{-1}(\vecu\bcdot\bnabla f)]\bcdot
         \bnabla[\Delta^{-1}(\vecu\bcdot\bnabla f)]\rangle|
         \\  
        &=|-\langle[\Delta^{-1}(\vecu\bcdot\bnabla f)]\,,(\vecu\bcdot\bnabla f)\rangle|
        \notag \\
        &\leq\|\Delta^{-1}(\vecu\bcdot\bnabla f)\|\, \|\vecu\bcdot\bnabla f\|
       \notag \\
       &\leq\|\Delta^{-1}\|\, \|\vecu\bcdot\bnabla f\|^2.   
       \notag
\end{align}
%
We now provide an upper bound for  $\|\vecu\bcdot\bnabla f\|$ when the
components $u_k$, $k=1,\ldots,d$, of the fluid velocity field $\vecu$ are
uniformly bounded on the period cell $\Vc$. By the Cauchy-Schwartz
inequality, $|\vecxi\bcdot\veczeta|\leq|\vecxi|\,|\veczeta|$, we have
%
\begin{align}\label{eq:Second_Bound_A}
  \|\vecu\bcdot\bnabla f\|^2
  =\langle|\vecu\bcdot\bnabla f|^2\rangle
  \leq\langle|\vecu|^2\;|\bnabla f|^2\rangle
  \leq\sup_{\vecx\in\Vc}|\vecu(\vecx)|^2\;\|f\|_{1,2}^2\,.      
\end{align}
%
The result in equation~\eqref{eq:Bounded_A} is now clear. 





\section{Curl-free fields and the effective diffusivity
  matrix} \label{app:Curl_Free}    
%
In this section, we adapt and extend an alternative
method~\cite{Avellaneda:PRL-753,Avellaneda:CMP-339} to the method
discussed in \secref{sec:Integral_Reps_Sobolev} which leads to the 
integral representations for the symmetric $\Sg^*$ and antisymmetric
$\Ag^*$ parts of the effective diffusivity matrix $\Dg^*$ in
equation~\eqref{eq:Integral_Rep_kappa*}. More 
specifically, in \appref{app:Functinals_curl-free} we 
provide functional formulas for $\Sg^*$ and $\Ag^*$ that are analogous 
to the formulas in~\eqref{eq:Eff_Diffusivity_Sobolev}. 
We review a Hilbert space formulation of this effective parameter 
problem~\cite{Avellaneda:CMP-339,Avellaneda:PRL-753,Fannjiang:1994:SIAM_JAM:333,Fannjiang:1997:1033,Majda:Kramer:1999:book} 
in \appref{app:Integral_Reps_Curl_free}
which leads to a resolvent formula for $\bnabla\chi_j$ that is analogous 
to the resolvent formula for $\chi_j$ in~\eqref{eq:Resolvent_Rep_Sobolev}, 
involving a self-adjoint
%random
operator. We use this result and the spectral
theorem~\cite{Stone:64,Reed-1980} to provide the Stieltjes integral
representations for $\Sg^*$ and $\Ag^*$ in~\eqref{eq:Integral_Rep_kappa*}, 
involving a
%(possibly different)
spectral measure of the operator. Finally, 
we prove the spectral measure corresponding to the Stieltjes integral representation 
for $\Dg^*$ in equation~\eqref{eq:Integral_Rep_kappa*} of \thmref{thm:Int_Rep_Sobolev} 
is identical to the spectral measure corresponding the Stieltjes integral 
representation for $\Dg^*$ developed in the current section. This establishes that the 
two different approaches yield equivalent spectral representations of
$\Dg^*$.
% We accomplish this by
% proving that the mass and all the moments of each spectral measure are 
% identical, which establishes the equivalence through the fact that the 
% Hausdorff moment problem is determinate for measures with bounded
% support~\cite{Stone:64,Akhiezer:Book:1965}. 


\subsection{Functional formulas for the effective
  diffusivity matrix} \label{app:Functinals_curl-free} 
%
% In this section, we adapt and extend an alternate
% method~\cite{Avellaneda:PRL-753,Avellaneda:CMP-339} to the method 
% presented in
% \secref{sec:Integral_Reps_Sobolev}~\cite{Bhattacharya:AAP:1999:951,Bhattacharya:1989:ASD,Pavliotis:PHD_Thesis},
% which provides integral representations for the effective diffusivity
% matrix $\Dg^*$. 
% %, and leads to a more direct approach for its computation than the
% %projection method discussed in \thmref{thm:Projection_Method}.
% In particular,
%
In this section, we derive functional formulas for the symmetric $\Sg^*$ and antisymmetric $\Ag^*$ parts of the effective diffusivity matrix $\Dg^*$ that 
are analogous to the formulas in equation~\eqref{eq:Eff_Diffusivity_Sobolev}. 
Using equation~\eqref{eq:H_Hessian} and the representation of the
fluid velocity field $\vecu$ in~\eqref{eq:u_DH}, the advection diffusion 
equation in~\eqref{eq:ADE} can be written as a diffusion 
equation~\cite{Fannjiang:1994:SIAM_JAM:333,Fannjiang:1997:1033},   
%
\begin{align}\label{eq:ADE_Divergence}
  \phi_t=\bnabla \bcdot[\Dm\bnabla \phi],
  \quad    
    \phi(0,\vecx)=\phi_0(\vecx),
    \qquad
    \Dm=\varepsilon\Ib+\Hm.
\end{align}
%
Moreover, the cell problem in~\eqref{eq:Random_Cell_Prob} can be
written as a steady-state diffusion
equation~\cite{Fannjiang:1994:SIAM_JAM:333,Fannjiang:1997:1033},        
% 
\begin{align}\label{eq:Cell_Problem}
  \bnabla \bcdot[\Dm(\bnabla \chi_k+\vece_k\,)]=0, \quad
    %=\bnabla \bcdot[(\varepsilon\Ib+\Hm)(\bnabla \chi_k+\vece_k\,)], \quad
  \langle\bnabla\chi_k\rangle=0,
  \quad   k=1,\ldots,d.
\end{align}
%
Here, 
%$\langle\cdot\rangle$ denotes
%ensemble
%volume averaging over the period cell $\Vc$, 
$\vece_k$ is a standard
basis vector, $k=1,\ldots,d$, and
$\Dm(\vecx)=\varepsilon\Ib+\Hm(\vecx)$ can be viewed as a
%random,
local diffusivity matrix with coefficients 
% 
\begin{align}\label{eq:kappa_coeff}
  \Dm_{jk}=\varepsilon\delta_{jk}+\Hmc_{jk},\quad j,k=1,\ldots,d\,,
\end{align}
%
where $\delta_{jk}$ is the Kronecker delta and $\Ib$ is the 
identity operator on $\mathbb{R}^d$.    






Substituting into equation~\eqref{eq:Djk} the expression for $u_j$
in~\eqref{eq:Random_Cell_Prob} and using
equation~\eqref{eq:u_DH}, $\vecu=\bnabla\bcdot\Hm$, shows the 
components $\Sg^*_{jk}$ and $\Ag^*_{jk}$
of $\Sg^*$ and $\Ag^*$ 
can be written in terms of the following functional formulas involving 
the \emph{real-valued} vector field $\bnabla\chi_k\,,$  
%
\begin{align}\label{eq:Eff_Diffusivity}
 \Sg^*_{jk}=\varepsilon(\delta_{jk}+\langle\bnabla \chi_j\bcdot\bnabla \chi_k\rangle), \qquad
 \Ag^*_{jk}=\langle\Hm\,\bnabla \chi_j\bcdot\bnabla \chi_k\rangle\,. \qquad 
\end{align}
%
% We emphasize that, while the 
% effective diffusivity matrix $\Dg^*$ is not symmetric in
% general~\cite{Fannjiang:1994:SIAM_JAM:333,Fannjiang:1997:1033}, only its
% symmetric part plays a role in the homogenized equation in
% \eqref{eq:phi_bar}~\cite{McLaughlin:SIAM_JAM:780}.
%
The functional formulas in~\eqref{eq:Eff_Diffusivity} 
are analogous to the functional formulas in
equation~\eqref{eq:Eff_Diffusivity_Sobolev}.
The symmetry $\Sg^*_{jk}=\Sg^*_{kj}\,$ of the matrix $\Sg^*$
follows from~\eqref{eq:Eff_Diffusivity} and the fact that the vector
field $\bnabla \chi_k$ is real-valued so that
$\langle\bnabla\chi_j\bcdot\bnabla\chi_k\rangle=\langle\bnabla\chi_k\bcdot\bnabla\chi_j\rangle$. Moreover,
the positivity condition
$\langle\bnabla\chi_k\bcdot\bnabla\chi_k\rangle=\langle|\bnabla\chi_k|^2\rangle\geq0$ demonstrates that
the effective transport of the scalar density $\phi$ is always
\emph{enhanced}
by the presence of an incompressible velocity field, i.e.,
$\Dg^*_{kk}=\Sg^*_{kk}\geq\varepsilon$. The equality $\Dg^*_{kk}=\Sg^*_{kk}$
follows from the skew-symmetry of the matrix $\Ag^*$, 
$\Ag^*_{kj}=-\Ag^*_{jk}$, hence $\Ag^*_{kk}=0$. The skew-symmetry of
$\Ag^*$ follows from the skew-symmetry of the \emph{real-valued}
matrix $\Hm$, 
$\Ag^*_{jk}=\langle\Hm\bnabla\chi_j\bcdot\bnabla\chi_k\rangle=-\langle\bnabla\chi_j\bcdot\Hm\bnabla\chi_k\rangle
=-\langle\Hm\bnabla\chi_k\bcdot\bnabla\chi_j\rangle=-\Ag^*_{kj}\,$.
%We discuss the properties of the matrix $\Hm$ and the vector field
%$\bnabla \chi_j$ in more detail in \appref{app:Integral_Reps_Curl_free}. 







\subsection{The analytic continuation method and  integral
representations of $\Dg^*$} \label{app:Integral_Reps_Curl_free}  
%
In this section we begin by noting that the cell problem in
equation~\eqref{eq:Cell_Problem} is
equivalent~\cite{Avellaneda:PRL-753,Avellaneda:CMP-339,Fannjiang:1994:SIAM_JAM:333,Fannjiang:1997:1033}
to the quasi-static limit of Maxwell's
equations~\cite{Golden:CMP-473,Milton:2002:TC,Jackson-1999}, which 
describe the transport properties of an electromagnetic wave in a
composite material, 
%
\begin{align}\label{eq:Maxwells_Equations}
  \bnabla\btimes\vecE_k=0, \quad
  \bnabla\bcdot\vecJ_k=0, \quad
  \vecJ_k=\Dm\vecE_k,\quad
  \langle\vecE_k\rangle=\vece_k,\qquad
  \Dm%=\Sg-(\bDelta^{-1})\Tb.
       %=\varepsilon\Ib+\Hm-(\bDelta^{-1})\Tb.
       =\varepsilon\Ib+\Hm.
\end{align}
%
Here, $\vecE_k=\bnabla \chi_k+\vece_k$ plays the role of the local
electric field,
%satisfying
%which satisfies
%$\langle\vecE_k\rangle=\vece_k$,
$\vecJ_k=\Dm\vecE_k$
plays the role of the local current density, and $\Dm=\varepsilon\Ib+\Hm$ plays
the role of the local conductivity matrix of the medium. Since $\Hm$
is skew-symmetric, the intensity-flux relation $\vecJ_k=\Dm\vecE_k$ is
not the usual Fourier law, but instead resembles that of a Hall
medium~\cite{Isichenko:JNS:1991:375,Fannjiang:1994:SIAM_JAM:333,Fannjiang:1997:1033,Milton:2002:TC}.





% In \appref{app:Integral_Reps_Curl_free}, we employ the representation of 
% the cell problem in~\eqref{eq:Cell_Problem} and
% adapt the analytic continuation method~\cite{Golden:CMP-473} for
% characterizing transport in composites to provide Stieltjes integral
% representations for both the symmetric and antisymmetric parts of the
% effective diffusivity matrix $\Dg^*$, involving a spectral measure of
% a self-adjoint
% %random
% operator.      

In~\cite{Avellaneda:CMP-339,Avellaneda:PRL-753}, the analytic
continuation method for representing transport in composites was
adapted to provide a Stieltjes integral representation for the
symmetric part of the effective diffusivity matrix $\Dg^*$, involving
a spectral measure of a self-adjoint
%random
operator. This method
provides Stieltjes integral representations for the bulk transport
coefficients of composite media~\cite{Golden:CMP-473}, such as the
effective electrical conductivity. This method is based on the spectral theorem
of Hilbert space theory and a resolvent formula for, say, the electric
field, involving a self-adjoint operator~\cite{Golden:CMP-473} or
matrix~\cite{Murphy:2015:CMS:13:4:825} which depends only on the
composite geometry. In this section, we adapt the
method developed in~\cite{Avellaneda:CMP-339,Avellaneda:PRL-753} to provide
Stieltjes integral representations for \emph{both} the symmetric and
antisymmetric parts of the effective diffusivity matrix $\Dg^*$, which
encodes the flow geometry of the fluid velocity field in a spectral
measure of a self-adjoint operator. 



Following the discussion leading to
equation~\eqref{eq:Sobolev}, we consider a fluid velocity field that
is spatially periodic on a region $\Vc\subset\mathbb{R}^d$, and define
the Hilbert space $\Hc$~\cite{Majda:Kramer:1999:book,Fannjiang:1997:1033}     
%
\begin{align}\label{eq:Hilbert_Space}
  \Hc=\{\vecxi\in\otimes_{n=1}^dL^2(\Vc,\nu)\,
              :\, \vecxi(\vecx)~\text{is periodic in}~\Vc
              \text{ and }\langle\vecxi\rangle=0\},
\end{align}
% 
which is analogous to the Hilbert space $\Hs$ defined in
equation~\eqref{eq:Hilbert_Sobolev}, where $\Hc$ is defined over
vector-fields instead of scalar-fields as in $\Hs$. The Hilbert space
$\Hc$ is equipped with a sesquilinear inner-product
$\langle\cdot,\cdot\rangle$ defined by
$\langle\vecxi,\veczeta\rangle=\langle\vecxi\bcdot\veczeta\rangle$,
with $\vecxi\bcdot\veczeta=\vecxi^{\,\dagger}\veczeta$, $\dagger$ is the operation
of complex-conjugate-transpose, and
$\langle\veczeta,\vecxi\rangle=\overline{\langle\vecxi,\veczeta\rangle}$
for $\,\vecxi,\veczeta\in\Hc$. This inner-product induces a norm $\|\cdot\|$
defined by $\|\vecxi\|=\langle\vecxi,\vecxi\rangle^{1/2}$ and
$\vecxi\in\Hc$ implies that $\|\vecxi\|<\infty$.
% Here,
% $\langle\cdot\rangle$ denotes volume average over the period cell 
% $\Vc$ with respect to the Lebesgue measure $\nu$, $\overline{a}$
% is the complex-conjugate of the scalar $a$, and we emphasize that the
% dot product $\vecxi\bcdot\veczeta=\vecxi^{\,\dagger}\veczeta$ includes
% the operation $\dagger$ of complex-conjugate-transpose.  
Now consider the associated Hilbert space $\Hc_\times$ of curl-free
%random
fields~\cite{Golden:CMP-473,Avellaneda:CMP-339,Fannjiang:1994:SIAM_JAM:333,Fannjiang:1997:1033,Milton:2002:TC,Murphy:ADSTPF-2017},    
%
\begin{align}\label{eq:curlfreeHilbert}
  &\Hc_\times=
  \left\{\vecxi\in \Hc \;:\; \bnabla\btimes\vecxi=0
    \text{~ weakly and~}
    \langle\vecxi\rangle=0
  \right\}.
\end{align}
%
The curl-free vector field $\bnabla\chi_k$ in the cell
problem in~\eqref{eq:Cell_Problem} is mean-zero $\langle\bnabla\chi_k\rangle=0\,.$ When
the matrix $\Dm$ in equation~\eqref{eq:ADE_Divergence} is bounded in the
operator norm $\|\cdot\|$ induced by the $\Hc$-inner-product~\cite{Folland:99:RealAnalysis},
$\|\Dm\|<\infty$, then there exists unique
$\bnabla\chi_k\in\Hc_\times$ satisfying
equation~\eqref{eq:Cell_Problem}~\cite{Papanicolaou:RF-835,Golden:CMP-473}. We
assume that   
%
\begin{align}\label{eq:Bounded_H}
  0<\varepsilon<\infty,
  \quad
  \|\Hm\|<\infty,   
\end{align}
%
which together imply $\|\Dm\|<\infty$.




The linear operator $\bGamma=\bnabla(\Delta^{-1})\bnabla\bcdot$ is a
projection onto the Hilbert space $\Hc_\times$ in the sense that
$\bGamma:\Hc\mapsto\Hc_\times$ 
and $\bGamma\vecxi=\vecxi$ (weakly) for all $\vecxi\in\Hc_\times$,
in particular $\bGamma\,\bnabla\chi_k=\bnabla\chi_k$~\cite{Fannjiang:1994:SIAM_JAM:333,Murphy:ADSTPF-2017}. It is  based on
convolution with respect to the Green's function for the Laplacian
$\Delta=\nabla^2$~\cite{Stakgold:BVP:2000,Melnikov:2012:Green}. Applying the
integro-differential operator $\bnabla\Delta^{-1}$ to the 
cell problem in equation~\eqref{eq:Cell_Problem} yields
$\bGamma[(\varepsilon\Ib+\Hm)(\bnabla\chi_k+\vece_k\,)]=0$. Since
$\bGamma\vece_k=0$ and $\bGamma\bnabla\chi_k=\bnabla\chi_k$, this formula is
equivalent to
$(\varepsilon\Ib+\bGamma\Hm\bGamma)\bnabla\chi_k=-\bGamma\Hm\vece_k$,
which yields the following resolvent formula for $\bnabla\chi_k$   
% 
\begin{align}\label{eq:Resolvent_Rep}
  \bnabla\chi_k=(\varepsilon\Ib+\Ab)^{-1}\vecg_k\,,
           %=(\varepsilon\Ib+\imath\Mm)^{-1}\vec{g}_k,
           \qquad
  \Ab=\bGamma\Hm\bGamma, \quad
  %\Mm=-\imath\Am, \quad
  \vecg_k=-\bGamma\Hm\vece_k\,,
\end{align}
%
which is analogous to equation \eqref{eq:Resolvent_Rep_Sobolev}.
Since $\bGamma$ is a projection operator onto $\Hc_\times\subset\Hc$, it is
bounded by unity in operator norm on $\Hc$,
$\|\bGamma\|\leq1$~\cite{PapaRudin:87,Folland:99:RealAnalysis}. Integration by
parts and the symmetry of the operator
$\Delta^{-1}$~\cite{Stakgold:BVP:2000} (or the projective nature of
$\bGamma$ itself) shows that $\bGamma$ is also a
\emph{symmetric} operator, i.e.,
$\langle\bGamma\vecxi\bcdot\veczeta\rangle=\langle\vecxi\bcdot\bGamma\veczeta\rangle$ for all
$\vecxi,\veczeta\in\Hc$~\cite{Murphy:ADSTPF-2017}. These
two properties show that $\bGamma$ with domain $\Hc$ is a \emph{self-adjoint}
operator~\cite{Reed-1980}. Since $\bGamma$ is self-adjoint
and $\bGamma\bnabla\chi_k=\bnabla\chi_k$, we may write $\Ag^*_{jk}$ in
equation~\eqref{eq:Eff_Diffusivity} as
$\Ag^*_{jk}=\langle\Ab\,\bnabla\chi_j\bcdot\bnabla \chi_k\rangle$. Consequently,
substituting the resolvent formula for $\bnabla\chi_k$
in equation~\eqref{eq:Resolvent_Rep} into the functional formulas
in equation~\eqref{eq:Eff_Diffusivity} yields  
%
\begin{align}\label{eq:Eff_Diffusivity_Resolvent}
 \Sg^*_{jk}&=\varepsilon(\delta_{jk}+\langle(\varepsilon\Ib+\Ab)^{-1}\vecg_j\bcdot(\varepsilon\Ib+\Ab)^{-1}\vecg_k\rangle), 
 \\
 \Ag^*_{jk}&=\langle\Ab\,(\varepsilon\Ib+\Ab)^{-1}\vecg_j\bcdot(\varepsilon\Ib+\Ab)^{-1}\vecg_k\rangle\,,
 \notag
\end{align}
%
which is a direct analogue of
equation~\eqref{eq:Eff_Diffusivity_Resolvent_Sobolev}. 




 Since $\bGamma$ is self-adjoint on $\Hc$, the anti-symmetry of
the matrix $\Hm$ implies that $\Ab=\bGamma\Hm\bGamma$ is an
\emph{antisymmetric} operator on $\Hc$, i.e.,   
$\langle\Ab\vecxi\bcdot\veczeta\rangle=-\langle\vecxi\bcdot\Ab\veczeta\rangle$.
We emphasize that the operator $\Ab$ depends
only on the fluid velocity field via equation~\eqref{eq:u_DH}. By
equation~\eqref{eq:Bounded_H}, the 
operator $\Ab$ is bounded on $\Hc$ with $\|\Ab\|\leq\|\Hm\|<\infty$. This, the
skew-symmetry of $\Ab$, and the sesquilinearity of the
$\Hc$-inner-product imply that $\Mb=-\imath\Ab$, where $\imath=\sqrt{-1}$, is a
bounded symmetric  operator, hence
self-adjoint on $\Hc$ with $\|\Mb\|=\|\Ab\|<\infty$. The spectrum
$\Sigma$ of the self-adjoint operator $\Mb$ is real-valued with spectral
radius equal to its operator norm~\cite{Reed-1980}, thus
%$\Sigma\subseteq[-\|\Hm\|,\|\Hm\|\,]$.
%
\begin{align}\label{eq:Spectral_Radius}
  \Sigma\subseteq[-\|\Hm\|,\|\Hm\|\,].
\end{align}
%
We are now ready to present the main results of this section. 
We start with the following corollary of \thmref{thm:Int_Rep_Sobolev}.


% For notational simplicity, denote
% the complex-valued function  
% $\tilde{\mu}_{jk}(\lambda)=\langle \Qb(\lambda)\vecg_j,\vecg_k\rangle$, 
% instead of $\tilde{\mu}_{\vecg_j\vecg_k}(\lambda)$, 
% where $\vecg_k=-\bGamma\Hm\vece_k$ is defined
% in~\eqref{eq:Resolvent_Rep}. Denote the real and imaginary
% parts of the function $\tilde{\mu}_{jk}(\lambda)$ by 
% $\Real\tilde{\mu}_{jk}(\lambda)$ and $\Imag\tilde{\mu}_{jk}(\lambda)$
% respectively. 
%
\begin{corollary}\label{cor:Int_Rep}
%
Let $\Qb(\lambda)$ denote the resolution of the identity corresponding
to the self-adjoint operator $\Mb$. Then, the components $\Sg^*_{jk}$
and $\Ag^*_{jk}$, 
$j,k=1,\ldots,d$, of the symmetric $\Sg^*$ and antisymmetric $\Ag^*$
parts of the effective diffusivity matrix $\Dg^*$ have the 
Stieltjes-Radon integral representations in~\eqref{eq:Integral_Rep_kappa*} with 
$\mu_{jk}$ replaced by the spectral measure $\tilde{\mu}_{jk}$ of
$\Mb$ in the $(\vecg_j,\vecg_k)$ state.
The bounds in \thmref{thm:Bounds} hold for these integral
representations for $\Sg^*_{jk}$ and $\Ag^*_{jk}$.
The mass $\tilde{\mu}^0_{jk}$ of 
the measure $\tilde{\mu}_{jk}$ is real-valued and satisfies
%
\begin{align}\label{eq:Mass}
  \tilde{\mu}^0_{jk}=\langle\Hm^T\bGamma\Hm\vece_j\cdot\vece_k\rangle,
        \qquad
        |\mu^0_{jk}|\leq\|\Hm\|^2<\infty.
\end{align}
%
%  
\end{corollary}



\noindent
\textbf{Proof of \corref{cor:Int_Rep}.}
Exactly as in the proof of \thmref{thm:Int_Rep_Sobolev},
equation~\eqref{eq:Eff_Diffusivity_Resolvent} and a direct analogue  
of equation~\eqref{eq:Spectral_Theorem}  
leads to the Stieltjes integral representations
in~\eqref{eq:Integral_Rep_kappa*}, involving a Stieltjes measure
$\tilde{\mu}_{jk}$ associated with the function of the spectral variable
$\lambda$ defined by $\tilde{\mu}_{jk}(\lambda)=\langle
\Qb(\lambda)\vecg_j,\vecg_k\rangle$. Here
$\vecg_k=-\bGamma\Hm\vece_k$
is defined in~\eqref{eq:Resolvent_Rep} and
$\{\Qb(\lambda)\}_{\lambda\in\Sigma}$ is the family of self-adjoint
projection operators that is in one-to-one correspondence with the
bounded linear self-adjoint operator $\Mb$ on the Hilbert space
$\Hc_{\times}$~\cite{Stone:64,Reed-1980}.
From equation~\eqref{eq:Mass_General} and the fact 
that $\bGamma$ 
is a self-adjoint projection operator on $\Hc_\times$,
the mass $\tilde{\mu}^0_{jk}$ of the measure $\tilde{\mu}_{jk}$ is real-valued and satisfies 
%
\begin{align}\label{eq:Mass}
  \tilde{\mu}^0_{jk}%=\int_I\d\mu_{jk}(z)
      %   =\int_I\d\langle\Qb(z) g_j,g_k\rangle
        =\langle \vecg_j, \vecg_k\rangle
        =\langle\bGamma\Hm \vece_j\cdot\bGamma\Hm \vece_k\rangle 
        =\langle\Hm^T\bGamma\Hm\vece_j\cdot\vece_k\rangle,
        \qquad
        |\tilde{\mu}^0_{jk}|\leq\|\Hm\|^2<\infty.
\end{align}
%
%hence $|\mu^0_{jk}|\leq\|\Hm\|^2<\infty$,
By analogy, the bounds in \thmref{thm:Bounds} also hold for these
integral representations for $\Sg^*_{jk}$ and $\Ag^*_{jk}$.
This concludes our proof of 
\corref{cor:Int_Rep} $\Box\,.$




It is worth making the following observation. In the current 
setting, the formulas for $\Sg^*_{jk}$ and $\Ag^*_{jk}$ 
in~\eqref{eq:Integral_Rep_kappa*} are computed with respect to the
standard basis $\{\vece_k\}$, through the definition of
$\tilde{\mu}_{jk}(\lambda)=\langle\Qb(\lambda)\vecg_j,\vecg_k\rangle$ with
$\vecg_k=-\bGamma\Hm\vece_k$. We now show that, given $\Sg^*_{jk}$  
and $\Ag^*_{jk}$, $j,k=1,\ldots,d$, the effective diffusivity matrix can
be computed relative to any directions. This is due to the bilinearity  
of the inner-product underlying the definition of $\tilde{\mu}_{jk}(\lambda)$. More
specifically, if $\vecxi,\veczeta\in\mathbb{R}^d$ are  
arbitrary directions of interest, then
$\langle\Qb(\lambda)\bGamma\Hm\vecxi,\bGamma\Hm\veczeta\rangle
=\sum_{jk}a_jb_k\langle\Qb(\lambda)\vecg_j,\vecg_k\rangle$, where the constants 
$a_j$ and $b_k$, $j,k=1,\ldots,d$, are the coordinates of the vectors
$\vecxi$ and $\veczeta$ with respect to the standard
basis. This immediately leads to integral representations for the
effective diffusivity matrix relative to any desired directions. This
observation had useful consequences
in~\cite{Fannjiang:1994:SIAM_JAM:333,Fannjiang:1997:1033}. 



%
\begin{theorem}\label{thm:Equivalence}
%
The spectral measure $\mu_{jk}$ in \thmref{thm:Int_Rep_Sobolev} is
identical to the spectral measure $\tilde{\mu}_{jk}$ in
\corref{cor:Int_Rep}. 
%
\begin{align}\label{eq:Equiv_measures_Hausdorff}
%\mu_{jk}^n=\tilde{\mu}_{jk}^n, \ 
%n=1,2,3,\ldots
%\quad
\mu_{jk}\equiv\tilde{\mu}_{jk}\,.
\end{align}
%  
%
\end{theorem}
%

Before we prove~\thmref{thm:Equivalence}, we give a brief outline of
the proof and display the key formulas that lead to the result
in~\eqref{eq:Equiv_measures_Hausdorff}.
The vector field $\vecg_k=-\bGamma\Hm\vece_k$ defined in 
equation~\eqref{eq:Resolvent_Rep} and the scalar field $g_k=(-\Delta)^{-1}u_k$
defined in~\eqref{eq:Resolvent_Rep_Sobolev} are (weakly) related by~\cite{Murphy:ADSTPF-2017}
%
\begin{align}\label{eq:g_relationship}
\vecg_k=\bnabla g_k\,.
\end{align}
%
The operator $\Ab=\bGamma\Hm\bGamma$ defined in equation~\eqref{eq:Resolvent_Rep} and 
the operator $A=\Delta^{-1}[\vecu\bcdot\bnabla]$ defined 
in~\eqref{eq:Resolvent_Rep_Sobolev}, hence $\Mb=-\imath\Ab$ and 
$M=-\imath A\,,$ are (weakly) related by
%
\begin{align}\label{eq:A_relationship}
\Ab\bnabla = \bnabla A,
\quad\quad
\Mb\bnabla = \bnabla M.
\end{align}
%
It follows that the masses and moments of the spectral measure 
$\mu_{jk}\,$, $j,k=1,\ldots,d$, 
in \thmref{thm:Int_Rep_Sobolev} and the spectral measure $\tilde{\mu}_{jk}$ 
in \corref{cor:Int_Rep} are identical. 
%
\begin{align}\label{eq:moment_equivalence}
\mu_{jk}^n=\tilde{\mu}_{jk}^n, 
\quad 
n=0,1,2,\ldots \ .
%\quad
%\mu_{jk}\equiv\tilde{\mu}_{jk}.
\end{align}
%
Since the self-adjoint operators $\Mb$ and $M$ are bounded, hence the
spectra of these operators are bounded subsets of $\mathbb{R}$ and 
the Hausdorff moment problem is 
determinate~\cite{Stone:64,Akhiezer:Book:1965}, 
this establishes equation~\eqref{eq:Equiv_measures_Hausdorff}. 




\noindent
\textbf{Proof of \thmref{thm:Equivalence}.}
In~\cite{Murphy:ADSTPF-2017}, equations~\eqref{eq:g_relationship} 
and~\eqref{eq:A_relationship} were established in a more general 
context than our considerations here, where $\vecu=\vecu(\vecx,t)$ 
is periodic in both space and time $t$. We will see that the proof 
of equation~\eqref{eq:moment_equivalence} essentially depends only 
on equations~\eqref{eq:g_relationship} 
and~\eqref{eq:A_relationship}. Consequently, en route, we will also
establish that equation~\eqref{eq:moment_equivalence} holds for this 
more general, time-dependent context.
Here, we will only sketch the key ideas that were developed at 
length in~\cite{Murphy:ADSTPF-2017}.
Equation~\eqref{eq:u_DH}, $\vecu=\bnabla\bcdot\Hm\,$, and the definitions 
$\bGamma=\bnabla(\Delta^{-1})\bnabla\bcdot\,$, 
$\vecg_k=-\bGamma\Hm\vece_k$, and $g_k=(-\Delta)^{-1}u_k$, together 
yield~\cite{Murphy:ADSTPF-2017} equation~\eqref{eq:g_relationship}
%
\begin{align}\label{eq:vecg_g}
  \vecg_j=-\bGamma\Hm\vece_j=\bnabla(-\Delta)^{-1}u_j=\bnabla g_j.
\end{align}
%
Consequently, by equation~\eqref{eq:Mass_General} the masses 
$\mu_{jk}^0$ and $\tilde{\mu}_{jk}^0$ of the spectral measures 
$\mu_{jk}$ and $\tilde{\mu}_{jk}$ are equal
%
\begin{align}
\tilde{\mu}_{jk}^0 = \langle \vecg_j\bcdot\vecg_k\rangle
                   = \langle \bnabla g_j\bcdot\bnabla g_k\rangle
                   = \langle g_j,g_k\rangle_{1,2}
                   = \mu_{jk}^0\,.
\end{align}




Since $\bGamma\bnabla\xi=\bnabla\xi$ (weakly)~\cite{Murphy:ADSTPF-2017},  
equation~\eqref{eq:u_DH} and equation~\eqref{eq:H_Hessian}, 
$\bnabla\bcdot[\Hm\bnabla]=[\bnabla\bcdot\Hm]\bcdot\bnabla$, imply 
that for all $\xi\in\Hs^{1,2}$
%
\begin{align}
\Ab\bnabla\xi = \bGamma\Hm\bGamma\bnabla\xi
              = \bGamma\Hm\bnabla\xi
              = \bnabla[(\Delta^{-1})\vecu\bcdot\bnabla]\xi 
              = \bnabla A\,\xi\,,
\end{align}
%
in a weak sense~\cite{Murphy:ADSTPF-2017}, which establishes 
equation~\eqref{eq:A_relationship}. It follows 
from~\eqref{eq:g_relationship} and~\eqref{eq:A_relationship} that
%
\begin{align}
\tilde{\mu}_{jk}^1 
&= \langle\Mb\vecg_j\bcdot\vecg_k\rangle
= \langle\Mb\bnabla g_j\bcdot\bnabla g_k\rangle
= \langle \bnabla M g_j\bcdot\bnabla g_k\rangle
= \langle M g_j,g_k\rangle_{1,2}
= \mu_{jk}^1
\\
\tilde{\mu}_{jk}^2 
&= \langle\Mb^2\vecg_j\bcdot\vecg_k\rangle
= \langle\Mb\bnabla M g_j\bcdot\bnabla g_k\rangle
= \langle \bnabla M^2 g_j\bcdot\bnabla g_k\rangle
= \langle M^2 g_j,g_k\rangle_{1,2}
= \mu_{jk}^2.
\notag
\end{align}
%
An inductive argument establishes equation~\eqref{eq:moment_equivalence}.
This concludes our proof of \thmref{thm:Equivalence} $\Box\,.$


We conclude this section with the following corollary of
\thmref{thm:Equivalence}. 
%
\begin{corollary}\label{cor:Equivalence}
%
For each $\,\xi\in\Hs^{1,2}$ we have $\bnabla\xi\in\Hc_\times\,$, and  
conversely, for each $\vecxi\in\Hc_\times$ there exists unique 
$\xi\in\Hs^{1,2}$ such that 
$\vecxi=\bnabla\xi$.
%
%\begin{align}
%\vecxi=\bnabla\xi.
%\end{align}
% 
Consequently,
by \thmref{thm:Equivalence}, for every $\xi,\zeta\in\Hs^{1,2}$ and
$\vecxi,\veczeta\in\Hc_\times\,$ related by $\vecxi=\bnabla\xi$ and
$\veczeta=\bnabla\zeta$ we have 
%
\begin{align}\label{eq:moment_equiv_general}
\mu_{\xi\zeta}^n=\tilde{\mu}_{\vecxi\veczeta}^n\,,
\quad
n=0,1,2,\ldots \ .
\end{align}
% Since the self-adjoint operators $\Mb$ and $M$ are bounded, 
% hence the spectrum of these operators are bounded subsets of 
% $\mathbb{R}$ and the Hausdorff moment problem is 
% determinate~\cite{Stone:64,Akhiezer:Book:1965},
This establishes 
that the spectral measures $\mu_{\xi\zeta}$ and 
$\tilde{\mu}_{\vecxi\veczeta}$ are \emph{identical}, 
%
\begin{align}\label{eq:measure_equiv_general}
%\mu_{jk}^n=\tilde{\mu}_{jk}^n, \ 
%n=1,2,3,\ldots
%\quad
\mu_{\xi\zeta}\equiv\tilde{\mu}_{\vecxi\veczeta}\,.
\end{align}
%
Moreover, equation~\eqref{eq:moment_equiv_general} also holds 
for the class of space-time periodic fluid velocity fields 
$\vecu=\vecu(\vecx,t)$ described in~\cite{Murphy:ADSTPF-2017}. 
%
%
\end{corollary}
%

\noindent
\textbf{Proof of \corref{cor:Equivalence}.}
The Hilbert spaces $\Hs^{1,2}$ and $\Hc_\times$ are in one-to-one
isometric correspondence~\cite{Murphy:ADSTPF-2017}. More 
specifically, for every $\xi\in\Hs^{1,2}$ we have 
$\bnabla\xi\in\Hc_\times$ satisfying $\|\xi\|_{1,2}=\|\bnabla\xi\|$ 
and $\|A\xi\|_{1,2}=\|\Ab\bnabla\xi\|$. Conversely, for each 
$\vecxi\in\Hc_\times$ there exists unique~\cite{Murphy:ADSTPF-2017}
$\xi\in\Hs^{1,2}$ (up to equivalence class) such that 
$\vecxi=\bnabla\xi\,,$ $\|\vecxi\|=\|\xi\|_{1,2}\,,$ and 
$\|\Ab\vecxi\,\|=\|A\xi\|_{1,2}\,.$ By this and 
\thmref{thm:Equivalence} we have 
equations~\eqref{eq:moment_equiv_general} 
and~\eqref{eq:measure_equiv_general}.



The proof of equation~\eqref{eq:moment_equivalence} depends only 
on equations~\eqref{eq:g_relationship} and~\eqref{eq:A_relationship}, 
which also hold~\cite{Murphy:ADSTPF-2017} for the class of space-time
periodic fluid velocity fields $\vecu=\vecu(\vecx,t)$ described 
in~\cite{Murphy:ADSTPF-2017}. 
The argument in the previous paragraph above also holds for this
time-dependent setting~\cite{Murphy:ADSTPF-2017}. Therefore, 
equation~\eqref{eq:moment_equiv_general} holds for the time-dependent 
setting as well. This concludes our proof of \corref{cor:Equivalence} $\Box\,.$


In order to establish equation~\eqref{eq:measure_equiv_general} for 
the time-dependent setting associated with operators having unbounded 
spectra~\cite{Murphy:ADSTPF-2017}, one must first establish 
Carleman's criterion~\cite{Akhiezer:Book:1965}, for example, 
associated with the Stieltjes or Hamburger moment problems. 
This requires an involved analysis that is outside of the scope of 
the current work. Our results in this direction will be published 
elsewhere. 









\section{Discrete setting:  Hilbert space of curl-free fields}
\label{app:Matrix_Formulation_Curl} 
% 
Here, we provide a matrix formulation for the effective parameter
problem discussed in \appref{app:Integral_Reps_Curl_free}, which
provides discrete versions of the Stieltjes integral representations
for $\Dg^*$ shown in equation~\eqref{eq:Integral_Rep_kappa*}. These
integrals involve a discrete spectral measure $\mu_{jk}$ analogous to
the discrete measure in
equation~\eqref{eq:Disc_Spec_Measure_Matrix_Sobolev}. More
specifically, we use a discretized version of the cell problem in
equation~\eqref{eq:Cell_Problem}  
and~\eqref{eq:Resolvent_Rep}, written as
$(\varepsilon\Ib+\Ab)\bnabla\chi_k=\vecg_k$, to express the discrete 
spectral measure $\mu_{jk}$ explicitly in terms of the eigenvalues and
eigenvectors of a matrix representation $\Am$ of the 
% random
operator $\Ab=\bGamma\Hm\bGamma$. We then develop a \emph{projection
  method} which additionally shows that the discrete measure $\mu_{jk}$
is actually determined by the eigenvalues and eigenvectors of a matrix
that is much smaller than the matrix $\Am$. This projection method is
used in \appref{app:Eigenvalue_method} to establish the equivalence of the
two discrete approaches given in this appendix and
\secref{sec:Matrix_Sobolev}, for the setting where the matrix gradient
$\nabla$ is rank-deficient. This equivalence proof establishes, en route, that
the common discrete spectral measure can be computed by a method that
combines the computational benefits of both approaches.




% \subsection{Matrix formulation of the effective parameter
%   problem} \label{app:Matrix_Formulation_Curl} 
% %
From the discussion in \secref{sec:Matrix_Sobolev}, the discrete
representation of the projection operator
$\bGamma=\bnabla(\Delta^{-1})\bnabla\bcdot$ is given by the 
symmetric projection matrix
$\Gamma=\nabla(\nabla^T\nabla)^{-1}\nabla^T$ satisfying 
$\Gamma^{\,2}=\Gamma$ and $\Gamma\nabla=\nabla$, 
where $(\nabla^T\nabla)^{-1}$ is now interpreted as a matrix inversion. We
assume here that the matrix $\nabla$ is of full-rank so that $(\nabla^T\nabla)^{-1}$
exists. The rank-deficient case, where the matrix $\nabla^T\nabla$ is  
singular, is examined in~\appref{app:Eigenvalue_method}. 
In this way, the integro-differential operator $\Ab=\bGamma\Hm\bGamma$ 
is represented by
an antisymmetric matrix 
%$\Am$ 
$\Am=\Gamma\Hm\Gamma$ 
satisfying $\Am^T=-\Am$.
%, which is not to be confused with the antisymmetric part $\Ag^*$ of
%the effective diffusivity matrix $\Dg^*$.
In a similar way, the vectors
$\vecg_k=-\bGamma\Hm\vece_k$, $k=1,\ldots,d$, are redefined for this matrix
setting. For simplicity, we will not make a notational distinction
between the continuum and discrete cases for the vectors $\vecg_k$ and 
$\vece_k$ as well as the matrix $\Hm$, as the
context will be clear. 



 
The spectrum of the antisymmetric matrix $\Am$ of size $N$, say,
is comprised of eigenvalues $\upsilon_n$, $n=1,\ldots,N$, with
corresponding eigenvectors  $\vecw_n$ satisfying 
$\Am\vecw_n=\upsilon_n\vecw_n$. Since $\Am$ is skew-symmetric, the
eigenvalues $\upsilon_n$ are purely imaginary~\cite{Horn_Johnson-1990},
$\upsilon_n=\imath\lambda_n$ with 
$\lambda_n\in\mathbb{R}$. Therefore, the matrix $\Mm=-\imath\Am$ is Hermitian
$(\Mm^\dagger=\Mm)$ and it has the same eigenvectors $\vecw_n$ as the matrix
$\Am$ and real eigenvalues given by $\lambda_n=\Imag\upsilon_n$. The
eigenvectors $\vecw_n$, $n=1,\ldots,N$, of the Hermitian matrix $\Mm$
form an orthonormal basis for
$\mathbb{C}^N$~\cite{Keener-2000,Horn_Johnson-1990}, i.e.,  
$\vecw_n^{\;\dagger}\vecw_m=\delta_{nm}$ and for every $\vecxi\in\mathbb{C}^N$ we
have $\vecxi=\sum_n(\vecw_n^\dagger\vecxi)\vecw_n
=\left(\sum_n\vecw_n\vecw_n^\dagger\right)\vecxi\,.$ Consequently,
defining $\Qc_n=\vecw_n\vecw_n^\dagger$, $n=1,\ldots,N$, to be the
mutually orthogonal Hermitian projection matrices onto 
the eigenspaces spanned by the $\vecw_n$, we have the following
analogue of 
equation~\eqref{eq:Matrix_Rep_Spec_Theorem_Sobolev} 
%
\begin{align}\label{eq:Matrix_Rep_Spec_Theorem}
  \sum_{n=1}^N\Qc_n=\Ib\,, \qquad
  \Qc_n=\vecw_n\vecw_n^\dagger\,,  \qquad
  \Qc_l \Qc_m=\Qc_l\,\delta_{l m}\,.
\end{align}
%


With an argument similar to the one in the paragraph following
equation~\eqref{eq:Matrix_Rep_Spec_Theorem_Sobolev}, one can show that
% In a similar way to 
% We now use equation~\eqref{eq:Matrix_Rep_Spec_Theorem} to prove the
% spectral theorem in~\eqref{eq:Spectral_Theorem} for this matrix setting.
% From $\Mm\vecw_n=\lambda_n\vecw_n$ and
% equation~\eqref{eq:Matrix_Rep_Spec_Theorem} we have  
% that $\Mm\Qc_n=\lambda_n\Qc_n$. This formula
% and~\eqref{eq:Matrix_Rep_Spec_Theorem} then imply that the matrix 
% $\Mm$ has the spectral decomposition $\Mm=\sum_n\lambda_n\Qc_n$. By the mutual
% orthogonality of the projection matrices $\Qc_n$ and by induction, we
% have that  
% $\Mm^m=\sum_n\lambda_n^m\Qc_n$ for all $m\in\mathbb{N}$. This, in turn,
% implies that $f(\Mm)=\sum_nf(\lambda_n)\Qc_n$ for any polynomial
% $f:\mathbb{R}\mapsto\mathbb{C}$. 
% From the mutual orthogonality and the symmetry of the projection matrices
% $\Qc_n$ it follows that, 
for 
all $\vecxi,\veczeta\in\mathbb{C}^N$ and complex-valued polynomials
$f(\lambda)$ and $h(\lambda)$, the bilinear functional
$\langle f(\Mm)\vecxi\bcdot h(\Mm)\veczeta\rangle$
has the integral representation   
 in equation~\eqref{eq:Spectral_Theorem}, with
$M$ substituted by $\Mm$ and the scalars $\xi$ and $\zeta$ replaced by
vectors $\vecxi$ and $\veczeta$. Moreover,
the complex-valued function 
$\mu_{\xi\zeta}(\lambda)=\langle Q(\lambda)\xi,\zeta\rangle$ in equation~\eqref{eq:Spectral_Theorem} is
now given by $\mu_{\xi\zeta}(\lambda)=\langle\Qc(\lambda)\vecxi\bcdot\veczeta\rangle$, where the
associated matrix representation $\Qc(\lambda)$ of the projection operator
$Q(\lambda)$ and the discrete spectral measure
$\d\mu_{\xi\zeta}(\lambda)$ are given by the following analogue of
equation~\eqref{eq:Disc_Spec_Measure_Matrix_Sobolev}  
% 
\begin{align}\label{eq:Disc_Spec_Measure_Matrix}
  \Qc(\lambda)=\sum_{n:\;\lambda_n\leq\lambda}\theta(\lambda-\lambda_n)\Qc_n\,,
  \qquad
  \d\mu_{\xi\zeta}(\lambda)=\sum_{n:\;\lambda_n\leq\lambda}\langle\delta_{\lambda_n}(\d\lambda)[\Qc_n\vecxi\bcdot\veczeta]\rangle\,.
  %\d\mu_{\xi\zeta}(\lambda)=\sum_{n=1}^N\langle\delta_{\lambda_n}(\d\lambda)[\Qc_n\vecxi\bcdot\veczeta]\rangle.
  %\quad
  %\theta(\lambda-\lambda_n)=\int_{-\infty}^{\lambda}\delta_{\lambda_n}(\d\lambda),
\end{align}
%
We are now ready to present the main results of this section, given in
the following corollary of \thmref{thm:Int_Rep_Sobolev_Matrix}. 






% The spectral theorem for symmetric matrices described above also holds
% for functions of the form $f(\lambda)=a\lambda^l(b+c\lambda)^{-m}$,
% with $l,m\in\mathbb{N}$, 
% $a,b,c\in\mathbb{C}$, and $b+c\lambda\neq0$ for all $\lambda\in\Sigma$. We will demonstrate this
% for the special cases that arise in the functional formulas for
% $\Sg^*_{jk}$ and $\Ag^*_{jk}$  in
% equation~\eqref{eq:Eff_Diffusivity_Resolvent}, with $\Ab$ substituted
% with $\Am$, which 
% yield the integral representations
% in~\eqref{eq:Integral_Rep_Discrete} involving a discrete spectral measure
% $\d\mu_{jk}(\lambda)$ associated with the measure 
% in~\eqref{eq:Disc_Spec_Measure_Matrix}. The argument involving the
% function $f(\lambda)$ above is a simple extension of that given here.  



% As $\Am$ is a real-valued matrix, its eigenvalues $\upsilon_n=\imath\lambda_n$,
% $n=1,\ldots,N$, and eigenvectors $\vecw_n$ come in complex-conjugate
% pairs \cite{Horn_Johnson-1990}. Therefore, if the size $N$ of $\Am$ is
% even, then we may re-number the index set $I_N$ as $I_N=\{-N/2,\ldots,-1,1,\ldots
% ,N/2\}$ such that $\upsilon_{-n}=\overline{\upsilon_n}=-\upsilon_n$ and
% $\vecw_{-n}=\overline{\vecw_n}$. If $N$ is odd then $\upsilon_0=0$ is also an
% eigenvalue with a 
% \emph{real-valued} eigenvector $\vecw_0$. Denoting by $\Wm$ the matrix
% with columns consisting of the eigenvectors $\vecw_n$,
% $\Upsilon=\text{diag}(\upsilon_{-N/2},\ldots,\upsilon_{N/2})$ the diagonal matrix with
% eigenvalues $\upsilon_n$ on the main diagonal, and
% $\Lambda=\text{diag}(\lambda_{-N/2},\ldots,\lambda_{N/2})$, we have that 
% $\Am=\Wm\Upsilon\Wm^\dagger=\imath\Wm\Lambda\Wm^\dagger$, where
% the matrix $\Wm$ is unitary
% $\Wm^\dagger\Wm=\Wm\Wm^\dagger=\Ib$~\cite{Horn_Johnson-1990}. Therefore, the
% matrix $\Mm=-\imath\Am=\Wm\Lambda\Wm^\dagger$ is Hermitian $(\Mm^\dagger=\Mm)$.


%
\begin{corollary}\label{cor:Int_Rep_Matrix}
%
Consider the standard eigenvalue problem
$\Mm\vecw_n=\lambda_n\vecw_n$ associated with the Hermitian matrix
$\Mm=-\imath\Am$. Let $\Wm$ be the matrix with 
columns consisting of the eigenvectors $\vecw_n$ and $\Lambda$ be the
diagonal matrix with eigenvalues $\lambda_n$ on its diagonal. 
The discrete, matrix representations of the bilinear functional
formulas for $\Sg^*_{jk}$ and $\Ag^*_{jk}$ in
equation~\eqref{eq:Eff_Diffusivity} with $\Hm$ replaced by $\Am$, as 
discussed right before equation~\eqref{eq:Eff_Diffusivity_Resolvent},
are given by   
%
\begin{align}\label{eq:Matrix_Functionals}
  \Sg^*_{jk}
  =\varepsilon(\delta_{jk}
  +\langle\nabla\vecchi_j\bcdot\nabla\vecchi_k\rangle)\,,
  \quad
  \Ag^*_{jk}
  =\langle\Am\nabla\vecchi_j\bcdot\nabla\vecchi_k\rangle\,,   
\end{align}
%
which is analogous to equation~\eqref{eq:Matrix_Functionals_Sobolev}.
Also, the discrete representation of the resolvent formula for 
$\bnabla\chi_j$ in equation~\eqref{eq:Resolvent_Rep} 
is given by
%
\begin{align}\label{eq:Disc_Reslovent_CurL}
  \nabla\vecchi_j=\Wm(\varepsilon\Ib+\imath\Lambda)^{-1}\Wm^\dagger\vecg_j\,,
  \quad
  \vecg_j=-\Gamma\Hm\vece_j\,,
\end{align}
%
which is analogous to equation~\eqref{eq:Matrix_chij}.
The discrete representations of the bilinear functional
formulas for $\Sg^*_{jk}$ and $\Ag^*_{jk}$
in~\eqref{eq:Eff_Diffusivity_Resolvent} are given by 
% 
\begin{align}\label{eq:Functionals_kappa_alpha}
  &\Sg^*_{jk}
  %=\langle(\varepsilon\Ib+\Am)^{-1}\vecg_j\bcdot(\varepsilon\Ib+\Am)^{-1}\vecg_k\rangle 
  =\varepsilon(\,\delta_{jk}+\langle(\varepsilon\Ib+\imath\Lambda)^{-1}\Wm^\dagger\vecg_j\bcdot(\varepsilon\Ib+\imath\Lambda)^{-1}\Wm^\dagger\vecg_k\rangle\,) ,
\\
&\Ag^*_{jk}
  %=\langle\Am(\varepsilon\Ib+\Am)^{-1}\vecg_j\bcdot(\varepsilon\Ib+\Am)^{-1}\vecg_k\rangle
  =\langle\imath\Lambda(\varepsilon\Ib+\imath\Lambda)^{-1}\Wm^\dagger\vecg_j\bcdot(\varepsilon\Ib+\imath\Lambda)^{-1}\Wm^\dagger\vecg_k\rangle
\notag
\end{align}
%
which are analogous to those in
equation~\eqref{eq:Functionals_kappa_alpha_Sobolev}. 
Consequently, the formulas for $\Sg^*_{jk}$ and $\Ag^*_{jk}$
in~\eqref{eq:Functionals_kappa_alpha} have the following series
representations   
%
\begin{align}\label{eq:Discrete_Integrals_full_rank}
  \Sg^*_{jk}/\varepsilon-\delta_{jk}=\sum_{n=1}^{N}
      \frac{\Real[\,\overline{(\vecw_n^\dagger\vecg_j)}
                              (\vecw_n^\dagger\vecg_k)\,]}
           {\varepsilon^2+\lambda_n^2}\,,
  \qquad
  \Ag^*_{jk}=\sum_{n=1}^{N}
      \frac{\lambda_n\,\Imag[\,\overline{(\vecw_n^\dagger\vecg_j)}
                                         (\vecw_n^\dagger\vecg_k)\,]}
           {\varepsilon^2+\lambda_n^2}\,,
\end{align}
%
which is analogous to 
equation~\eqref{eq:Discrete_Integrals_full_rank_Sobolev}.
Finally, recalling the projection matrix $\Qc_n=\vecw_n\vecw_n^\dagger$ 
in~\eqref{eq:Matrix_Rep_Spec_Theorem}, we have 
%
%
\begin{align}\label{eq:weights}
  \overline{(\vecw_n^\dagger\vecg_j)}(\vecw_n^\dagger\vecg_k)
  %=\vecz_n\vecz_n^{\,\dagger}\vecu_j\bcdot\vecu_k
  %=\vecz_n\vecz_n^{\,\dagger}[\nabla^T\nabla]\vecg_j
    %\bcdot[\nabla^T\nabla]\vecg_k
  %=[\nabla\vecz_n][\nabla\vecz_n]^{\,\dagger}\nabla\vecg_j
   %\bcdot\nabla\vecg_k
  =\Qc_n\vecg_j\bcdot\vecg_k,
\end{align}
%
which is analogous to equation~\eqref{eq:Sobolev_weights}.
It follows from equation~\eqref{eq:weights}
that the series representations for $\Sg^*_{jk}$ and $\Ag^*_{jk}$
in~\eqref{eq:Discrete_Integrals_full_rank} have the Stieltjes   
integral representations in equation~\eqref{eq:Integral_Rep_kappa*},
involving the discrete spectral measure $\mu_{jk}$ 
in~\eqref{eq:Disc_Spec_Measure_Matrix} with 
$\vecxi=\vecg_j=-\Gamma\Hm\vece_j$ and $\veczeta=\vecg_k=-\Gamma\Hm\vece_k\,$.
%
\end{corollary}
%


\noindent
\textbf{Proof of \corref{cor:Int_Rep_Matrix}.}
Equation~\eqref{eq:Matrix_Functionals} follows from
the discrete version of~\eqref{eq:Eff_Diffusivity} and 
$\Gamma\nabla=\nabla$ so 
$\langle\Hm\,\nabla\vecchi_j\bcdot\nabla\vecchi_k\rangle 
=\langle\Am\,\nabla\vecchi_j\bcdot\nabla\vecchi_k\rangle$.
Consider the spectral decomposition $\Mm=\Wm\Lambda\Wm^\dagger$ of
the Hermitian matrix $\Mm=-\imath\Am$ involving the real eigenvalues
$\lambda_n$ comprising the main  
diagonal of the diagonal matrix $\Lambda$ and the eigenvectors 
$\vecw_n$ comprising the columns of the unitary matrix $\Wm$
satisfying 
$\Wm^\dagger\Wm=\Wm\Wm^\dagger=\Ib$~\cite{Horn_Johnson-1990}.  
%
% This spectral decomposition of $\Am$ demonstrates that
% the matrix $(\varepsilon\Ib+\Am)^{-1}$, is well defined for all $0<\varepsilon<\infty$. In
% particular, since $\Wm^\dagger=\Wm^{-1}$, it has the following useful 
% representation $(\varepsilon\Ib+\Am)^{-1}=\Wm(\varepsilon\Ib+\imath\Lambda)^{-1}\Wm^\dagger$, where
% $(\varepsilon\Ib+\imath\Lambda)^{-1}$ is a diagonal matrix with entries $1/(\varepsilon+\imath\lambda)$. This
% allows the discrete version of the resolvent formula in
% equation~\eqref{eq:Resolvent_Rep} to be written as the following
% analogue of equation~\eqref{eq:Matrix_chij}
%
Equation~\eqref{eq:Disc_Reslovent_CurL} follows from 
this spectral decomposition of $\Am=\imath\Mm$, the discrete version of 
the resolvent formula in equation~\eqref{eq:Resolvent_Rep} written 
as $\nabla\vecchi_j=(\varepsilon\Ib+\Am)^{-1}\vecg_j$, 
the fact that $\Wm$ is unitary satisfying $\Wm^\dagger=\Wm^{-1}$,
and elementary properties of matrix
inversion~\cite{Horn_Johnson-1990}. Substituting the resolvent formula
for $\nabla\vecchi_j$ in~\eqref{eq:Disc_Reslovent_CurL}
into equation~\eqref{eq:Matrix_Functionals} and using $\Wm^\dagger=\Wm^{-1}$
yields equation~\eqref{eq:Functionals_kappa_alpha}. The quadratic form 
$\Wm^\dagger\vecg_j\bcdot\Wm^\dagger\vecg_k$ arising
in~\eqref{eq:Functionals_kappa_alpha} can be written in terms of the
projection matrices $\Qc_n=\vecw_n\vecw_n^\dagger$ defined
in~\eqref{eq:Matrix_Rep_Spec_Theorem} as follows   
%
\begin{align}\label{eq:Quadratic_W}
  \Wm^\dagger\vecg_j\bcdot\Wm^\dagger\vecg_k
  =\sum_{n=1}^N\overline{(\vecw_n^\dagger\vecg_j)}(\vecw_n^\dagger\vecg_k)
  =\sum_{n=1}^N\Qc_n\,\vecg_j\bcdot\vecg_k\,.  
\end{align}
%
%which is analogous to equation~\eqref{eq:Quadratic_Z}.
Exactly as in \thmref{thm:Int_Rep_Sobolev_Matrix} we have
equation~\eqref{eq:Discrete_Integrals_full_rank}, which can be written
in terms of the integral representations  in
equation~\eqref{eq:Integral_Rep_kappa*} involving the discrete
spectral 
measure $\d\mu_{jk}(\lambda)$ in
equation~\eqref{eq:Disc_Spec_Measure_Matrix} with
$\vecxi=\vecg_j=-\Gamma\Hm\vece_j$ and
$\veczeta=\vecg_k=-\Gamma\Hm\vece_k$.
This concludes our proof of \corref{cor:Int_Rep_Matrix} $\Box\,.$



% Exactly as in \secref{sec:Integral_Reps_Sobolev}, we may rewrite
% these integral representations for $\Sg^*_{jk}$ and $\Ag^*_{jk}$ in 
% equation~\eqref{eq:Integral_Rep_Discrete}, involving the
% \emph{complex measure} $\mu_{jk}$, as the integral representations in
% equation~\eqref{eq:Integral_Rep_kappa*} involving the real \emph{signed
%   measures}  $\Real\mu_{jk}$ and $\Imag\mu_{jk}$. The associated
% real-valued functions $\Real\mu_{jk}(\lambda)$ and $\Imag\mu_{jk}(\lambda)$ 
% are given by equation~\eqref{eq:Signed_mu_Dis} with
% $[\nabla(\Qm_n+\overline{\Qm}_n)\vecg_j\bcdot\nabla\vecg_k\,]$
% substituted by
% $[(\Qc_n+\overline{\Qc}_n)\vecg_j\bcdot\vecg_k\,]$, for example.
% Furthermore, due to the generalized eigenvalues and eigenvectors of
% the antisymmetric matrix $\Am=\Gamma\Hm\Gamma$ coming in complex
% conjugate pairs, the discrete 
% measure $\mu_{jk}$ depends only on a subset of the index set $I_N$ in
% \eqref{eq:Quadratic_W} and satisfy
% equation~\eqref{eq:Real_Spectral}, with
% $[\nabla\Qm_n\vecg_j\bcdot\nabla\vecg_k\,]$
% substituted by
% $[\Qc_n\vecg_j\bcdot\vecg_k\,]$.




We conclude this section with the development of a \emph{projection method}
that is used in the proof of \thmref{thm:Spectral_Equivalence_Rank_Def}
below. This method shows that the presence of the projection matrix
$\Gamma$ in the operator $\Gamma\Hm\Gamma$ and
$\vecg_k=-\Gamma\Hm\vece_k$ projects out contributions of the null space
of $\Gamma$ in the series representation for $\Dg^*$
in~\eqref{eq:Discrete_Integrals_full_rank}. This is reminiscent of
problems encountered for many elliptic equations on periodic domains,
where the analysis requires a deficient dimension to be
projected out~\cite{Bensoussan:Book:1978}. 
\thmref{thm:Spectral_Equivalence_Rank_Def}  establishes that
the discrete framework developed in \secref{sec:Matrix_Sobolev} and
this section yield equivalent Stieltjes integral representations for
the effective diffusivity matrix $\Dg^*$, involving a discrete
spectral measure, and also generalizes this result to the setting
where the matrix $\nabla$ is rank-deficient. The proof of 
\thmref{thm:Spectral_Equivalence_Rank_Def} demonstrates that the
common discrete spectral measure can be computed by a method that 
combines the computational benefits of both approaches. This, in turn,
is used in our numerical computations of Stieltjes integral
representations of $\Dg^*$ in \secref{sec:Num_Results}.


%
\begin{theorem}[Projection Method] \label{thm:Projection_Method}
The real-symmetric projection matrix $\Gamma$ of size $N$ has
eigenvalues satisfying $\gamma_n=0,1$, hence the spectral decomposition
%
\begin{align}\label{eq:Gamma_spec_decomp}
\Gamma=\Pm\Gm\Pm^T,
\quad
\Gm=\text{diag}(\mathbf{0}_{N_0},\mathbf{1}_{N_1}),
\quad
\Pm=[\Pm_0\;\Pm_1].
\end{align}
%
Here, $\mathbf{0}_{N_0}$ and $\mathbf{1}_{N_1}$ are vectors of zeros
and ones with $N_0$ and $N_1$ components, respectively, where
$N=N_0+N_1$. Moreover, the columns comprising the $N\times N_0$ matrix
$\Pm_0$ and 
the $N\times N_1$ matrix $\Pm_1$ are orthonormal eigenvectors that
span the null space and range of $\Gamma$, respectively. Consequently, 
the matrix $\Gamma$ can be written as
%
\begin{align}\label{eq:Eig_Decomp_Gamma}
  \Gamma=\Pm_1\Pm_1^T.
\end{align}
%



Consider the spectral composition of the antisymmetric
matrix $\Pm_1^T\Hm\Pm_1$ of size $N_1\,,$
%
\begin{align}\label{eq:Eig_Decomp_PHP}
  \Pm_1^T\Hm\Pm_1=\imath\Rm_{11}\Lambda_{11}\Rm_{11}^\dagger\,,
\end{align}
%
where $\Rm_{11}$ is a unitary matrix,
$\Rm_{11}^\dagger\Rm_{11}=\Rm_{11}\Rm_{11}^\dagger=\Ib_{11}$, 
$\Lambda_{11}$ is a real-valued diagonal matrix, and $\Ib_{11}$ is the
identity matrix of size $N_1\times N_1$. It follows that the matrix
$\Am=\imath\Gamma\Hm\Gamma$ has the spectral decomposition 
% 
\begin{align}\label{eq:Eig_Block_Matrices}
%
\Am=\imath\Wm\Lambda\Wm^\dagger,
\quad
\Wm=[\Pm_0 \ \ \Pm_1\Rm_{11}],
\quad
\Lambda=\text{diag}(\Om_{00} \ \Lambda_{11}).
% \Wm=\Pm\Rm,
% \qquad
% \Rm=
% \left[
%   \begin{array}{ccc}
%     \Ib_{00}&\Om_{01}\\
%     \Om_{10}&\Rm_{11}  
%     \end{array}
% \right],
% \quad
% \Lambda=
% \left[
%   \begin{array}{ccc}
%     \Om_{00}&\Om_{01}\\
%     \Om_{10}&\Lambda_{11}
%     \end{array}
% \right],
\end{align}
% 
% where $\Ib_{00}$ is the identity matrix of size $N_0\times N_0$.
Equations~\eqref{eq:Eig_Decomp_Gamma}
and~\eqref{eq:Eig_Block_Matrices}, and the mutual orthogonality of the
matrices $\Pm_0$ and $\Pm_1$ yield
%
\begin{align}\label{eq:Wt_gk}
  \Wm^\dagger\vecg_k
  =\Wm^\dagger\Gamma\Hm\vece_k
  =[\Om_{0} \ \; \Pm_1\Rm_{11}]^\dagger\Hm\vece_k\,,
\end{align}
%
where $\Om_{0}$ is a matrix of zeros of size $N\times
N_0\,$. Consequently, equation~\eqref{eq:Quadratic_W} can be written
%
\begin{align}\label{eq:Quad_Proj_method}
  \Wm^\dagger\vecg_j\bcdot\Wm^\dagger\vecg_k
  =(\Pm_1\Rm_{11})^\dagger\Hm\vece_k\bcdot(\Pm_1\Rm_{11})^\dagger\Hm\vece_k\,.
\end{align}
%
Moreover, the spectral weights in equation~\eqref{eq:Quadratic_W}
associated with $\gamma_n=0$ are identically zero
$[\Qc_n\vecg_j\bcdot\vecg_k]=0\,.$ 
\end{theorem}



\noindent
\textbf{Proof of \thmref{thm:Projection_Method}.}
% Since $\Gamma$ is a real-symmetric projection matrix of size $N$, its
% eigenvalues $\gamma_n$, $n=1,\ldots,N$, satisfy
% $\gamma_n=0,1$~\cite{Horn_Johnson-1990}. Consequently, $\Gamma$ has the spectral     
% decomposition $\Gamma=\Pm\Gm\Pm^T$, where $\Pm$ is an orthogonal matrix,
% $\Pm\Pm^T=\Pm^T\Pm=\Ib$, with columns consisting of the eigenvectors
% of $\Gamma$, and the diagonal matrix $\Gm$ has the
% eigenvalues $\gamma_n$ along its main diagonal. Write $\Pm=[\Pm_0\;\Pm_1]$,
% where the columns of the $N\times N_0$ matrix $\Pm_0$ and the $N\times N_1$ matrix
% $\Pm_1$ are orthonormal eigenvectors that span the null space and
% range of $\Gamma$, respectively, with
% $N_0+N_1=N$~\cite{Horn_Johnson-1990}. It follows that 
% $\Gm=\text{diag}(\mathbf{0}_{N_0},\mathbf{1}_{N_1})$, where 
% $\mathbf{0}_{N_0}$ and $\mathbf{1}_{N_1}$ are vectors of zeros
% and ones of length $N_0$ and $N_1$, respectively. This, in turn,
% implies that the matrix $\Gamma$ can be written as
% equation~\eqref{eq:Eig_Decomp_Gamma}.
% The matrix $\Pm$ being orthogonal implies that $\Pm_1^T\Pm_1=I_{11}$,
% where $\Ib_{11}$ is the identity matrix of size $N_1\times N_1$. This
% demonstrates that the matrix $\Gamma$ in \eqref{eq:Eig_Decomp_Gamma}
% satisfies $\Gamma^2=\Gamma$. Moreover, $\Gamma$ projects 
% vectors in $\mathbb{R}^N$ on to the subspace spanned by the columns of
% $\Pm_1$, comprised of eigenvectors with eigenvalues $\gamma_n=1$.
%
Using the spectral decomposition $\Gamma=\Pm\Gm\Pm^T$
in~\eqref{eq:Gamma_spec_decomp}, we write the  
matrix $\Am=\Gamma\Hm\Gamma$ as
$\Am=\Pm[\Gm(\Pm^T\Hm\Pm)\Gm]\Pm^T
=\Pm\,\text{diag}(\Om_{00}\ \Pm_1^T\Hm\Pm_1 )\,\Pm^T$,
% $\Am=\Pm[\Gm(\Pm^T\Hm\Pm)\Gm]\Pm^T$.  The block
% matrix form of the matrix $\Gm(\Pm^T\Hm\Pm)\Gm$ is given by  
% %
% \begin{align}\label{eq:Block_Matrices}
% %
% \Gm(\Pm^T\Hm\Pm)\Gm=
% \left[
%   \begin{array}{cc}
%     \Om_{00}&\Om_{01}\\
%     \Om_{10}&\Ib_{11}  
%     \end{array}
% \right]
% \left[
%   \begin{array}{c}
%     \Pm_0^T\\\Pm_1^T   
%     \end{array}
% \right]
% \Hm
% \left[
%   \begin{array}{cc}
%     \Pm_0&\Pm_1\\   
%     \end{array}
% \right]
% \left[
%   \begin{array}{cc}
%     \Om_{00}&\Om_{01}\\
%     \Om_{10}&\Ib_{11}   
%     \end{array}
% \right]
% =
% \left[
%   \begin{array}{cc}
%     \Om_{00}&\Om_{01}\\
%     \Om_{10}&\Pm_1^T\Hm\Pm_1 
%     \end{array}
% \right].
% \end{align}
% %
% Here $\Om_{ab}$ is a matrix of zeros of size $N_a\times N_b$, where $a,b=0,1$.
where $\Om_{00}$ is a matrix of zeros of size $N_0\times N_0$. Due
to the skew-symmetry of $\Hm$, the $N_1\times N_1$ matrix
$\Pm_1^T\Hm\Pm_1$ is also skew-symmetric. Consequently, it has the
spectral decomposition in equation~\eqref{eq:Eig_Decomp_PHP}, hence
$\Am=\Pm\,\text{diag}(\Om_{00}\
\imath\Rm_{11}\Lambda_{11}\Rm_{11}^\dagger)\,\Pm^T$. Writing this in
block matrix form~\cite{Murphy:2015:CMS:13:4:825} and multiplying
yields
equation~\eqref{eq:Eig_Block_Matrices}. Equations~\eqref{eq:Eig_Decomp_Gamma}
and~\eqref{eq:Eig_Block_Matrices}, and the mutual orthogonality of the
matrices $\Pm_0$ and $\Pm_1$ yield equation~\eqref{eq:Wt_gk}. This
concludes our proof of \thmref{thm:Projection_Method} $\Box\,.$



% Equation \eqref{eq:Eig_Block_Matrices} demonstrates that the
% eigenvalues $\imath\lambda_n$ of the matrix $\Am$ are zero 
% for all $n=1,\ldots,N_0$. We now show that the associated spectral weights 
% $[\Qc_n\vecg_j\bcdot\vecg_k]$ are also zero for all $n=1,\ldots,N_0$. Here,
% $\vecg_j=-\Gamma\Hm\vece_j$, $\Qc_n=\vecw_n\vecw_n^{\,\dagger}$, and $\vecw_n$, $n=1,\ldots,N$,
% are the eigenvectors of $\Am$ which comprise the columns of the matrix   
% $\Wm$. From equation \eqref{eq:Eig_Block_Matrices}, we see that
% $\Wm^\dagger=\Rm^\dagger\Pm^T$. Since $\Gamma=\Pm\Gm\Pm^T$ and $\Pm$ is an
% orthogonal matrix, this implies that
% $\Wm^\dagger\Gamma=\Rm^\dagger\Gm\Pm^T$. It follows from the block forms 
% of the matrices $\Gm$, $\Pm$, and $\Rm$  in equations
% \eqref{eq:Block_Matrices} and \eqref{eq:Eig_Block_Matrices},
% that $\Wm^\dagger\Gamma$ is given by     
% %
% \begin{align}\label{eq:Block_g_vectors}
% %
% \Wm^\dagger\Gamma=\Rm^\dagger\Gm\Pm^T
% %
% =\left[
%   \begin{array}{ccc}
%     \Ib_{00}&\Om_{01}\\
%     \Om_{10}&\Rm_{11}^\dagger  
%     \end{array}
% \right]
% \left[
%   \begin{array}{ccc}
%     \Om_{00}&\Om_{01}\\
%     \Om_{10}&\Ib_{11}   
%     \end{array}
% \right]
% \left[
%   \begin{array}{cc}
%     \Pm_0^T\\
%     \Pm_1^T
%     \end{array}
% \right]
% =
% \left[
%   \begin{array}{cc}
%     \Om_{0N}\\
%     \Rm_{11}^\dagger\Pm_1^T   
%     \end{array}
% \right],
% \end{align}
% %
% where $\Om_{0N}$ is a matrix of zeros of size $N_0\times N$.
% It follows from $\vecg_j=-\Gamma\Hm\vece_j$ and
% equation~\eqref{eq:Block_g_vectors} that $\vecw_n^{\,\dagger}\vecg_j=0$ for
% all $n=1,\ldots,N_0$. This and equation~\eqref{eq:Quadratic_W} imply that 
% $[\Qc_n\vecg_j\bcdot\vecg_k]=0$ for all $n=1,\ldots,N_0$, as claimed.





 


\section{Discrete equivalence of the effective parameter
  problems} \label{app:Discrete_Equivalence}
%
In \secref{sec:Integral_Reps_Sobolev}, we provided Stieltjes
integral representations for the symmetric $\Sg^*$ and antisymmetric
$\Ag^*$ parts of the effective diffusivity matrix $\Dg^*$,
associated with an incompressible fluid velocity field. A discrete
version of this mathematical framework was formulated in 
\secref{sec:Matrix_Sobolev}. An alternate approach to the effective
parameter problem was formulated in
\appref{app:Curl_Free}, and its discrete version was
formulated in \appref{app:Matrix_Formulation_Curl}.  In this section, we
demonstrate that the discrete versions of these effective parameter
problems yield equivalent spectral representations of $\Dg^*$ when the
matrix $\nabla$ is of full-rank, as in the case of Dirichlet boundary
conditions, so that the matrix Laplacian is
invertible. In \appref{app:Eigenvalue_method}, this result is extended
to the setting where $\nabla$ is rank-deficient, as in the case of
periodic boundary conditions. 



Let $\nabla=\Um\Sigma\Vm^T$ be the singular value decomposition (SVD)
of the matrix $\nabla$ of size $N\times K$, where $K=L^d$ and
$N=Kd$. Here, $\Sigma=\text{diag}(\sigma_1,\ldots,\sigma_K)$, where
$0\leq\sigma_1\leq\cdots\leq\sigma_K\,,$ and the matrices $\Um$ and
$\Vm$ are of size $N\times K$ and $K\times K$, respectively, and 
satisfy~\cite{Demmel:1997} 
%
\begin{align}\label{eq:Inverse_U_V}
  \Um^T\Um=\Ib,
  \quad
  \Vm^T\Vm=\Vm\Vm^T=\Ib,
\end{align}
%
where $\Ib$ is the $K\times K$ identity matrix. The columns of $\Um$ are called left
singular vectors, the columns of $\Vm$ are called right singular
vectors, and the $\sigma_i$ are called singular values.



It follows from $\nabla=\Um\Sigma\Vm^T$ and equation~\eqref{eq:Inverse_U_V} that
the spectral decomposition of the negative matrix Laplacian $\nabla^T\nabla$ is
given by $\nabla^T\nabla=\Vm\Sigma^2\Vm^T$~\cite{Demmel:1997}. We assume that $\nabla$ is
of full-rank so that $\sigma_i>0$ for all $i=1,\ldots,K$. This implies that
$\Sigma^{-1}$ exists so that the matrix Laplacian is invertible. In this
case, it follows from $\nabla=\Um\Sigma\Vm^T$ and
equation~\eqref{eq:Inverse_U_V} that the projection matrix
$\Gamma=\nabla(\nabla^T\nabla)^{-1}\nabla^T$ is given by  
%
\begin{align}\label{eq:Gamma_U}
  \Gamma=\Um\Um^T,
\end{align}
%
which is a $N\times N$ symmetric projection matrix satisfying
$\Gamma^2=\Gamma$ and $\Gamma\nabla=\nabla$.
%(see equation~\eqref{eq:Inverse_U_V}).
A key property of the 
SVD of the \emph{full-rank} matrix $\nabla$ is
that its range is spanned by the columns of
$\Um$~\cite{Demmel:1997}, hence $\Gamma=\Um\Um^T$ projects subspaces of
$\mathbb{R}^N$ onto the range of $\nabla$. 

 


From equations~\eqref{eq:Inverse_U_V} and~\eqref{eq:Gamma_U}, we
can write the eigenvalue problem $-\imath\Gamma\Hm\Gamma\vecw_n=\lambda_n\vecw_n$ discussed
in \appref{app:Matrix_Formulation_Curl} as   
%
\begin{align}\label{eq:Equivalence_Curl}
  [-\imath\Um^T\Hm\Um][\Um^T\vecw_n]=\lambda_n[\Um^T\vecw_n].
\end{align}
%
Now consider the generalized eigenvalue problem
$-\imath\nabla^T\Hm\nabla\vecz_n=\alpha_n\nabla^T\nabla\vecz_n$ discussed in
\secref{sec:Matrix_Sobolev} and recall that $\nabla=\Um\Sigma\Vm^T$ and
$\nabla^T\nabla=\Vm\Sigma^2\Vm^T$. Since 
$\Sigma$ is invertible, by equation~\eqref{eq:Inverse_U_V} we can write
this generalized eigenvalue problem as the following standard
eigenvalue problem 
%
\begin{align}\label{eq:Equivalence_Sobolev}
  [-\imath\Um^T\Hm\Um][\Sigma\Vm^T\vecz_n]=\alpha_n[\Sigma\Vm^T\vecz_n].
\end{align}
%
Comparing the formulas in equations~\eqref{eq:Equivalence_Curl} 
and~\eqref{eq:Equivalence_Sobolev} indicates that spectrum associated
with each of these eigenvalue problems is identical, $\alpha_n=\lambda_n$, and
that the eigenvectors are related by
$\Um^T\vecw_n=\Sigma\Vm^T\vecz_n$. Since $\Gamma$ is a projection matrix,
$\Gamma^2=\Gamma$, the eigenvalue problem $\Gamma\Hm\Gamma\vecw_n=\imath\lambda_n\vecw_n$ can be
written as $\Gamma\Hm\Gamma[\Gamma\vecw_n]=\imath\lambda_n[\Gamma\vecw_n]$, which implies that
$\Gamma\vecw_n=\vecw_n$.
% \begin{align}\label{eq:Gamma_wn}
%   \Gamma\vecw_n=\vecw_n.
% \end{align}
%
% This can also be seen from
% equations~\eqref{eq:Eig_Block_Matrices}
% and~\eqref{eq:Block_g_vectors} with $\Um=\Pm_1$ (as discussed
% above), and equations~\eqref{eq:Gamma_U} and~\eqref{eq:Inverse_U_V}.
Consequently, applying the matrix $\Um$ to 
both sides of the formula $\Um^T\vecw_n=\Sigma\Vm^T\vecz_n$ and recalling
that $\Gamma=\Um\Um^T$ and $\nabla=\Um\Sigma\Vm^T$ we have 
%
\begin{align}\label{eq:Evector_relation}
  \vecw_n=\nabla\vecz_n.
\end{align}
%


In the following lemma, we make precise the correspondence
between the standard eigenvalue problem $-\imath\Gamma\Hm\Gamma\vecw_n=\lambda_n\vecw_n$
and the generalized eigenvalue problem
$-\imath\nabla^T\Hm\nabla\vecz_n=\alpha_n\nabla^T\nabla\vecz_n$, as well as the associated spectral measures
in equations~\eqref{eq:Disc_Spec_Measure_Matrix}
and~\eqref{eq:Disc_Spec_Measure_Matrix_Sobolev}, respectively.  
%
\begin{lemma}\label{lem:Spectral_Equivalence}
Consider the standard eigenvalue problem and the generalized
eigenvalue problem given, respectively, in
equations~\eqref{eq:Standard_Eval} and~\eqref{eq:Generalized_Eval}
below 
%
\begin{align}
  \label{eq:Standard_Eval}
  -\imath\Gamma\Hm\Gamma\vecw_n&=\lambda_n\vecw_n,
  \\
  \label{eq:Generalized_Eval}
  -\imath\nabla^T\Hm\nabla\vecz_n&=\lambda_n\nabla^T\nabla\vecz_n.
\end{align}
%
Let $\nabla=\Um\Sigma\Vm^T$ be the SVD of the matrix
$\nabla$, which we assume to be of full-rank. Then
equation~\eqref{eq:Standard_Eval} implies and is implied by
equation~\eqref{eq:Generalized_Eval}, with $\vecw_n$ and $\vecz_n$
related as in equation~\eqref{eq:Evector_relation}. This
implies that the spectrum associated with each of these 
eigenvalue problems is identical. Moreover, 
the spectral weights in equations~\eqref{eq:Quadratic_W}
and~\eqref{eq:Sobolev_weights}
are identical; specifically
%
\begin{align}\label{eq:Discrete_Weight_Equivalence}
  \Qc_n\Gamma\Hm\vece_j\bcdot\Gamma\Hm\vece_k=\nabla\Qm_n[\nabla^T\nabla]^{-1}\vecu_j\bcdot\nabla[\nabla^T\nabla]^{-1}\vecu_k\,. 
\end{align}
%
This, in turn, implies that the associated spectral measures in
equations~\eqref{eq:Disc_Spec_Measure_Matrix} and
~\eqref{eq:Disc_Spec_Measure_Matrix_Sobolev} are identical. 
\end{lemma}
%

\noindent\textbf{Proof of \lemref{lem:Spectral_Equivalence}}\\
%
Recall that $\nabla=\Um\Sigma\Vm^T$, $\nabla^T\nabla=\Vm\Sigma^2\Vm^T$, and $\Gamma=\Um\Um^T$, where
$\Sigma$ is invertible,  and the matrices $\Vm$ and $\Um$ satisfy
equation~\eqref{eq:Inverse_U_V}. 
First consider equation~\eqref{eq:Standard_Eval} written as in
equation~\eqref{eq:Equivalence_Curl},
$[-\imath\Um^T\Hm\Um][\Um^T\vecw_n]=\lambda_n[\Um^T\vecw_n]$. Since the matrix
$\Sigma$ is invertible and $\Vm^T\Vm=\Ib$, we can rewrite 
equation~\eqref{eq:Equivalence_Curl} as  
%
\begin{align}
  \Vm\Sigma[-\imath\Um^T\Hm\Um](\Sigma\Vm^T)(\Vm\Sigma^{-1})[\Um^T\vecw_n]
     =\lambda_n(\Vm\Sigma^2\Vm^T)(\Vm\Sigma^{-1})[\Um^T\vecw_n],
\end{align}
%
which is precisely equation~\eqref{eq:Generalized_Eval} written in
terms of $\nabla=\Um\Sigma\Vm^T$ with $\vecz_n=\Vm\Sigma^{-1}\Um^T\vecw_n$. This
formula for $\vecz_n$, equation~\eqref{eq:Inverse_U_V}, and the formula
$\Gamma\vecw_n=\vecw_n$ above equation~\eqref{eq:Evector_relation} imply
that $\vecw_n=\Um\Sigma\Vm^T\vecz_n=\nabla\vecz_n$, which is the
formula in~\eqref{eq:Evector_relation}. Now consider 
equation~\eqref{eq:Generalized_Eval} written as
in~\eqref{eq:Equivalence_Sobolev},
$[-\imath\Um^T\Hm\Um][\Sigma\Vm^T\vecz_n]=\lambda_n[\Sigma\Vm^T\vecz_n]$. Since
$\Um^T\Um=\Ib$, we can rewrite equation~\eqref{eq:Equivalence_Sobolev}
as   
%
\begin{align}
  \Um[-\imath\Um^T\Hm\Um](\Um^T\Um)[\Sigma\Vm^T\vecz_n]=\lambda_n\Um[\Sigma\Vm^T\vecz_n],
\end{align}
%
which is precisely~\eqref{eq:Standard_Eval} written in
terms of $\Gamma=\Um\Um^T$ with $\vecw_n=\Um\Sigma\Vm^T\vecz_n=\nabla\vecz_n$. 


We now establish
equation~\eqref{eq:Discrete_Weight_Equivalence}.
From the formula $\vecu=\bnabla\bcdot\Hm$ in~\eqref{eq:u_DH}, for the
continuum setting,  
we have that
$u_j=(\bnabla\bcdot\Hm)\bcdot\vece_j=\bnabla\bcdot(\Hm\vece_j)$. Since
$\nabla=\Um\Sigma\Vm^T$, the discrete version of this formula is given by
%
\begin{align}\label{eq:vecuj_Full_Rank}
  \vecu_j=-\nabla^T\Hm\vece_j=-\Vm\Sigma\Um^T\Hm\vece_j.
\end{align}
%
From $\nabla=\Um\Sigma\Vm^T$ and
$(\nabla^T\nabla)^{-1}=\Vm\Sigma^{-2}\Vm^T$ we have
$\nabla(\nabla^T\nabla)^{-1}=\Um\Sigma^{-1}\Vm^T$. Consequently,
$\Gamma=\Um\Um^T$, and equations~\eqref{eq:Inverse_U_V}
and~\eqref{eq:vecuj_Full_Rank}, yield     
$\nabla(\nabla^T\nabla)^{-1}\vecu_j=-\Gamma\Hm\vece_j$. Equation~\eqref{eq:Discrete_Weight_Equivalence}
now follows from the formula $\vecw_n=\nabla\vecz_n$
in~\eqref{eq:Evector_relation}
%
\begin{align}\label{eq:Discrete_Weight_Equivalence_Proof}  
  \vecw_n\vecw_n^\dagger\nabla(\nabla^T\nabla)^{-1}\vecu_j\bcdot\nabla(\nabla^T\nabla)^{-1}\vecu_k&=
  [(\nabla^T\nabla)^{-1}\nabla^T\vecw_n][(\nabla^T\nabla)^{-1}\nabla^T\vecw_n]^\dagger\vecu_j\bcdot\vecu_k
  \notag\\
  &=[(\nabla^T\nabla)^{-1}\nabla^T\nabla\vecz_n][(\nabla^T\nabla)^{-1}\nabla^T\nabla\vecz_n]^\dagger\vecu_j\bcdot\vecu_k
  \notag\\
  &=\vecz_n\vecz_n^\dagger\vecu_j\bcdot\vecu_k,
\end{align}
%
where we have used that the inverse of a symmetric matrix is also
symmetric~\cite{Horn_Johnson-1990}.
The equivalence of equations~\eqref{eq:Discrete_Weight_Equivalence}
and~\eqref{eq:Discrete_Weight_Equivalence_Proof} now follows from
equations~\eqref{eq:Matrix_Rep_Spec_Theorem},~\eqref{eq:Matrix_Rep_Spec_Theorem_Sobolev},
and~\eqref{eq:Sobolev_weights}. This concludes our proof of
\lemref{lem:Spectral_Equivalence} $\Box\,.$   



We conclude this section with a discussion regarding numerical
computations of the effective diffusivity matrix $\Dg^*$. The approach
discussed in this section and the projection method discussed in
\thmref{thm:Projection_Method} demonstrate that a hybrid of these two
approaches leads to the most efficient algorithm for numerical
computations of spectral representations for $\Dg^*$ --- combining the
computational advantages of both  the methods discussed in
\appref{app:Matrix_Formulation_Curl} and
\secref{sec:Matrix_Sobolev}. More specifically, in the full 
rank setting, the spectral measure underlying the discrete integral
representation for $\Dg^*$ was calculated in
\appref{app:Matrix_Formulation_Curl} in terms of the \emph{standard}
eigenvalue problem $-\imath\Gamma\Hm\Gamma\vecw_m=\lambda_n\vecw_n$, where the matrix
$-\imath\Gamma\Hm\Gamma$ is of size $N\times N$. In \secref{sec:Matrix_Sobolev}, 
$\Dg^*$ was calculated in terms of the \emph{generalized} eigenvalue 
problem $-\imath\nabla^T\Hm\nabla\vecz_n=\lambda_n\nabla^T\nabla\vecz_n$, involving the $K\times K$ matrices
$-\imath\nabla^T\Hm\nabla$ and $\nabla^T\nabla$. Since $\nabla^T=(\nabla_1^T,\ldots,\nabla_d^T)$ is of size $K\times N$
we have that $K=N/d<N$. However, the generalized eigenvalue problem is
more computationally intensive than the standard eigenvalue
problem~\cite{Parlett:1980}. For the case of randomly perturbed flows,
many statistical iterations are necessary to compute $\Dg^*$ and the
efficiency of the numerical algorithm is key. Neither of the methods
discussed in \appref{app:Matrix_Formulation_Curl} and
\secref{sec:Matrix_Sobolev} are optimal.




The projection method developed in \thmref{thm:Projection_Method}
demonstrates that, by first computing the standard eigenvalue
decomposition of the non-random matrix $\Gamma$, the spectral statistics of
the eigenvalue problem $-\imath\Gamma\Hm\Gamma\vecw_n=\lambda_n\vecw_n$ can then be
obtained by repeatedly computing the standard eigenvalue 
decomposition of smaller matrices. They are of size $K\times K$ by
equations~\eqref{eq:Eig_Decomp_Gamma}--\eqref{eq:Eig_Block_Matrices}
and~\eqref{eq:Gamma_U}. We emphasize that in the setting of full-rank
$\nabla$, the parameter $K$ in this section and $N_1$ in
\thmref{thm:Projection_Method} both denote the rank of the matrix
$\Gamma$, i.e., $K=N_1$. Note, computing the matrix
$\Gamma=\nabla(\nabla^T\nabla)^{-1}\nabla^T$ itself involves the cost
of numerically solving $N$ linear systems of size $K\times
K$. Alternatively, the proof of \lemref{lem:Spectral_Equivalence}
illustrates that by first computing the SVD of the matrix gradient,
$\nabla=\Um\Sigma\Vm^T$, the spectral statistics of the generalized
eigenvalue problem
$-\imath\nabla^T\Hm\nabla\vecz_n=\lambda_n\nabla^T\nabla\vecz_n$ can
then be obtained by repeatedly computing the \emph{standard}
eigenvalue decomposition of the matrix $-\imath\Um^T\Hm\Um$ which is
of size $K\times K$. When $N$ is large, these equivalent methods are
more numerically efficient than the other approaches discussed in
\appref{app:Matrix_Formulation_Curl} and \secref{sec:Matrix_Sobolev}.    



In \appref{app:Eigenvalue_method}, we
generalize~\lemref{lem:Spectral_Equivalence} to the case where
$\nabla$ is rank-deficient with rank $K_1$ satisfying $K_1<K$. Our
analysis demonstrates that the two formulations discussed in
\appref{app:Matrix_Formulation_Curl} and \secref{sec:Matrix_Sobolev}
yield equivalent spectral representations of the effective diffusivity
matrix $\Dg^*$ in this rank-deficient setting. Moreover, en route, a
more efficient numerical algorithm for computations of $\Dg^*$ is
revealed. More specifically, we demonstrate by first computing
the SVD of the matrix gradient, $\Dg^*$ can be computed via a
\emph{standard} eigenvalue problem for matrices of size $K_1\times
K_1\,.$ Consequently, the rank deficiency of the problem actually
increases the numerical efficiency of computations.  







\section{Rank deficiency and a unifying standard eigenvalue
  problem} \label{app:Eigenvalue_method}  
%


In \secref{sec:Matrix_Sobolev} and \appref{app:Matrix_Formulation_Curl} we
provided two discrete, matrix formulations of the effective parameter
problem for advection enhanced diffusion, which led to discrete
Stieltjes integral representations for the effective diffusivity
matrix $\Dg^*$ involving spectral measures of Hermitian
matrices. These two formulations assume that the $N\times K$ 
matrix $\nabla$ is of full-rank $K$ so 
that the negative matrix Laplacian $\nabla^T\nabla$ is
invertible. \lemref{lem:Spectral_Equivalence} of
\appref{app:Discrete_Equivalence} shows that, given this  
condition, the two formulations yield equivalent spectral
representations of $\Dg^*$. This analysis also demonstrates that a
hybrid of the two formulations leads to a numerical algorithm for
computing $\Dg^*$ that is more efficient than the numerical algorithms
for either approach. In this 
section, we generalize~\lemref{lem:Spectral_Equivalence} to the case where
$\nabla$ is rank-deficient, with rank $K_1$ satisfying $K_1<K$. We
demonstrate that the two 
formulations are equivalent in this rank-deficient setting and we also
derive an efficient hybrid numerical algorithm for computations of
$\Dg^*$. This framework is used in \secref{sec:Num_Results} to compute
$\Dg^*$ for  
% randomly perturbed
periodic flows, for which the matrix $\nabla$ with
periodic boundary conditions is rank-deficient.  









Toward this goal, let $\nabla=\Um\Sigma\Vm^T$ be the SVD of the
matrix gradient $\nabla$ of size $N\times K$, introduced in 
\secref{sec:Matrix_Sobolev} and \appref{app:Discrete_Equivalence}. We
assume that $\nabla$ is rank 
deficient so that $\nabla^T\nabla=\Vm\Sigma^2\Vm^T$ is singular,
with $K_1$ non-zero eigenvalues and $K_0=K-K_1$ zero eigenvalues, and
write 
%
\begin{align}\label{eq:SVD_Decomposition}
  \Um=[\Um_0\;\Um_1],  
  \quad
  \Sigma = \text{diag}(\Om_{00},\Sigma_1),
  % \Sigma=
%   \left[
%   \begin{array}{ccc}
%     \Om_{00}&\Om_{01}\\
%     \Om_{10}&\Sigma_1   
%     \end{array}
% \right],
\quad
  \Vm=[\Vm_0\;\Vm_1].
\end{align}  
%
Here, we denote $\Om_{ab}\,,$ $a,b=0,1$, to be matrices of zeros of size
$K_a\times K_b\,,$ $\Um_a$ is of size $N\times K_a\,,$ $\Vm_a$
is $K\times K_a\,,$  and $\Sigma_1$ is a $K_1\times K_1$ diagonal,
\emph{invertible} matrix. By equation~\eqref{eq:Inverse_U_V} the
matrices $\Um_1$ and $\Vm_1$ satisfy $\Um_1^T\Um_1=\Ib_1$ and
$\Vm_1^T\Vm_1=\Ib_1\,,$ 
% %
% \begin{align}\label{eq:Inverse_U1_V1}
%   \Um_1^T\Um_1=\Ib_1,
%   \quad
%   \Vm_1^T\Vm_1=\Ib_1,
% \end{align}
% %
where $\Ib_1$ is the $K_1\times K_1$ identity matrix, but
$\Vm_1\Vm_1^T\neq\Ib$. Due to the blocks of zeros in the matrix
$\Sigma$ in equation~\eqref{eq:SVD_Decomposition}, we can write the 
matrix gradient as $\nabla=\Um_1\Sigma_1\Vm_1^T$ and the 
negative matrix Laplacian as
$\nabla^T\nabla=\Vm_1\Sigma_1^2\Vm_1^T$. 
An important property  
of the SVD of the matrix $\nabla$ is that its null
space is spanned by the columns of $\Vm_0$ and its range is spanned by
the columns of $\Um_1$~\cite{Demmel:1997}. We emphasize that in the
setting where $\nabla$ is full-rank, we have $\Um_1=\Um$,
$\Sigma_1=\Sigma$, and $\Vm_1=\Vm$ satisfies
$\Vm_1\Vm_1^T=\Vm_1^T\Vm_1=\Ib$. 




Consider the cell problem in
equation~\eqref{eq:Random_Cell_Prob} 
written, via~\eqref{eq:u_DH} and 
$[\bnabla\bcdot\Hm\,]\bcdot\bnabla\varphi =\bnabla\bcdot[\Hm\bnabla\varphi]$,
as $\bnabla\bcdot\Hm\bnabla\chi_j+\varepsilon\Delta\chi_j=-u_j.$
% %
% \begin{align}\label{eq:Cell_H}
%   \bnabla\bcdot\Hm\bnabla\chi_j+\varepsilon\Delta\chi_j=-u_j.
% \end{align}
% %
Discretizing
this formula yields
%equation~\eqref{eq:Cell_H} yields
(see the discussion following equation~\eqref{eq:H_Hessian} for
details regarding the discretization of these differential operators
etc.), 
%
\begin{align}\label{eq:Cell_H_Discrete}
  \nabla^T\Hm\nabla\vecchi_j+\varepsilon\nabla^T\nabla\vecchi_j=\vecu_j,
\end{align}
%
where $\vecu_j$ is the discrete, vector representation of the $j$th
component of the fluid velocity field $u_j$, and similarly for $\vecchi_j$. 
Substituting the formula for $\vecu_j$ in~\eqref{eq:Cell_H_Discrete}
into the discrete version $\Dg^*_{jk}=\varepsilon\delta_{jk}+\langle\vecu_j\bcdot\vecchi_k\rangle$
of equation~\eqref{eq:Djk} yields    
%
\begin{align}\label{eq:Discrete_Djk}
  \Dg^*_{jk}=\Sg^*_{jk}+\Ag^*_{jk},
  \qquad
  \Sg^*_{jk}=\varepsilon(\delta_{jk}+\langle\nabla\vecchi_j\bcdot\nabla\vecchi_k\rangle),
  \quad
  \Ag^*_{jk}=\langle\nabla^T\Hm\nabla\vecchi_j\bcdot\vecchi_k\rangle,
\end{align}
%
where, as before, $\Sg^*_{kj}=\Sg^*_{jk}$ and
$\Ag^*_{kj}=-\Ag^*_{jk}$. We are now ready to state
the key result of this Appendix.



%
\begin{theorem}\label{thm:Spectral_Equivalence_Rank_Def}
%  
Let the matrix gradient $\nabla$ be rank-deficient and let
$\nabla=\Um_1\Sigma_1\Vm_1^T$ be its SVD.
%, as described above in detail in this Appendix.
Also, let $\Um_1^T\Hm\Um_1=\imath\Rm_1\Lambda_1\Rm_1^\dagger$
be the spectral decomposition of the antisymmetric matrix
$\Um_1^T\Hm\Um_1$.
%, where $\Lambda_1$ is a diagonal real-valued matrix
%and $\Rm_1$ is a unitary matrix.
Consider the discrete formulation of the effective parameter
problem developed in \secref{sec:Matrix_Sobolev}.
%In the rank-deficient setting,
We have the following generalization of
equation~\eqref{eq:Matrix_Rep_Spec_Theorem_Sobolev} 
%
\begin{align}\label{eq:Matrix_Rep_Spec_Theorem_Sobolev_U1}
  \sum_{n=1}^{K_1}\Qm^1_n=\Vm_1\Vm_1^T, \qquad
  \Qm^1_n=\vecz^1_n[\nabla\vecz^1_n]^{\,\dagger}\nabla,  \qquad
  \Qm^1_l \Qm^1_m=\Qm^1_l\,\delta_{l m},
\end{align}
%
where the matrices $\Qm^1_n$, $n=1,\ldots,K_1$, are self-adjoint with
respect to the \emph{discrete} inner-product
$\langle\cdot,\cdot\rangle_{1,2}$ defined by
$\langle\vecxi,\veczeta\rangle_{1,2}
=\langle\nabla\vecxi\bcdot\nabla\veczeta\rangle$,
i.e., 
$\langle\Qm^1_n\vecxi,\veczeta\rangle_{1,2}
=\langle\vecxi,\Qm^1_n\veczeta\rangle_{1,2}\,$ for 
$\vecxi,\veczeta\in\mathbb{C}^{K_1}$.  
Moreover, the generalization of the resolvent formula
$\vecchi_j=\Zm(\varepsilon\Ib+\imath\Lambda)^{-1}\Zm^{\,\dagger}\vecu_j$ 
in equation~\eqref{eq:Matrix_chij}
%of \secref{sec:Matrix_Sobolev}
is given by 
%
\begin{align}\label{eq:V1_chij}  
  \Vm_1^T\vecchi_j=
  \Vm_1^T\Zm_1(\varepsilon\Ib_1+\imath\Lambda_1)^{-1}\Zm_1^\dagger\vecu_j\,,
  \qquad
  \Zm_1=\Vm_1\Sigma_1^{-1}\Rm_1\,.
\end{align}
%





Now consider the discrete formulation of the effective parameter
problem developed in \appref{app:Matrix_Formulation_Curl}. Let
$\Gamma_1\Hm\Gamma_1=\imath\Wm_1\tilde{\Lambda}\Wm_1^\dagger$ be the
spectral decomposition of the antisymmetric matrix
$\Gamma_1\Hm\Gamma_1$, where $\Gamma_1=\Um_1\Um_1^T$,
$\tilde{\Lambda}$ is a diagonal real-valued matrix with
$\tilde{\lambda}_n^1$, $n=1,\ldots,N$, on its diagonal, and $\Wm_1$ is
a unitary matrix with columns $\vecw_n^1\,$. Then,
equations~\eqref{eq:Matrix_Rep_Spec_Theorem}
and~\eqref{eq:Disc_Spec_Measure_Matrix} hold with $\Qc_n$ and $\lambda_n$
replaced by $\Qc_n^1=[\vecw_n^1][\vecw_n^1]^\dagger$ and
$\tilde{\lambda}_n^1\,$. Also, the resolvent formula in
equation~\eqref{eq:Disc_Reslovent_CurL} holds with $\Wm$ replaced by
$\Wm_1$ and $\Lambda$ replaced by $\tilde{\Lambda}\,$. 


These two discrete formulations of the effective parameter problem are
related as follows. The diagonal eigenvalue matrices $\tilde{\Lambda}$
and $\Lambda_1$ are related by
$\tilde{\Lambda}=\text{diag}(\Om_{00},\Lambda_1)$. The eigenvector
matrices $\Wm_1$ and $\Zm_1$ are related by the following
generalization of equation~\eqref{eq:Evector_relation} (also
see~\eqref{eq:Eig_Block_Matrices}) 
%
\begin{align}\label{eq:Evector_relation_U1}
  \Wm_1=[\Pm_0\;\;\nabla\Zm_1], \qquad
  \nabla\Zm_1=\Um_1\Rm_1\,.
\end{align}
%
where the columns of $\Pm_0$ are eigenvectors of $\Gamma_1$ which span
its null space. Moreover, the following formulas generalize
equations~\eqref{eq:Quad_Proj_method},~\eqref{eq:Discrete_Weight_Equivalence},
and~\eqref{eq:Discrete_Weight_Equivalence_Proof},
% 
\begin{align}\label{eq:Measure_Weight_Equivalence_Gen}
 \Wm_1^\dagger\Gamma_1\Hm\vece_j\bcdot\Wm_1^\dagger\Gamma_1\Hm\vece_k
 =[\nabla\Zm_1]^\dagger\Hm\vece_j\bcdot[\nabla\Zm_1]^\dagger\Hm\vece_k
 =\Zm_1^\dagger\vecu_j\bcdot\Zm_1^\dagger\vecu_k\,.
\end{align}
%

In this rank-deficient setting, these two approaches both yield discrete
Stieltjes integral representations for the functional formulas
in~\eqref{eq:Discrete_Djk} for the symmetric $\Sg^*$ and 
antisymmetric $\Ag^*$ parts of the effective diffusivity matrix
$\Dg^*$. The two representations are equivalent by the relations
discussed above and are given by
equation~\eqref{eq:Discrete_Integrals_full_rank_Sobolev},
% %
% \begin{align}\label{eq:Discrete_Integrals_rank_deficient}
%   \Sg^*_{jk}/\varepsilon-\delta_{jk}=\sum_{n=1}^{K_1}
%       \frac{\Real[\,\overline{([\vecz^1_n]^\dagger\vecu_j)}
%                               ([\vecz^1_n]^\dagger\vecu_k)\,]}
%            {\varepsilon^2+[\lambda^1_n]^2}\,,
%   \qquad
%   \Ag^*_{jk}=\sum_{n=1}^{K_1}
%       \frac{\lambda^1_n\,\Imag[\,\overline{([\vecz^1_n]^\dagger\vecu_j)}
%                                          ([\vecz^1_n]^\dagger\vecu_k)\,]}
%            {\varepsilon^2+[\lambda^1_n]^2}\,.
% \end{align}
% %
with $\vecz_n$ and $\lambda_n$ replaced by
$\vecz_n^1=\Vm_1\Sigma_1^{-1}\vecr_n^1$ and $\lambda^1_n$,
$n=1,\ldots,K_1$, where $(\lambda^1_n,\vecr_n^1)$ are eigen-pairs of
matrix $-\imath\Um_1^T\Hm\Um_1$. The results 
discussed here also 
hold for the setting where $\nabla$ is of full-rank, and therefore
generalize the discrete mathematical frameworks developed in
\secref{sec:Matrix_Sobolev}, \appref{app:Matrix_Formulation_Curl}, and
\appref{app:Discrete_Equivalence}.   
%
\end{theorem}

% 

\noindent\textbf{Proof of \thmref{thm:Spectral_Equivalence_Rank_Def}.}
%
We first work with equation~\eqref{eq:Cell_H_Discrete} directly and
develop a mathematical framework which parallels the framework of
\secref{sec:Matrix_Sobolev} for the rank-deficient setting. We then
transform equation~\eqref{eq:Cell_H_Discrete} into a discrete analogue
of equation~\eqref{eq:Resolvent_Rep} written as
$\big(\varepsilon\Ib+\Gamma\Hm\Gamma\big)\nabla\vecchi_j=-\Gamma\Hm\vece_j$,
with a suitable 
generalization of the formula for the matrix $\Gamma$ 
in~\eqref{eq:Gamma_U}, and develop a mathematical framework 
which parallels the framework of \appref{app:Matrix_Formulation_Curl}. We
then generalize \lemref{lem:Spectral_Equivalence} of
\appref{app:Discrete_Equivalence}, establishing the equivalence of
these two formulations for the rank-deficient setting and, en route,
derive a hybrid numerical algorithm for computing spectral
representations of $\Dg^*$ which is more efficient than both of the
other numerical algorithms.  




Since
%Due to the blocks of zeros in
$\Sigma=\text{diag}(\Om_{00},\Sigma_1)$  
%in $\Sigma$ in~\eqref{eq:SVD_Decomposition},
%the matrix elements of $\nabla$ and $\nabla^T\nabla$
%do not depend on $\Um_0$ nor $\Vm_0$ and can be written as
we have
$\nabla=\Um_1\Sigma_1\Vm_1^T$ and 
$\nabla^T\nabla=\Vm_1\Sigma^2_1\Vm_1^T$,
so
%Consequently,
the cell
problem in~\eqref{eq:Cell_H_Discrete} can be written 
$[\Vm_1\Sigma_1][\Um_1^T\Hm\Um_1][\Sigma_1\Vm_1^T]\vecchi_j+\varepsilon\Vm_1\Sigma_1^2\Vm_1^T\vecchi_j=\vecu_j.$
% %
% \begin{align}\label{eq:Disc_Cell_ProB}
%   [\Vm_1\Sigma_1][\Um_1^T\Hm\Um_1][\Sigma_1\Vm_1^T]\vecchi_j+\varepsilon\Vm_1\Sigma_1^2\Vm_1^T\vecchi_j=\vecu_j.
% \end{align}
% %
The $K_1\times K_1$ antisymmetric matrix $\Um_1^T\Hm\Um_1$ has the
spectral decomposition
$\Um_1^T\Hm\Um_1=\imath\Rm_1\Lambda_1\Rm_1^\dagger$, where $\Lambda_1$
is a diagonal real-valued matrix and $\Rm_1$ is a unitary 
matrix, $\Rm_1^\dagger\Rm_1=\Rm_1\Rm_1^\dagger=\Ib_1\,.$
%It follows that
% $\nabla^T\Hm\nabla=\imath[\Vm_1\Sigma_1\Rm_1]\Lambda_1[\Vm_1\Sigma_1\Rm_1]^\dagger,$
% and
% $\nabla^T\nabla=[\Vm_1\Sigma_1\Rm_1][\Vm_1\Sigma_1\Rm_1]^\dagger,$ so
% %
% \begin{align}\label{eq:Discrete_Functionals}
%   \nabla^T\Hm\nabla=\imath[\Vm_1\Sigma_1\Rm_1]\Lambda_1[\Vm_1\Sigma_1\Rm_1]^\dagger,
%   \quad
%   \nabla^T\nabla=[\Vm_1\Sigma_1\Rm_1][\Vm_1\Sigma_1\Rm_1]^\dagger.
% \end{align}
% %
% Consequently, equation~\eqref{eq:Disc_Cell_ProB}
% the cell problem
% can be rewritten as
% $[\Vm_1\Sigma_1\Rm_1](\varepsilon\Ib_1+\imath\Lambda_1)[\Vm_1\Sigma_1\Rm_1]^\dagger\vecchi_j=\vecu_j$,
% %. This formula and equation~\eqref{eq:Inverse_U1_V1} together imply
% which yields
Equation~\eqref{eq:V1_chij} follows from these formulas, which is an
analogue of equation~\eqref{eq:Matrix_chij}. 
The formula $\nabla=\Um_1\Sigma_1\Vm_1^T$ and
equation~\eqref{eq:V1_chij}, in turn, imply that
$\nabla\vecchi_j=\Um_1\Rm_1(\varepsilon\Ib_1+\imath\Lambda_1)^{-1}
                   [\Vm_1\Sigma_1^{-1}\Rm_1]^\dagger\vecu_j.$
% %
% \begin{align}\label{eq:grad_chi_rank_def}
%   \nabla\vecchi_j=\Um_1\Rm_1(\varepsilon\Ib_1+\imath\Lambda_1)^{-1}
%                   [\Vm_1\Sigma_1^{-1}\Rm_1]^\dagger\vecu_j.
% \end{align}
% %


Substituting this formula for $\nabla\vecchi_j$
%in~\eqref{eq:grad_chi_rank_def}
into the formula for $\Sg^*_{jk}$ in
equation~\eqref{eq:Discrete_Djk} and using $\Um_1^T\Um_1=\Ib_1$ and
$\Rm_1^\dagger\Rm_1=\Ib_1$, yields the functional formula for
$\Sg^*_{jk}$ in~\eqref{eq:Functionals_kappa_alpha_Sobolev}
with $\Lambda$ replaced by $\Lambda_1$ and $\Zm$ replaced by
$\Zm_1=\Vm_1\Sigma_1^{-1}\Rm_1$.
%(see equation~\eqref{eq:Spectral_Discrete_Functionals} below).
%From equation~\eqref{eq:Discrete_Functionals}
We also have 
%
\begin{align}
  \langle\nabla^T\Hm\nabla\vecchi_j\bcdot\vecchi_k\rangle
   =\langle\imath[\Vm_1\Sigma_1\Rm_1]
           \Lambda_1[\Rm_1^\dagger\Sigma_1\Vm_1^T]\vecchi_j\bcdot\vecchi_k
     \rangle
   =\langle\imath[\Sigma_1\Rm_1\Lambda_1\Rm_1^\dagger\Sigma_1]
           \Vm_1^T\vecchi_j\bcdot\Vm_1^T\vecchi_k
    \rangle.
\end{align}
%
This formula, $\Rm_1^\dagger\Rm_1=\Ib_1$, $\Sigma_1^T=\Sigma_1$, and 
equations~\eqref{eq:Discrete_Djk} and~\eqref{eq:V1_chij}
yield the functional formula for $\Ag^*_{jk}$
in~\eqref{eq:Functionals_kappa_alpha_Sobolev} with $\Lambda$ replaced
by $\Lambda_1$ and $\Zm$ replaced by $\Zm_1=\Vm_1\Sigma_1^{-1}\Rm_1$.
%(see equation~\eqref{eq:Spectral_Discrete_Functionals} below).
% In summary,
% %
% \begin{align}\label{eq:Spectral_Discrete_Functionals}
%    \Sg^*_{jk}&=\varepsilon(\delta_{jk}+\langle(\varepsilon\Ib_1+\imath\Lambda_1)^{-1}\Zm_1^\dagger\vecu_j
%      \bcdot
%      (\varepsilon\Ib_1+\imath\Lambda_1)^{-1}\Zm_1^\dagger\vecu_k\rangle),     
%      \\
%    \Ag^*_{jk}&=
%    \langle\imath\Lambda_1(\varepsilon\Ib_1+\imath\Lambda_1)^{-1}\Zm_1^\dagger\vecu_j
%      \bcdot
%      (\varepsilon\Ib_1+\imath\Lambda_1)^{-1}\Zm_1^\dagger\vecu_k\rangle.
%      \notag
% \end{align}
% %
The quadratic form $\Zm_1^\dagger\vecu_j\bcdot\Zm_1^\dagger\vecu_k$ arising
in these functional formulas yields
%in~\eqref{eq:Spectral_Discrete_Functionals} yields
equation~\eqref{eq:Quadratic_Z} with $\vecz_n$ replaced by
$\vecz_n^1=\Vm_1\Sigma_1^{-1}\vecr_n^1$ and other appropriate
notational changes,
%the following analogue of equation~\eqref{eq:Quadratic_Z}  
% %
% \begin{align}\label{eq:Quadratic_Z1}
%   \Zm_1^\dagger\vecu_j\bcdot\Zm_1^\dagger\vecu_k
%   =\sum_{n=1}^{K_1}\overline{([\vecz^1_n]^\dagger\vecu_j)}([\vecz^1_n]^\dagger\vecu_k)
%   =\sum_{n=1}^{K_1}[\vecz^1_n][\vecz^1_n]^{\,\dagger}\vecu_j\bcdot\vecu_k,
%   \qquad
%   \vecz_n^1=\Vm_1\Sigma_1^{-1}\vecr_n^1,
% \end{align}
% %
where $\vecr_n^1$, $n=1,\ldots,K_1$, are the orthonormal eigenvectors of
the matrix $\Um_1^T\Hm\Um_1$ which comprise the columns of
$\Rm_1$.
% From $\nabla=\Um_1\Sigma_1\Vm_1^T$ and
% equations~\eqref{eq:Inverse_U1_V1} and~\eqref{eq:Quadratic_Z1} we have
% that $\nabla\vecz_n^1=\Um_1\vecr_n^1$.
The formula $\vecz_n^1=\Vm_1\Sigma_1^{-1}\vecr_n^1$ can be written
$\nabla\vecz_n^1=\Um_1\vecr_n^1$.
The orthogonality properties of $\Rm_1$ and $\Um_1$
% The orthogonality condition $\vecr^1_n\bcdot\vecr^1_m=\delta_{nm}$ and
% equation~\eqref{eq:Inverse_U1_V1}
then imply that the vectors 
$\vecz_n^1$ satisfy the Sobolev-type orthogonality condition in 
equation~\eqref{eq:Sobolev_orthogonality},
$\nabla\vecz^1_n\bcdot\nabla\vecz^1_m=\delta_{nm}$. Moreover, since
$\sum_nr_n^1[r_n^1]^\dagger=I_1$
% as the vectors $\vecr_n^1$ form an orthonormal basis for
% $\mathbb{C}^{K_1}$,
we also have 
equation~\eqref{eq:Matrix_Rep_Spec_Theorem_Sobolev_U1}.
It follows
% from equations~\eqref{eq:Spectral_Discrete_Functionals}
% and~\eqref{eq:Quadratic_Z1}
that the functional representations of
$\Sg^*_{jk}$ and $\Ag^*_{jk}$ in equation~\eqref{eq:Discrete_Djk}
have the discrete integral representations in
equation~\eqref{eq:Discrete_Integrals_full_rank_Sobolev} 
with $\vecz_n$ and $\lambda_n$ are replaced by
$\vecz_n^1$ and $\lambda^1_n$ and $(\lambda^1_n,\vecr_n^1)$ are
eigen-pairs of Hermitian matrix $-\imath\Um_1^T\Hm\Um_1$ of size
$K_1$. These summations have the same 
properties as the summations discussed in the proof of
\thmref{thm:Int_Rep_Sobolev_Matrix} and
equations~\eqref{eq:Signed_mu_Dis} and~\eqref{eq:Real_Spectral}.    
%  Here, as in equation~\eqref{eq:Integral_Rep_kappa*}, we
% have used the fact that the matrices $\nabla$ and $\Hm$, as well as the
% vectors $\vecchi_1$ and $\vecu_j$, and the molecular diffusivity $\varepsilon$
% are real valued, so that
% $\langle\nabla\vecchi_j\bcdot\nabla\vecchi_k\rangle=\langle\nabla\vecchi_k\bcdot\nabla\vecchi_j\rangle$ and
% $\langle\nabla^T\Hm\nabla\vecchi_j\bcdot\vecchi_k\rangle=\langle\vecchi_k\bcdot\nabla^T\Hm\nabla\vecchi_j\rangle$.  As in \secref{sec:Matrix_Sobolev}, we can
% write the summation formulas in
% equation~\eqref{eq:Discrete_Integrals_rank_deficient} as the integral
% representations in equation~\eqref{eq:Integral_Rep_kappa*}
% involving the discrete spectral measures 
% $\d\mu_{jk}(\lambda)$ in equation~\eqref{eq:Disc_Spec_Measure_Matrix_Sobolev}.
% Moreover, due to the eigenvalues and eigenvectors of the antisymmetric
% matrix $\Um_1^T\Hm\Um_1$ coming in complex conjugate pairs, as in
% equations~\eqref{eq:Signed_mu_Dis} and~\eqref{eq:Real_Spectral}, the
% sums in~\eqref{eq:Discrete_Integrals_rank_deficient} depend only on a
% subset of the index set.









We now argue that the mathematical framework developed so far in this
proof generalizes the full-rank case in \secref{sec:Matrix_Sobolev}. Indeed,
in the full-rank setting the matrix $\Vm_1=\Vm$ 
is orthogonal, $\Sigma_1=\Sigma$ is invertible, and $\Rm_1=\Rm$ is orthogonal,
so that the matrix $\Zm_1=\Zm$ defined in~\eqref{eq:V1_chij} is given
by $\Zm=\Vm\Sigma^{-1}\Rm$ 
% %
% \begin{align}\label{eq:Zmat}
%   \Zm=\Vm\Sigma^{-1}\Rm,
% \end{align}
% %
and is invertible with $\Zm^{-1}=\Rm^\dagger\Sigma\Vm^T$.
%and satisfies $\Zm^{-1}\Zm=\Zm\Zm^{-1}=\Ib$. Consequently,
% equation~\eqref{eq:Discrete_Functionals} implies that
Equation~\eqref{eq:Simultaneous_Diag} is satisfied with
$\Zm^{-\dagger}=\Vm\Sigma\Rm$ and $\Lambda=\Lambda_1$ which, in turn,
establishes the current rank-deficient setting generalizes the
full-rank setting summarized by   
equations~\eqref{eq:Strong_eval_prob}--\eqref{eq:Quadratic_Z}.
% are identical to
% equations~\eqref{eq:V1_chij},~\eqref{eq:Spectral_Discrete_Functionals},
% and~\eqref{eq:Quadratic_Z1}, respectively.
% This approach utilizes the special structure that is
% specific to the discretized cell problem in
% equation~\eqref{eq:Cell_H_Discrete} to reduce the generalized
% eigenvalue problem discussed in \secref{sec:Matrix_Sobolev} to a
% standard eigenvalue problem which is less computationally
% intensive. In the case of a randomly perturbed periodic flow, the
% SVD of the matrix $\nabla$ needs to be computed
% only once, while spectral statistics are obtained from repeated
% diagonalizing of the random Hermitian matrix $-\imath\Um_1^T\Hm\Um_1$ of
% size $K_1\times K_1$. As a consequence, the presence of a null space in the 
% matrix Laplacian actually decreases the computational cost of the
% method. 





We now generalize the mathematical framework developed in
\appref{app:Matrix_Formulation_Curl} to the case that the matrix $\nabla$ is rank
deficient. Using $\nabla=\Um_1\Sigma_1\Vm_1^T$,
$\nabla^T\nabla=\Vm_1\Sigma^2_1\Vm_1^T$, 
$\Um_1^T\Um_1=\Vm_1^T\Vm_1=\Ib_1$ 
%equation~\eqref{eq:Inverse_U1_V1}
and the invertibility of the matrix $\Sigma_1$, 
the cell problem in~\eqref{eq:Cell_H_Discrete} can be written as 
%we can rewrite equation~\eqref{eq:Disc_Cell_ProB} as
$\Um_1[\Um_1^T\Hm\Um_1][\Um_1^T\Um_1][\Sigma_1\Vm_1^T]\vecchi_j+\varepsilon\Um_1\Sigma_1\Vm_1^T\vecchi_j 
  =\Um_1\Sigma_1^{-1}\Vm_1^T\vecu_j.$
% %
% \begin{align}\label{eq:Disc_Cell_V1_Vu}
%   \Um_1[\Um_1^T\Hm\Um_1][\Um_1^T\Um_1][\Sigma_1\Vm_1^T]\vecchi_j+\varepsilon\Um_1\Sigma_1\Vm_1^T\vecchi_j 
%   =\Um_1\Sigma_1^{-1}\Vm_1^T\vecu_j.
% \end{align}
% %
Substituting $\vecu_j=-\nabla^T\Hm\vece_j$
%of~\eqref{eq:vecuj_Full_Rank} into equation~\eqref{eq:Disc_Cell_V1_Vu}
into this expression yields
%and using $\nabla=\Um_1\Sigma_1\Vm_1^T$ yields
$\big(\varepsilon\Ib+\Gamma_1\Hm\Gamma_1\big)\nabla\vecchi_j=\vecg^1_j$,
where $\Gamma_1=\Um_1\Um_1^T$ and $\vecg^1_j=-\Gamma_1\Hm\vece_j$
% %
% \begin{align}\label{eq:Discrete_Cell_Gamma1}
%   \big(\varepsilon\Ib+\Gamma_1\Hm\Gamma_1\big)\nabla\vecchi_j&=\vecg^1_j,
%   \qquad
%   \Gamma_1=\Um_1\Um_1^T,
%   \quad
%   \vecg^1_j=-\Gamma_1\Hm\vece_j,  
% \end{align}
% %
which is analogous to equation~\eqref{eq:Resolvent_Rep}. As in
\appref{app:Discrete_Equivalence}, the matrix $\Gamma_1=\Um_1\Um_1^T$
projects subspaces of $\mathbb{R}^N$ onto the \emph{range} of
$\nabla$.
%Since the matrix $\Gamma_1\Hm\Gamma_1$ is antisymmetric, it has the 
The antisymmetric matrix $\Gamma_1\Hm\Gamma_1$ has the 
spectral decomposition
$\Gamma_1\Hm\Gamma_1=\imath\Wm_1\tilde{\Lambda}\Wm_1^\dagger$, where
$\tilde{\Lambda}$ is a diagonal real-valued matrix and $\Wm_1$ is a unitary
matrix $\Wm_1^\dagger\Wm_1=\Wm_1\Wm_1^\dagger=\Ib$, which yields a
generalization of~\eqref{eq:Disc_Reslovent_CurL}. 
%This and equation~\eqref{eq:Discrete_Cell_Gamma1} yield the resolvent
%formula for $\nabla\vecchi_j$ in~\eqref{eq:Disc_Reslovent_CurL}, with
%corresponding notational changes.
From $\Gamma_1\nabla=\nabla$ we have
$\langle\nabla^T\Hm\nabla\vecchi_j\bcdot\vecchi_k\rangle=\langle\Gamma_1\Hm\Gamma_1\nabla\vecchi_j\bcdot\nabla\vecchi_k\rangle$. 
% In equation~\eqref{eq:Discrete_Djk}, write 
% $\langle\nabla^T\Hm\nabla\vecchi_j\bcdot\vecchi_k\rangle=\langle\Gamma_1\Hm\Gamma_1\nabla\vecchi_j\bcdot\nabla\vecchi_k\rangle$,
% where the second formula follows from $\Um_1^T\Um_1=\Ib_1$ and
% $\nabla=\Um_1\Sigma_1\Vm_1$ which imply that
% $\Gamma_1\nabla=\nabla$.
Exactly as in \appref{app:Matrix_Formulation_Curl}, these formulas
lead to generalizations of
equations~\eqref{eq:Matrix_Rep_Spec_Theorem}--\eqref{eq:Quad_Proj_method},
with appropriate notational changes for the rank-deficient setting.
% , with corresponding notational changes. This, in turn, leads to the
% integral representations for $\Sg^*_{jk}$ and $\Ag^*_{jk}$ in
% equation~\eqref{eq:Integral_Rep_kappa*}, involving a discrete spectral
% measure $\mu_{jk}$ associated with the function
% $\mu_{jk}(\lambda)=\langle\Qc^1(\lambda)\vecg^1_j\bcdot\vecg^1_j\rangle$, defined
% by~\eqref{eq:Disc_Spec_Measure_Matrix} with $\Qc(\lambda)$ and $\Qc_n$
% substituted by $\Qc^1(\lambda)$ and $\Qc_n^1$, respectively, where
% $\Qc_n^1=[\vecw_n^1][\vecw_n^1]^\dagger$ and 
% $\vecw_n^1$, $n=1,\ldots,K_1$, are the eigenvectors of the matrix $\Gamma_1\Hm\Gamma_1$
% which comprise the columns of $\Wm_1$.
In the case that the matrix $\nabla$
is of full-rank, we have $\Um_1=\Um$ hence $\Wm_1=\Wm$, which
establishes that the mathematical framework discussed in this
paragraph generalizes the mathematical framework in
\appref{app:Matrix_Formulation_Curl} for the full-rank setting.    


We now establish that these two different approaches provide
equivalent spectral representations for the effective diffusivity
matrix $\Dg^*$ for the rank-deficient setting.
% We now employ the projection method developed in
% \thmref{thm:Projection_Method} to generalize the mathematical
% framework in \appref{app:Discrete_Equivalence} to the rank-deficient
% setting, establishing the equivalence of the two approaches that
% follow from equations~\eqref{eq:Disc_Cell_ProB} 
% and~\eqref{eq:Discrete_Cell_Gamma1}.
% % We emphasize that the matrix $\Gamma_1$
% % defined in this section is identical to the matrix $\Gamma$ defined in
% % \thmref{thm:Projection_Method}, $\Gamma_1=\Gamma$, which were given different
% % notations to clarify our discussion here.
% Since $\Gamma_1=\Um_1\Um_1^T$ is a projection matrix, the projection method
% discussed in \thmref{thm:Projection_Method} can be directly applied to 
% equation~\eqref{eq:Discrete_Cell_Gamma1}.  However, the mathematical
% framework developed here provides deeper insight into
% equation~\eqref{eq:Eig_Block_Matrices} of the method. In particular, in
In equation~\eqref{eq:Gamma_spec_decomp} of
\thmref{thm:Projection_Method} we wrote the eigenvalue decomposition
$\Gamma=\Pm\Gm\Pm^T$, where $\Pm$ is an orthogonal matrix and $\Gm$ is
a diagonal matrix. Moreover, we wrote 
$\Pm=[\Pm_0\;\Pm_1]$, where the columns of the matrices $\Pm_0$ and
$\Pm_1$ are orthonormal eigenvectors that span the null space and
range of $\Gamma$, respectively. Since the eigenvalues $\gamma_n$
associated with the eigenvectors in the matrix $\Pm_1$ satisfy
$\gamma_n=1$, any linear combination of the corresponding eigenvectors is
also an eigenvector of $\Gamma$ with eigenvalue $\gamma_n=1$. Therefore, since
the orthonormal columns of the matrix $\Um_1$ span the range of 
$\Gamma_1=\Um_1\Um_1^T$,
%without loss of generality,
we may take
$\Pm_1=\Um_1$ so that $\Pm=[\Pm_0\;\Um_1]$. Consequently, we can
rewrite  equation~\eqref{eq:Eig_Decomp_PHP} as
$\Um_1^T\Hm\Um_1=\imath\Rm_{11}\Lambda_{11}\Rm_{11}^\dagger$. 
% %
% \begin{align}\label{eq:Eig_Decomp_PHP_U1}
%   \Um_1^T\Hm\Um_1=\imath\Rm_{11}\Lambda_{11}\Rm_{11}^\dagger.
% \end{align}
% %
%From equation~\eqref{eq:Eig_Decomp_PHP_U1} and the
%comment after equation~\eqref{eq:Disc_Cell_ProB}, we have that
Identifying the notations of \appref{app:Matrix_Formulation_Curl} with
the current section, we have
$\Rm_{11}=\Rm_1$ and $\Lambda_{11}=\Lambda_1$. This and
equation~\eqref{eq:Eig_Block_Matrices} establish that
$\tilde{\Lambda}=\text{diag}(\Om_{00},\Lambda_1)$.



We now establish that the spectral weights associated with each
approach are identical.
Using the notation of the present section,
equation~\eqref{eq:Eig_Block_Matrices} yields
% From the formula $\Pm=[\Pm_0\;\Um_1]$ and
% equation~\eqref{eq:Eig_Block_Matrices} with $\Rm_{11}=\Rm_1$ and $\Wm$
% redefined as $\Wm_1$, we have that
%$\Wm_1=\Pm\Rm=[\Pm_0\;\;\Um_1\Rm_1]$.
$\Wm_1=[\Pm_0\;\;\Um_1\Rm_1]$.
Consequently, from
$\nabla=\Um_1\Sigma_1\Vm_1^T$,
$\Um_1^T\Um_1=\Vm_1^T\Vm_1=\Ib_1$,
%equation~\eqref{eq:Inverse_U1_V1},
and the formula
$\Zm_1=\Vm_1\Sigma_1^{-1}\Rm_1$ in~\eqref{eq:V1_chij}, we have that
$\nabla\Zm_1=\Um_1\Rm_1$, implying
equation~\eqref{eq:Evector_relation_U1}, which is a generalization of
equation~\eqref{eq:Evector_relation}. Equation~\eqref{eq:Measure_Weight_Equivalence_Gen}
now follows from $\Gamma_1^T=\Gamma_1$, $\Gamma_1\Pm_0=\Om$,
$\Gamma_1\Um_1=\Um_1$, and the 
formula $\vecu_j=-\nabla^T\Hm\vece_j$ in~\eqref{eq:vecuj_Full_Rank}.
This establishes that the spectral weights associated with both of the 
approaches discussed in~\secref{sec:Matrix_Sobolev}
and~\appref{app:Matrix_Formulation_Curl} are identical which, in turn, 
establishes that both approaches provide equivalent spectral
representations of the effective diffusivity matrix $\Dg^*$.
This concludes our proof of \thmref{thm:Spectral_Equivalence_Rank_Def}
$\Box\,.$ 


In \secref{sec:Num_Results}, spectral representations of the symmetric
$\Sg^*$ and antisymmetric 
$\Ag^*$ parts of the effective diffusivity matrix $\Dg^*$ are computed
for various periodic flows, and spectral characteristics are related
to flow geometry and transport properties. We accomplish this by using  
equation~\eqref{eq:Discrete_Integrals_full_rank_Sobolev} with 
$\vecz_n$ and $\lambda_n$ replaced by
$\vecz_n^1=\Vm_1\Sigma_1^{-1}\vecr_n^1$ and $\lambda^1_n$,
respectively, where $(\lambda^1_n,\vecr_n^1)$ are eigen-pairs of
matrix $-\imath\Um_1^T\Hm\Um_1$. We emphasize that this matrix is of
size $K_1$ which is more than a factor of $d$ smaller than the matrix
$-\imath\Gamma_1\Hm\Gamma_1$. Moreover, we have established that the discrete
spectral measure at the heart of the integral representation for
$\Dg^*$ is given in terms of the \emph{standard} eigenvalues and
eigenvectors of $-\imath\Um_1^T\Hm\Um_1$, which is a less costly
numerical computation than the computation associated with the 
\emph{generalized} eigenvalue problem~\cite{Parlett:1980}
in~\eqref{eq:Strong_eval_prob_matrix}. Consequently, in the process of 
establishing the equivalence of the effective parameter problems
discussed in~\secref{sec:Matrix_Sobolev}
and~\appref{app:Matrix_Formulation_Curl} for the rank-deficient setting, we
have also developed a hybrid numerical algorithm that is more
efficient at computing the spectral representation of $\Dg^*$ than
both of the other approaches.  









\bibliographystyle{plain}
% Note the spaces between the initials

%\bibliography{jfm-instructions}
\bibliography{murphy}

\end{document}

% LocalWords:  clet anelastic co SIAP sec Functinals Num ij Prob Pavliotis eps
% LocalWords:  Strouhal Pe DH Rep jk diag Reslovent CurL symm mu Dis Decomp PHP
% LocalWords:  CMP chik eval prob Gen chij SVD wn thm ProB Djk Sjk Ajk Vu eig
% LocalWords:  mjk Matlab's condeig mldivide BC
